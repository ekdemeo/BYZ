\renewcommand{\chaptertitle}{Mateusza}
\begingl
\lettrine[loversize=1,lraise=-1.3]{1 }{}%
\gla
 976 {} 1078 2424 5547 5207 1138 5207 11
{} 11 1080 2464 1161 2464 1080 2384 1161 2384 1080 2455 2532 80 846
{} 1161 2455 1080 5329 2532 2196 1537 2283 1161 5329 1080 2074 1161 2074 1080 689
{} 1161 689 1080 284 1161 284 1080 3476 1161 3476 1080 4533
{} 1161 4533 1080 1003 1537 4477 1161 1003 1080 5601 1537 4503 1161 5601 1080 2421
{} 1161 2421 1080 1138 935 1161 1138 935 1080 4672 1537 3588 {} 3774
{} 1161 4672 1080 4497 1161 4497 1080 7 1161 7 1080 760
{} 1161 760 1080 2498 1161 2498 1080 2496 1161 2496 1080 3604
{} 1161 3604 1080 2488 1161 2488 1080 881 1161 881 1080 1478
{} 1161 1478 1080 3128 1161 3128 1080 300 1161 300 1080 2502
{} 1161 2502 1080 2423 2532 80 846 1909 3350 {} 897
{} 1161 3326 3350 {} 897 2423 1080 4528 1161 4528 1080 2216
{} 1161 2216 1080 10 1161 10 1080 1662 1161 1662 1080 107
{} 1161 107 1080 4524 1161 4524 1080 885 1161 885 1080 1664
{} 1161 1664 1080 1648 1161 1648 1080 3157 1161 3157 1080 2384
{} 1161 2384 1080 2501 435 3137 1537 3739 1080 2424 3004 5547
{} 3956 3767 1074 575 11 2193 1138 1074 1180 2532 575 1138 2193 3350 {} 897 1074 1180 2532 575 3350 {} 897 2193 5547 1074 1180
{} 1161 1083 2424 5547 2258 3779 1063 3384 846 3137 3423 2501 4250 2228 4905 846 4905 2147 2192 {} 1722 1064 1537 4151 40
{} 1161 2501 846 435 5607 1342 2532 3361 2309 846 3856 1014 2977 846 630
{} 1161 {} 846 5023 1760 2400 32 2962 5316 846 2596 3677 3004 2501 5207 1138 3361 5399 3880 {} 3137 4675 1135 1063 3588 1722 846 1080 2076 1537 4151 40
{} 1161 5088 5207 2532 2564 846 3686 2424 1063 846 4982 2992 846 575 846 266
{} 1161 5124 3650 1096 2443 4137 3588 4483 5259 2962 1223 4396 3004
{} 2400 3933 2192 1722 1064 2532 5088 5207 2532 2564 846 3686 1694 3739 2076 3177 3326 2257 2316
{} 1161 2501 1326 575 5258 4160 5613 4367 846 32 2962 2532 3880 846 1135
{} 2532 3756 1097 846 2193 3739 5088 5207 846 4416 2532 2564 846 3686 2424
//
\glb
 \underline{Zwój księgi} (o) narodzinach Jezusa Chrystusa, syna Dawida, syna Abrahama.
\vs{2} Abraham zrodził Izaaka, a~ Izaak zrodził Jakuba, a~ Jakub zrodził Judę i~ braci jego,
\vs{3} a~ Juda zrodził Faresa i~ Zarę z~ Tamary, a~ Fares zrodził Esroma, a~ Esrom zrodził Arama,
\vs{4} a~ Aram zrodził Aminadaba, a~ Aminadab zrodził Naassona, a~ Naasson zrodził Salmona,
\vs{5} a~ Salmon zrodził Booza z~ Rahab, a~ Booz zrodził Jobeda z~ Rut, a~ Jobed zrodził Jesaja,
\vs{6} a~ Jesaj zrodził Dawida– króla, a~ Dawid– król zrodził Salomona z~ tej, (która należała do) Uriasza,
\vs{7} a~ Salomon zrodził Roboama, a~ Roboam zrodził Abiasa, a~ Abias zrodził Asafa,
\vs{8} a~ Asaf zrodził Jozafata, a~ Jozafat zrodził Jorama, a~ Joram zrodził Ozjasza,
\vs{9} a~ Ozjasz zrodził Joatama, a~ Joatam zrodził Achaza, a~ Achaz zrodził Ezechiasza,
\vs{10} a~ Ezechiasz zrodził Manassesa, a~ Manasses zrodził Amona, a~ Amon zrodził Jozjasza,
\vs{11} a~ Jozjasz zrodził Jechoniasza oraz braci jego \underline{w czasie} przesiedlenia (do) Babilonu,
\vs{12} a~ po przesiedleniu (do) Babilonu Jechoniasz zrodził Salatiela, a~ Salatiel zrodził Zorobabela,
\vs{13} a~ Zorobabel zrodził Abiuda, a~ Abiud zrodził Eliakima, a~ Eliakim zrodził Azora,
\vs{14} a~ Azor zrodził Sadoka, a~ Sadok zrodził Achima, a~ Achim zrodził Eliuda,
\vs{15} a~ Eliud zrodził Eleazara, a~ Eleazar zrodził Mattana, a~ Mattan zrodził Jakuba,
\vs{16} a~ Jakub zrodził Józefa, męża Marii, z~ której \underline{został zrodzony} Jezus, nazywany Chrystusem.
\vs{17} Wszystkich więc pokoleń od Abrahama do Dawida– pokoleń czternaście, i~ od Dawida do przesiedlenia (do) Babilonu– pokoleń czternaście, i~ od przesiedlenia (do) Babilonu do Chrystusa– pokoleń czternaście.
\vs{18} A~ narodzenie Jezusa Chrystusa było takie: ponieważ matka jego, Maria, \underline{będąc zaślubiona} Józefowi, \doubleline{wcześniej, zanim} [-] \underline{się} oni \underline{zeszli,} \doubleline{znalazła się} \underline{tą, która ma} (dziecko) w~ łonie \doubleline{za sprawą} Ducha Świętego,
\vs{19} a~ Józef, jej mąż, będąc sprawiedliwym i~ nie chcąc jej \underline{wystawić na pośmiewisko,} chciał potajemnie ją uwolnić,
\vs{20} a~ (gdy) on to obmyślił, oto anioł Pana \underline{ukazał się} mu podczas snu, mówiąc: Józefie, synu Dawida, nie \doubleline{bój się} wziąć (do siebie) Marii, twojej żony, ponieważ \underline{to, co} w~ niej, zrodzone jest \doubleline{za sprawą} Ducha Świętego;
\vs{21} i~ urodzi syna, i~ nazwiesz go imieniem Jezus, ponieważ On wybawi lud swój od ich grzechów.
\vs{22} A~ to wszystko \underline{stało się,} aby \doubleline{wypełniło się} \underline{to, co} \doubleline{zostało powiedziane} poprzez Pana \underline{za pośrednictwem} proroka, mówiącego:
\vs{23} Oto panna \underline{„będzie mieć} w~ łonie” i~ urodzi syna, i~ nazwą go imieniem Emmanuel, co jest tłumaczone: \doubleline{razem z} nami Bóg.
\vs{24} A~ Józef, \underline{obudziwszy się} ze snu, uczynił, jak nakazał mu anioł Pana, i~ przyjął swoją żonę,
\vs{25} i~ nie poznawał jej, aż [-] urodziła syna swego pierworodnego; i~ nazwał go imieniem Jezus.
//
\endgl
\begingl
\lettrine[loversize=1,lraise=-1.3]{2 }{}%
\gla
 1161 1080 2424 1080 1722 965 2449 1722 2250 2264 935 2400 3097 575 395 3854 1519 2414
{} 3004 4226 2076 {} 5088 935 2453 1492 1063 846 792 1722 395 2532 2064 4352 846 4352
{} 1161 191 {} 191 935 2264 5015 2532 3326 846 3956 2414
{} 2532 4863 3956 749 2532 1122 {} 2992 4441 3844 846 4226 1080 5547
{} 1161 3588 2036 846 1722 965 2449 3779 1063 1125 1223 4396
{} 2532 4771 965 1093 2448 3760 1488 1646 1722 2232 2448 1063 1537 4675 1831 2233 3748 4165 2992 3450 2474
{} 5119 2264 2977 2564 3097 198 3844 846 {} 5550 5316 792
{} 2532 3992 846 1519 965 2036 4198 199 1833 4012 3813 1161 1875 {} 2147 518 3427 3704 2504 2064 {} 4352 846 4352
{} 1161 3588 191 935 4198 2532 2400 792 3739 1492 1722 395 4254 846 2193 2064 {} 2476 1883 {} 3757 2258 3813
{} 1161 1492 792 5463 5479 4970 3173
{} 2532 2064 1519 3614 3708 3813 3326 3137 846 3384 2532 4098 4352 846 4352 2532 455 846 2344 4374 846 1435 5557 2532 3030 2532 4666
{} 2532 5537 2596 3677 5537 {} 3361 344 4314 2264 1223 243 3598 402 1519 846 5561
{} 1161 402 846 402 2400 32 2962 5316 2596 3677 2501 3004 1453 3880 3813 2532 846 3384 2532 5343 1519 125 2532 2468 1563 2193 302 4671 2036 3195 1063 2264 2212 3813 {} 846 622
{} 1161 3588 1453 3880 3571 3588 3813 2532 846 3384 2532 402 1519 125
{} 2532 2258 1563 2193 5054 {} 2264 2443 4137 3588 4483 5259 2962 1223 4396 3004 1537 125 2564 5207 3450 {}
{} 5119 2264 1492 3754 1702 5259 3097 3029 2373 2532 649 {} 337 3956 3816 1722 965 2532 1722 3956 846 3725 575 1332 {} 2532 2736 {} 2596 5550 {} 3739 198 3844 3097
{} 5119 4137 {} 4483 5259 4396 2408 3004
{} 1722 4471 191 5456 2355 2532 2805 2532 4183 3602 4478 2799 846 5043 2532 3756 2309 3870 3754 {} 3756 1526 {}
{} 1161 5053 2264 5053 {} 2400 32 2962 2596 3677 5316 2501 1722 125
{} 3004 1453 3880 3813 2532 846 3384 2532 4198 1519 1093 2474 2348 1063 3588 2212 5590 3813
{} 1161 3588 1453 3880 3813 2532 846 3384 2532 2064 1519 1093 2474
{} 1161 191 3754 745 936 1909 2449 473 2264 846 3962 5399 1563 565 1161 5537 2596 3677 402 1519 3313 1056
{} 2532 2064 {} 2064 2730 1519 4172 3004 3478 3704 4137 3588 4483 1223 4396 3754 2564 3480
//
\glb
 A~ \underline{kiedy} Jezus \underline{urodził się} w~ Betlejem, \doubleline{w Judei,} za dni Heroda– króla, oto magowie ze wschodu przybyli do Jerozolimy,
\vs{2} mówiąc: Gdzie \underline{się znajduje} (ten) narodzony król Judejczyków? Zobaczyliśmy bowiem jego gwiazdę na wschodzie i~ przyszliśmy \doubleline{oddać} mu \doubleline{pokłon.}
\vs{3} A~ \underline{gdy} (to) \underline{usłyszał} król Herod, \doubleline{przestraszył się,} a~ \underline{razem z} nim cała Jerozolima.
\vs{4} I~ zebrawszy wszystkich arcykapłanów i~ \underline{znawców Pisma} (spośród) ludu, \doubleline{dowiadywał się} od nich, gdzie \underline{się rodzi} Mesjasz.
\vs{5} A~ oni powiedzieli mu: w~ Betlejem, \underline{w Judei,} tak bowiem \doubleline{jest napisane} przez proroka:
\vs{6} I~ ty, Betlejem, ziemio Judy, \underline{wcale nie} jesteś najmniejsze spośród rządców Judy, ponieważ z~ ciebie wyjdzie rządzący, który \doubleline{paść będzie} lud mój– Izraela.
\vs{7} Wtedy Herod, potajemnie wezwawszy magów, \underline{dokładnie dowiedział się} od nich (o) czasie \doubleline{ukazującej się} gwiazdy.
\vs{8} I~ wysławszy ich do Betlejem, powiedział: dotrzyjcie, dokładnie wypytajcie o~ dziecko. A~ gdy (je) znajdziecie, oznajmijcie mi, żebym \underline{i ja} przyszedł (i) \doubleline{oddał} mu \doubleline{pokłon.}
\vs{9} A~ oni, \underline{gdy wysłuchali} króla, wyruszyli; a~ oto gwiazda, którą dostrzegli na wschodzie, wyprzedzała ich, aż przybyła (i) stanęła ponad (miejscem), gdzie było dziecko.
\vs{10} A~ \underline{gdy dostrzegli} gwiazdę, \doubleline{uradowali się} radością bardzo wielką.
\vs{11} I~ \underline{gdy przyszli} do domu, zobaczyli dziecko, \doubleline{razem z} Marią, jego matką, i~ upadłszy, \underline{oddali} mu \underline{pokłon,} a~ otworzywszy swoje skarby, ofiarowali mu dary: złoto i~ kadzidło, i~ mirrę.
\vs{12} A~ \underline{gdy} podczas snu \underline{otrzymali pouczenie,} (aby) nie powracać do Heroda, poprzez inną drogę odeszli na swoje terytorium.
\vs{13} A~ \underline{gdy} oni \underline{odeszli,} oto anioł Pana \doubleline{ukazał się} podczas snu Józefowi, mówiąc: powstań, weź dziecko i~ jego matkę, i~ uciekaj do Egiptu, i~ bądź tam, aż [-] ci powiem. Zamierza bowiem Herod odszukać dziecko, (żeby) je stracić.
\vs{14} A~ on, \underline{gdy powstał,} wziął nocą to dziecko oraz jego matkę i~ odszedł do Egiptu.
\vs{15} I~ był tam \underline{aż do} końca (życia) Heroda, aby \doubleline{wypełniło się} \underline{to, co} \doubleline{zostało powiedziane} przez Pana \underline{za pośrednictwem} proroka, mówiącego: z~ Egiptu wywołałem syna mego. (Oz 11:1)
\vs{16} Wtedy Herod, \underline{gdy dostrzegł,} że \doubleline{został wyszydzony} przez magów, bardzo \underline{się rozgniewał} i~ posławszy (swoich ludzi), zgładził wszystkich chłopców w~ Betlejem oraz we wszelkich jego granicach, od dwulatków (począwszy) oraz poniżej (tego wieku), \doubleline{zgodnie z} czasem, (o) którym \underline{dokładnie się dowiedział} od magów.
\vs{17} Wtedy \underline{wypełniło się} (to, co) \doubleline{zostało powiedziane} przez proroka Jeremiasza, mówiącego:
\vs{18} W~ Rama \underline{był słyszany} głos: lament i~ płacz, i~ wielkie narzekanie; Rachel opłakuje swoje dzieci i~ nie chce \doubleline{zostać pocieszona,} ponieważ (ich już) nie ma. (Jr 31:15)
\vs{19} A~ \underline{gdy} Herod \underline{doszedł do końca} (życia), oto anioł Pana poprzez sen \doubleline{ukazał się} Józefowi w~ Egipcie,
\vs{20} mówiąc: Powstań, weź dziecko i~ jego matkę i~ wyrusz do ziemi Izraela. Umarli bowiem \underline{ci, którzy} szukają duszy dziecka.
\vs{21} A~ On, \underline{gdy powstał,} wziął dziecko i~ jego matkę, i~ przybył do ziemi Izraela.
\vs{22} Ale \underline{gdy usłyszał,} że Archelaos \doubleline{jest królem} w~ Judei– zamiast Heroda, swojego ojca– \underline{przestraszył się} tam wejść; a~ \doubleline{otrzymawszy pouczenie} podczas snu, odszedł ku obszarom Galilei.
\vs{23} A~ \underline{gdy} (tam) \underline{przybył,} zamieszkał w~ mieście, zwanym Nazaret, aby \doubleline{wypełniło się} \underline{to, co} \doubleline{zostało powiedziane} \underline{za pośrednictwem} proroków, że \doubleline{będzie nazwany} Nazarejczykiem.
//
\endgl
\begingl
\lettrine[loversize=1,lraise=-1.3]{3 }{}%
\gla
 1161 1722 1565 2250 3854 2491 910 2784 1722 2048 2449
{} 2532 3004 3340 1063 1448 932 3772
{} 2076 3778 1063 4483 5259 4396 2268 3004 5456 994 1722 2048 2090 3598 2962 2117 4160 846 5147
{} 846 1161 2491 2192 846 1742 575 2359 2574 2532 1193 2223 4012 846 3751 1161 5160 846 2258 200 2532 66 3192
{} 5119 1607 4314 846 2414 2532 3956 2449 2532 3956 4066 2446
{} 2532 907 5259 846 1722 2446 1843 846 266
{} 1161 1492 4183 5330 2532 4523 2064 1909 {} 846 908 2036 846 1081 2191 5101 5213 5263 5343 575 3195 3709
{} 4160 3767 2590 514 3341
{} 2532 3361 1380 {} 3004 1722 1438 2192 3962 11 1063 3004 5213 3754 2316 1410 1537 5130 3037 1453 5043 11
{} 1161 2235 2532 513 4314 4491 1186 2749 3956 3767 1186 3361 4160 2570 2590 1581 2532 906 1519 4442
{} 1473 3303 5209 907 1722 5204 1519 3341 1161 3588 2064 3694 3450 2076 2478 3450 3756 1510 2425 {} 3739 5266 941 846 5209 907 1722 4151 40
{} 3739 4425 1722 846 5495 2532 1245 846 257 2532 4863 846 4621 1519 596 1161 892 2618 762 4442 762
{} 5119 3854 2424 575 1056 1909 2446 4314 2491 907 5259 846 907
{} 1161 2491 1254 846 3004 1473 2192 5532 {} 907 5259 4675 2532 4771 2064 4314 3165
{} 1161 2424 611 2036 4314 846 863 737 3779 1063 4241 2076 4137 2254 3956 1343 5119 846 863
{} 2424 2532 907 2117 305 575 5204 2532 2400 455 846 3772 2532 1492 4151 2316 2597 5616 4058 2532 2064 1909 846
{} 2532 2400 5456 1537 3772 3004 3778 2076 3450 27 5207 1722 3739 2106
//
\glb
 A~ w~ owe dni przybył Jan Chrzciciel, \underline{który głosił} na pustkowiu Judei
\vs{2} i~ mówił: \underline{zmieniajcie myślenie,} ponieważ \doubleline{zbliżyło się} królestwo niebios.
\vs{3} Jest on bowiem przepowiedziany przez proroka Izajasza, \underline{który mówi:} Głos wołającego na pustkowiu: przygotujcie drogę Pana; prostymi czyńcie Jego ścieżki.
\vs{4} Sam zaś Jan miał swe odzienie z~ sierści wielbłąda oraz skórzany pas wokół swoich bioder, a~ pokarmem jego była szarańcza i~ dziki miód.
\vs{5} Wtedy wychodziła do niego Jerozolima i~ cała Judea, i~ cała \underline{sąsiednia okolica} Jordanu;
\vs{6} i~ \underline{byli zanurzani} przez niego w~ Jordanie– \doubleline{ci, którzy wyznawali} swoje grzechy.
\vs{7} A~ \underline{gdy dostrzegł} licznych faryzeuszów i~ saduceuszów, przychodzących do (dokonywanego przez) niego zanurzenia, powiedział im: potomstwo żmij, kto wam pokazał, \doubleline{jak uciec} od \underline{mającego nadejść} gniewu?
\vs{8} Zrodźcie zatem owoc, godny \underline{zmiany myślenia.}
\vs{9} I~ nie \underline{sądźcie, że słuszne} (jest) mówić wobec siebie: mamy ojca– Abrahama, gdyż mówię wam, że Bóg \doubleline{jest w stanie} z~ tych kamieni wzbudzić dzieci Abrahamowi.
\vs{10} Ale \underline{już właśnie} i~ siekiera do korzenia drzew \doubleline{jest przystawiona.} Każde więc drzewo, \underline{które nie} rodzi dobrego owocu, \doubleline{zostaje wycięte} i~ wrzucane w~ ogień.
\vs{11} Ja [-] was zanurzam w~ wodzie ku \underline{zmianie myślenia,} ale \doubleline{ten, który} przychodzi za mną, jest potężniejszy \underline{ode mnie;} nie jestem \doubleline{dostatecznie ważny,} (by) mu sandały podnieść. On was zanurzy w~ Duchu Świętym
\vs{12} którego \underline{szufla do odwiewania} \doubleline{jest w} jego ręku; i~ oczyści swoje klepisko, i~ zbierze swą pszenicę do magazynu, zaś plewy spali \underline{w~} ogniu, \underline{niemożliwym do ugaszenia.}
\vs{13} Wtedy przybył Jezus z~ Galilei nad Jordan do Jana, \underline{by dać się} przez niego \underline{zanurzyć.}
\vs{14} Zaś Jan powstrzymywał go, mówiąc: ja mam potrzebę, (by) \underline{być zanurzonym} przez ciebie, a~ ty przychodzisz do mnie?
\vs{15} A~ Jezus, odpowiadając, powiedział do niego: dopuść teraz, tak bowiem stosowne jest wypełnić nam wszelką sprawiedliwość. Wtedy go dopuścił.
\vs{16} Jezus zaś, \underline{gdy został zanurzony,} natychmiast wyszedł z~ wody. I~ oto \doubleline{otwarte zostały} mu niebiosa i~ dostrzegł ducha Boga, zstępującego \underline{jak gdyby} gołąb i~ przychodzącego na niego.
\vs{17} I~ oto głos z~ nieba mówił: Ten jest moim umiłowanym synem, w~ którym \underline{mam upodobanie.}
//
\endgl
\begingl
\lettrine[loversize=1,lraise=-1.3]{4 }{}%
\gla
 5119 2424 321 1519 2048 5259 4151 {} 3985 5259 1228
{} 2532 3522 5062 2250 2532 5062 3571 5305 3983
{} 2532 4334 3985 4334 846 2036 1487 1488 5207 2316 2036 2443 3778 3037 1096 740
{} 1161 3588 611 2036 1125 3756 1909 3441 740 444 2198 235 1909 3956 4487 1607 1223 4750 2316
{} 5119 3880 846 1228 1519 40 4172 2532 2476 846 1909 4419 2411
{} 2532 3004 846 1487 1488 5207 2316 906 4572 2736 1125 1063 1125 3754 846 32 1781 4012 4675 2532 1909 5495 4571 142 3379 4350 4675 4228 4314 3037
{} 2424 846 5346 1125 3825 1125 3756 1598 2962 4675 2316
{} 3880 846 3825 1228 1519 3029 5308 3735 2532 1166 846 3956 932 2889 2532 846 1391
{} 2532 3004 846 1325 5023 3956 4671 1437 4098 {} 4352 3427 4352
{} 5119 2424 3004 846 5217 3694 3450 4567 1063 1125 2962 4675 2316 4352 2532 846 3441 3000
{} 5119 863 846 1228 2532 2400 32 4334 2532 1247 846
{} 1161 191 2424 191 3754 2491 3860 402 1519 1056
{} 2532 2641 3478 2064 {} 2730 1519 2584 3864 1722 3725 2194 2532 3508
{} 2443 4137 3588 4483 1223 4396 2268 3004
{} 1093 2194 2532 1093 3508 3598 2281 4008 2446 1056 1484
{} 2992 2521 1722 4655 3708 3173 5457 2532 846 2521 1722 5561 2532 4639 2288 393 5457
{} 575 5119 2424 756 2784 2532 3004 3340 1063 1448 932 3772
{} 4043 1161 3844 2281 1056 1492 1417 80 4613 3004 4074 2532 406 846 80 906 293 1519 2281 2258 1063 231
{} 2532 3004 846 1205 3694 3450 2532 4160 5209 231 444
{} 1161 3588 2112 863 1350 190 846
{} 2532 4260 1564 4260 1492 1417 243 80 2385 {} 3588 2199 2532 2491 846 80 {} 1722 4143 3326 3962 846 2199 2675 846 1350 2532 2564 846
{} 3588 1161 2112 863 4143 2532 3962 846 190 846
{} 2532 4013 2424 3650 1056 1321 1722 846 4864 2532 2784 2098 932 2532 2323 3956 3554 2532 3956 3119 1722 2992
{} 2532 565 189 846 1519 3650 4947 2532 4374 846 3956 2560 2192 4912 4164 3554 2532 931 2532 1139 2532 4583 2532 3885 2532 2323 846
{} 2532 190 846 4183 3793 575 1056 2532 1179 2532 2414 2532 2449 2532 4008 2446
//
\glb
 Wtedy Jezus \underline{został wyprowadzony} na pustkowie przez ducha, (by) \doubleline{poddawanym próbie} przez diabła.
\vs{2} I~ \underline{gdy przepościł} czterdzieści dni i~ czterdzieści nocy, wreszcie \doubleline{odczuł głód.}
\vs{3} I~ \underline{gdy} \doubleline{poddający próbie} \underline{podszedł} \underline{do niego,} powiedział: jeżeli jesteś synem Boga powiedz, aby te kamienie \doubleline{stały się} chlebami.
\vs{4} A~ on, odpowiadając, powiedział: \underline{Jest napisane:} nie \doubleline{z powodu} samego chleba człowiek \underline{będzie żył,} lecz \doubleline{z powodu} każdego słowa, wychodzącego przez usta Boga.
\vs{5} Wtedy wziął go diabeł do świętego miasta i~ postawił go na skraju świątyni
\vs{6} i~ mówił mu: jeśli jesteś synem Boga, zrzuć \underline{się sam} \doubleline{na dół,} \underline{jest} bowiem \underline{napisane,} że swym aniołom rozkaże odnośnie ciebie; i~ na rękach cię uniosą, \doubleline{abyś nie} uderzył swą nogą o~ kamień.
\vs{7} Jezus mu powiedział: \underline{jest} także \underline{napisane:} nie \doubleline{będziesz wystawiał na próbę} Pana, twego Boga.
\vs{8} Wziął go znów diabeł na bardzo wysoką górę i~ pokazał mu wszystkie królestwa świata oraz ich chwałę.
\vs{9} I~ powiedział mu: dam to wszystko tobie, jeśli upadniesz (i) \underline{oddasz} mi \underline{pokłon.}
\vs{10} Wtedy Jezus powiedział mu: odejdź ode mnie, szatanie, ponieważ \underline{jest napisane:} Panu, twemu Bogu, \doubleline{będziesz oddawał pokłon} i~ jemu samemu \underline{będziesz służył.}
\vs{11} Wówczas opuścił go diabeł, a~ oto aniołowie podeszli i~ usługiwali mu.
\vs{12} A~ \underline{gdy} Jezus \underline{usłyszał,} że Jan \doubleline{został wydany,} odszedł ku Galilei.
\vs{13} I~ opuściwszy Nazaret, przybył (by) zamieszkać w~ Kafarnaum nadmorskim, w~ granicach Zabulona i~ Neftalego,
\vs{14} aby \underline{wypełniło się} \doubleline{to, co} \underline{zostało powiedziane} \doubleline{za pośrednictwem} proroka Izajasza, mówiącego:
\vs{15} Ziemia Zabulona i~ ziemia Neftalego, droga morska \underline{po drugiej stronie} Jordanu, Galilea narodów.
\vs{16} Lud, siedzący w~ ciemności, ujrzał wielkie światło, a~ im, osiadłym w~ krainie i~ cieniu śmierci, wzeszło światło.
\vs{17} Od wtedy Jezus zaczął głosić i~ mówić: \underline{zmieniajcie myślenie,} gdyż \doubleline{zbliżyło się} królestwo niebios.
\vs{18} Idąc zaś wzdłuż morza Galilejskiego, dostrzegł dwóch braci: Szymona, nazywanego Piotrem, oraz Andrzeja, jego brata, zarzucających niewód w~ morze. Byli bowiem rybakami.
\vs{19} I~ powiedział im: chodźcie za mną, a~ uczynię was rybakami ludzi.
\vs{20} A~ oni, natychmiast zostawiwszy sieci, \underline{poszli za} nim.
\vs{21} I~ \underline{gdy przeszedł} stamtąd \underline{dalej,} dostrzegł dwóch innych braci: Jakuba, (syna) owego Zebedeusza i~ Jana, jego brata, (jak) w~ łodzi z~ ojcem swym, Zebedeuszem, przygotowywali swoje sieci. I~ wezwał ich.
\vs{22} Oni zaś natychmiast, zostawiwszy łódź i~ ojca swego, \underline{zaczęli iść za} nim.
\vs{23} I~ obchodził Jezus całą Galileę, ucząc w~ ich synagogach, i~ głosił \underline{dobrą nowinę} królestwa, i~ uzdrawiał wszelką chorobę i~ każdą niemoc wśród ludzi.
\vs{24} I~ \underline{rozeszła się} wieść \doubleline{o nim} po całej Syrii i~ znosili mu wszystkich, źle \underline{się mających,} przyciśniętych rozmaitymi chorobami i~ męczarniami, i~ opętanych, i~ epileptyków, i~ sparaliżowanych, i~ uzdrowił ich.
\vs{25} I~ \underline{szły za} nim liczne tłumy z~ Galilei i~ Dekapolis, i~ Jerozolimy, i~ Judei, i~ \doubleline{z drugiej strony} Jordanu.
//
\endgl