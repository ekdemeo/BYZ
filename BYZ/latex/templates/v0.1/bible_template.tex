\documentclass[10pt,paper=a5,pagesize=pdftex]{article}
\usepackage[utf8]{inputenc}
\usepackage[T1]{fontenc}
\usepackage{polski}
\usepackage[pagestyles]{titlesec}
\usepackage{lettrine}
\usepackage{setspace}
\usepackage[polish]{babel}
\usepackage{soul}
\usepackage{ebgaramond}
%usepackage{lmodern}  
\usepackage{lipsum}
\usepackage{ragged2e}
%\usepackage[usenames,dvipsnames,svgnames,table]{xcolor}
%\usepackage{fancyhdr}
%  \pagestyle{fancy}
%  \fancyhf{}
%  \fancyhead[RO,LE]{\rightmark}
%  \renewcommand{\headrulewidth}{.5pt}
\selectlanguage{polish}



\newcounter{versecounter}
\renewcommand{\thesection}{\arabic{section}}
\newcommand{\theverse}{\arabic{versecounter}}

% fonts
\font\strong=phvr at 5pt

% definition of the page style with required headers
%\newpagestyle{Biblestyle}{
%  \setheadrule{.009pt}
%  \sethead[\thepage][\chaptertitle]
%    [\toptitlemarks\thesection:\toptitlemarks\theverse---%
%      \bottitlemarks\thesection:\bottitlemarks\theverse]%
%   {\toptitlemarks\thesection:\toptitlemarks\theverse---%
%     \bottitlemarks\thesection:\bottitlemarks\theverse}{\chaptertitle}{\thepage}
%}

% definition of the page style with required headers
\newpagestyle{Biblestyle}{
  \setheadrule{.009pt}
  \sethead[\thepage][\chaptertitle][]{}{\chaptertitle}{\thepage}
  %  [\toptitlemarks\thesection:\toptitlemar{ks\theverse---%
  %    \bottitlemarks\thesection:\bottitlemarks\theverse]%
  % {\toptitlemarks\thesection:\toptitlemarks\theverse---%
  %   \bottitlemarks\thesection:\bottitlemarks\theverse}{\chaptertitle}{\thepage}
}


% sets the marks to be used (section and subsection)
\setmarks{section}{subsection}

% sections and subsections formatting
\titleformat{\section}
{}{\lettrine{\thesection}}{0em}{}[\vskip-0.65\baselineskip]
\titleformat{\subsection}[runin]
{\small\bfseries}{\thesubsection}{1em}{}
\titlespacing{\section}{1em}{-1pt}{0pt}


\pagestyle{Biblestyle}

\renewcommand{\LettrineFontHook}{\bfseries}

\setlength{\parindent}{1pt}

\newlength\NumLen
\newlength\LinLen
% indents one line of text. Indentation= width of section number + 1em
\newcommand\IndOne{%
  \settowidth\NumLen{\thesection}
  \addtolength\NumLen{0.7em}
  \setlength\LinLen{\dimexpr\textwidth-\NumLen}%\the\NumLen\the\LinLen
  \parshape 2 \NumLen \LinLen 0pt \textwidth}
  
% indents two lines of text. Indentation= width of section number + 1em
\newcommand\IndTwo{%
  \settowidth\NumLen{\thesection}
  \addtolength\NumLen{0.7em}
  \setlength\LinLen{\dimexpr\textwidth-\NumLen}%\the\NumLen\the\LinLen
  \parshape 3 \NumLen \LinLen \NumLen \LinLen 0pt \textwidth}
 
% macros

% word
\newcommand\wrd[2]{%
  \leavevmode
  \vbox{\offinterlineskip
    \halign{%
      \hfil##\hfil\cr
      {\strong\vphantom{p}#1}\cr
      \noalign{\vskip\lineskip}%
      \vphantom{A}#2\cr
    }%
  }%
}

\newcommand\doubleline[1]{\underline{{\underline{#1}}}}
\newcommand*\bverse[1]{%
 \stepcounter{versecounter}%
 \textsuperscript{\textbf{\normalsize  #1}}}
 
\begin{document}
\sloppy
\doublespacing
\renewcommand{\chaptertitle}{Mateusza}
%\justify
 \section{} \IndTwo
                \bverse{1}
                        \underline{\wrd{976}{Zwój księgi}}
                        \wrd{}{(o)}
                        \wrd{1078}{narodzinach}
                        \wrd{2424}{Jezusa}
                        \wrd{5547}{Chrystusa,}
                        \wrd{5207}{syna}
                        \wrd{1138}{Dawida,}
                        \wrd{5207}{syna}
                        \wrd{11}{Abrahama.}
                \bverse{2}
                        \wrd{11}{Abraham}
                        \wrd{1080}{zrodził}
                        \wrd{2464}{Izaaka,}
                        \wrd{1161}{a~}
                        \wrd{2464}{Izaak}
                        \wrd{1080}{zrodził}
                        \wrd{2384}{Jakuba,}
                        \wrd{1161}{a~}
                        \wrd{2384}{Jakub}
                        \wrd{1080}{zrodził}
                        \wrd{2455}{Judę}
                        \wrd{2532}{i~}
                        \wrd{80}{braci}
                        \wrd{846}{jego,}
                \bverse{3}
                        \wrd{1161}{a~}
                        \wrd{2455}{Juda}
                        \wrd{1080}{zrodził}
                        \wrd{5329}{Faresa}
                        \wrd{2532}{i~}
                        \wrd{2196}{Zarę}
                        \wrd{1537}{z~}
                        \wrd{2283}{Tamary,}
                        \wrd{1161}{a~}
                        \wrd{5329}{Fares}
                        \wrd{1080}{zrodził}
                        \wrd{2074}{Esroma,}
                        \wrd{1161}{a~}
                        \wrd{2074}{Esrom}
                        \wrd{1080}{zrodził}
                        \wrd{689}{Arama,}
                \bverse{4}
                        \wrd{1161}{a~}
                        \wrd{689}{Aram}
                        \wrd{1080}{zrodził}
                        \wrd{284}{Aminadaba,}
                        \wrd{1161}{a~}
                        \wrd{284}{Aminadab}
                        \wrd{1080}{zrodził}
                        \wrd{3476}{Naassona,}
                        \wrd{1161}{a~}
                        \wrd{3476}{Naasson}
                        \wrd{1080}{zrodził}
                        \wrd{4533}{Salmona,}
                \bverse{5}
                        \wrd{1161}{a~}
                        \wrd{4533}{Salmon}
                        \wrd{1080}{zrodził}
                        \wrd{1003}{Booza}
                        \wrd{1537}{z~}
                        \wrd{4477}{Rahab,}
                        \wrd{1161}{a~}
                        \wrd{1003}{Booz}
                        \wrd{1080}{zrodził}
                        \wrd{5601}{Jobeda}
                        \wrd{1537}{z~}
                        \wrd{4503}{Rut,}
                        \wrd{1161}{a~}
                        \wrd{5601}{Jobed}
                        \wrd{1080}{zrodził}
                        \wrd{2421}{Jesaja,}
                \bverse{6}
                        \wrd{1161}{a~}
                        \wrd{2421}{Jesaj}
                        \wrd{1080}{zrodził}
                        \wrd{1138}{Dawida–}
                        \wrd{935}{króla,}
                        \wrd{1161}{a~}
                        \wrd{1138}{Dawid–}
                        \wrd{935}{król}
                        \wrd{1080}{zrodził}
                        \wrd{4672}{Salomona}
                        \wrd{1537}{z~}
                        \wrd{3588}{tej,}
                        \wrd{}{(która należała do)}
                        \wrd{3774}{Uriasza,}
                \bverse{7}
                        \wrd{1161}{a~}
                        \wrd{4672}{Salomon}
                        \wrd{1080}{zrodził}
                        \wrd{4497}{Roboama,}
                        \wrd{1161}{a~}
                        \wrd{4497}{Roboam}
                        \wrd{1080}{zrodził}
                        \wrd{7}{Abiasa,}
                        \wrd{1161}{a~}
                        \wrd{7}{Abias}
                        \wrd{1080}{zrodził}
                        \wrd{760}{Asafa,}
                \bverse{8}
                        \wrd{1161}{a~}
                        \wrd{760}{Asaf}
                        \wrd{1080}{zrodził}
                        \wrd{2498}{Jozafata,}
                        \wrd{1161}{a~}
                        \wrd{2498}{Jozafat}
                        \wrd{1080}{zrodził}
                        \wrd{2496}{Jorama,}
                        \wrd{1161}{a~}
                        \wrd{2496}{Joram}
                        \wrd{1080}{zrodził}
                        \wrd{3604}{Ozjasza,}
                \bverse{9}
                        \wrd{1161}{a~}
                        \wrd{3604}{Ozjasz}
                        \wrd{1080}{zrodził}
                        \wrd{2488}{Joatama,}
                        \wrd{1161}{a~}
                        \wrd{2488}{Joatam}
                        \wrd{1080}{zrodził}
                        \wrd{881}{Achaza,}
                        \wrd{1161}{a~}
                        \wrd{881}{Achaz}
                        \wrd{1080}{zrodził}
                        \wrd{1478}{Ezechiasza,}
                \bverse{10}
                        \wrd{1161}{a~}
                        \wrd{1478}{Ezechiasz}
                        \wrd{1080}{zrodził}
                        \wrd{3128}{Manassesa,}
                        \wrd{1161}{a~}
                        \wrd{3128}{Manasses}
                        \wrd{1080}{zrodził}
                        \wrd{300}{Amona,}
                        \wrd{1161}{a~}
                        \wrd{300}{Amon}
                        \wrd{1080}{zrodził}
                        \wrd{2502}{Jozjasza,}
                \bverse{11}
                        \wrd{1161}{a~}
                        \wrd{2502}{Jozjasz}
                        \wrd{1080}{zrodził}
                        \wrd{2423}{Jechoniasza}
                        \wrd{2532}{oraz}
                        \wrd{80}{braci}
                        \wrd{846}{jego}
                        \underline{\wrd{1909}{w czasie}}
                        \wrd{3350}{przesiedlenia}
                        \wrd{}{(do)}
                        \wrd{897}{Babilonu,}
                \bverse{12}
                        \wrd{1161}{a~}
                        \wrd{3326}{po}
                        \wrd{3350}{przesiedleniu}
                        \wrd{}{(do)}
                        \wrd{897}{Babilonu}
                        \wrd{2423}{Jechoniasz}
                        \wrd{1080}{zrodził}
                        \wrd{4528}{Salatiela,}
                        \wrd{1161}{a~}
                        \wrd{4528}{Salatiel}
                        \wrd{1080}{zrodził}
                        \wrd{2216}{Zorobabela,}
                \bverse{13}
                        \wrd{1161}{a~}
                        \wrd{2216}{Zorobabel}
                        \wrd{1080}{zrodził}
                        \wrd{10}{Abiuda,}
                        \wrd{1161}{a~}
                        \wrd{10}{Abiud}
                        \wrd{1080}{zrodził}
                        \wrd{1662}{Eliakima,}
                        \wrd{1161}{a~}
                        \wrd{1662}{Eliakim}
                        \wrd{1080}{zrodził}
                        \wrd{107}{Azora,}
                \bverse{14}
                        \wrd{1161}{a~}
                        \wrd{107}{Azor}
                        \wrd{1080}{zrodził}
                        \wrd{4524}{Sadoka,}
                        \wrd{1161}{a~}
                        \wrd{4524}{Sadok}
                        \wrd{1080}{zrodził}
                        \wrd{885}{Achima,}
                        \wrd{1161}{a~}
                        \wrd{885}{Achim}
                        \wrd{1080}{zrodził}
                        \wrd{1664}{Eliuda,}
                \bverse{15}
                        \wrd{1161}{a~}
                        \wrd{1664}{Eliud}
                        \wrd{1080}{zrodził}
                        \wrd{1648}{Eleazara,}
                        \wrd{1161}{a~}
                        \wrd{1648}{Eleazar}
                        \wrd{1080}{zrodził}
                        \wrd{3157}{Mattana,}
                        \wrd{1161}{a~}
                        \wrd{3157}{Mattan}
                        \wrd{1080}{zrodził}
                        \wrd{2384}{Jakuba,}
                \bverse{16}
                        \wrd{1161}{a~}
                        \wrd{2384}{Jakub}
                        \wrd{1080}{zrodził}
                        \wrd{2501}{Józefa,}
                        \wrd{435}{męża}
                        \wrd{3137}{Marii,}
                        \wrd{1537}{z~}
                        \wrd{3739}{której}
                        \underline{\wrd{1080}{został zrodzony}}
                        \wrd{2424}{Jezus,}
                        \wrd{3004}{nazywany}
                        \wrd{5547}{Chrystusem.}
                \bverse{17}
                        \wrd{3956}{Wszystkich}
                        \wrd{3767}{więc}
                        \wrd{1074}{pokoleń}
                        \wrd{575}{od}
                        \wrd{11}{Abrahama}
                        \wrd{2193}{do}
                        \wrd{1138}{Dawida–}
                        \wrd{1074}{pokoleń}
                        \wrd{1180}{czternaście,}
                        \wrd{2532}{i~}
                        \wrd{575}{od}
                        \wrd{1138}{Dawida}
                        \wrd{2193}{do}
                        \wrd{3350}{przesiedlenia}
                        \wrd{}{(do)}
                        \wrd{897}{Babilonu–}
                        \wrd{1074}{pokoleń}
                        \wrd{1180}{czternaście,}
                        \wrd{2532}{i~}
                        \wrd{575}{od}
                        \wrd{3350}{przesiedlenia}
                        \wrd{}{(do)}
                        \wrd{897}{Babilonu}
                        \wrd{2193}{do}
                        \wrd{5547}{Chrystusa–}
                        \wrd{1074}{pokoleń}
                        \wrd{1180}{czternaście.}
                \bverse{18}
                        \wrd{1161}{A~}
                        \wrd{1083}{narodzenie}
                        \wrd{2424}{Jezusa}
                        \wrd{5547}{Chrystusa}
                        \wrd{2258}{było}
                        \wrd{3779}{takie:}
                        \wrd{1063}{ponieważ}
                        \wrd{3384}{matka}
                        \wrd{846}{jego,}
                        \wrd{3137}{Maria,}
                        \underline{\wrd{3423}{będąc zaślubiona}}
                        \wrd{2501}{Józefowi,}
                        \doubleline{\wrd{4250}{wcześniej, zanim}}
                        \wrd{2228}{[-]}
                        \underline{\wrd{4905}{się}}
                        \wrd{846}{oni}
                        \underline{\wrd{4905}{zeszli,}}
                        \doubleline{\wrd{2147}{znalazła się}}
                        \underline{\wrd{2192}{tą, która ma}}
                        \wrd{}{(dziecko)}
                        \wrd{1722}{w~}
                        \wrd{1064}{łonie}
                        \doubleline{\wrd{1537}{za sprawą}}
                        \wrd{4151}{Ducha}
                        \wrd{40}{Świętego,}
                \bverse{19}
                        \wrd{1161}{a~}
                        \wrd{2501}{Józef,}
                        \wrd{846}{jej}
                        \wrd{435}{mąż,}
                        \wrd{5607}{będąc}
                        \wrd{1342}{sprawiedliwym}
                        \wrd{2532}{i~}
                        \wrd{3361}{nie}
                        \wrd{2309}{chcąc}
                        \wrd{846}{jej}
                        \underline{\wrd{3856}{wystawić na pośmiewisko,}}
                        \wrd{1014}{chciał}
                        \wrd{2977}{potajemnie}
                        \wrd{846}{ją}
                        \wrd{630}{uwolnić,}
                \bverse{20}
                        \wrd{1161}{a~}
                        \wrd{}{(gdy)}
                        \wrd{846}{on}
                        \wrd{5023}{to}
                        \wrd{1760}{obmyślił,}
                        \wrd{2400}{oto}
                        \wrd{32}{anioł}
                        \wrd{2962}{Pana}
                        \underline{\wrd{5316}{ukazał się}}
                        \wrd{846}{mu}
                        \wrd{2596}{podczas}
                        \wrd{3677}{snu,}
                        \wrd{3004}{mówiąc:}
                        \wrd{2501}{Józefie,}
                        \wrd{5207}{synu}
                        \wrd{1138}{Dawida,}
                        \wrd{3361}{nie}
                        \doubleline{\wrd{5399}{bój się}}
                        \wrd{3880}{wziąć}
                        \wrd{}{(do siebie)}
                        \wrd{3137}{Marii,}
                        \wrd{4675}{twojej}
                        \wrd{1135}{żony,}
                        \wrd{1063}{ponieważ}
                        \underline{\wrd{3588}{to, co}}
                        \wrd{1722}{w~}
                        \wrd{846}{niej,}
                        \wrd{1080}{zrodzone}
                        \wrd{2076}{jest}
                        \doubleline{\wrd{1537}{za sprawą}}
                        \wrd{4151}{Ducha}
                        \wrd{40}{Świętego;}
                \bverse{21}
                        \wrd{1161}{i~}
                        \wrd{5088}{urodzi}
                        \wrd{5207}{syna,}
                        \wrd{2532}{i~}
                        \wrd{2564}{nazwiesz}
                        \wrd{846}{go}
                        \wrd{3686}{imieniem}
                        \wrd{2424}{Jezus,}
                        \wrd{1063}{ponieważ}
                        \wrd{846}{On}
                        \wrd{4982}{wybawi}
                        \wrd{2992}{lud}
                        \wrd{846}{swój}
                        \wrd{575}{od}
                        \wrd{846}{ich}
                        \wrd{266}{grzechów.}
                \bverse{22}
                        \wrd{1161}{A~}
                        \wrd{5124}{to}
                        \wrd{3650}{wszystko}
                        \underline{\wrd{1096}{stało się,}}
                        \wrd{2443}{aby}
                        \doubleline{\wrd{4137}{wypełniło się}}
                        \underline{\wrd{3588}{to, co}}
                        \doubleline{\wrd{4483}{zostało powiedziane}}
                        \wrd{5259}{poprzez}
                        \wrd{2962}{Pana}
                        \underline{\wrd{1223}{za pośrednictwem}}
                        \wrd{4396}{proroka,}
                        \wrd{3004}{mówiącego:}
                \bverse{23}
                        \wrd{2400}{Oto}
                        \wrd{3933}{panna}
                        \underline{\wrd{2192}{„będzie mieć}}
                        \wrd{1722}{w~}
                        \wrd{1064}{łonie”}
                        \wrd{2532}{i~}
                        \wrd{5088}{urodzi}
                        \wrd{5207}{syna,}
                        \wrd{2532}{i~}
                        \wrd{2564}{nazwą}
                        \wrd{846}{go}
                        \wrd{3686}{imieniem}
                        \wrd{1694}{Emmanuel,}
                        \wrd{3739}{co}
                        \wrd{2076}{jest}
                        \wrd{3177}{tłumaczone:}
                        \doubleline{\wrd{3326}{razem z}}
                        \wrd{2257}{nami}
                        \wrd{2316}{Bóg.}
                \bverse{24}
                        \wrd{1161}{A~}
                        \wrd{2501}{Józef,}
                        \underline{\wrd{1326}{obudziwszy się}}
                        \wrd{575}{ze}
                        \wrd{5258}{snu,}
                        \wrd{4160}{uczynił,}
                        \wrd{5613}{jak}
                        \wrd{4367}{nakazał}
                        \wrd{846}{mu}
                        \wrd{32}{anioł}
                        \wrd{2962}{Pana,}
                        \wrd{2532}{i~}
                        \wrd{3880}{przyjął}
                        \wrd{846}{swoją}
                        \wrd{1135}{żonę,}
                \bverse{25}
                        \wrd{2532}{i~}
                        \wrd{3756}{nie}
                        \wrd{1097}{poznawał}
                        \wrd{846}{jej,}
                        \wrd{2193}{aż}
                        \wrd{3739}{[-]}
                        \wrd{5088}{urodziła}
                        \wrd{5207}{syna}
                        \wrd{846}{swego}
                        \wrd{4416}{pierworodnego;}
                        \wrd{2532}{i~}
                        \wrd{2564}{nazwał}
                        \wrd{846}{go}
                        \wrd{3686}{imieniem}
                        \wrd{2424}{Jezus.} \section{} \IndTwo
                \bverse{1}
                        \wrd{1161}{A~}
                        \underline{\wrd{1080}{kiedy}}
                        \wrd{2424}{Jezus}
                        \underline{\wrd{1080}{urodził się}}
                        \wrd{1722}{w~}
                        \wrd{965}{Betlejem,}
                        \doubleline{\wrd{2449}{w Judei,}}
                        \wrd{1722}{za}
                        \wrd{2250}{dni}
                        \wrd{2264}{Heroda–}
                        \wrd{935}{króla,}
                        \wrd{2400}{oto}
                        \wrd{3097}{magowie}
                        \wrd{575}{ze}
                        \wrd{395}{wschodu}
                        \wrd{3854}{przybyli}
                        \wrd{1519}{do}
                        \wrd{2414}{Jerozolimy,}
                \bverse{2}
                        \wrd{3004}{mówiąc:}
                        \wrd{4226}{Gdzie}
                        \underline{\wrd{2076}{się znajduje}}
                        \wrd{}{(ten)}
                        \wrd{5088}{narodzony}
                        \wrd{935}{król}
                        \wrd{2453}{Judejczyków?}
                        \wrd{1492}{Zobaczyliśmy}
                        \wrd{1063}{bowiem}
                        \wrd{846}{jego}
                        \wrd{792}{gwiazdę}
                        \wrd{1722}{na}
                        \wrd{395}{wschodzie}
                        \wrd{2532}{i~}
                        \wrd{2064}{przyszliśmy}
                        \doubleline{\wrd{4352}{oddać}}
                        \wrd{846}{mu}
                        \doubleline{\wrd{4352}{pokłon.}}
                \bverse{3}
                        \wrd{1161}{A~}
                        \underline{\wrd{191}{gdy}}
                        \wrd{}{(to)}
                        \underline{\wrd{191}{usłyszał}}
                        \wrd{935}{król}
                        \wrd{2264}{Herod,}
                        \doubleline{\wrd{5015}{przestraszył się,}}
                        \wrd{2532}{a~}
                        \underline{\wrd{3326}{razem z}}
                        \wrd{846}{nim}
                        \wrd{3956}{cała}
                        \wrd{2414}{Jerozolima.}
                \bverse{4}
                        \wrd{2532}{I~}
                        \wrd{4863}{zebrawszy}
                        \wrd{3956}{wszystkich}
                        \wrd{749}{arcykapłanów}
                        \wrd{2532}{i~}
                        \underline{\wrd{1122}{znawców Pisma}}
                        \wrd{}{(spośród)}
                        \wrd{2992}{ludu,}
                        \doubleline{\wrd{4441}{dowiadywał się}}
                        \wrd{3844}{od}
                        \wrd{846}{nich,}
                        \wrd{4226}{gdzie}
                        \underline{\wrd{1080}{się rodzi}}
                        \wrd{5547}{Mesjasz.}
                \bverse{5}
                        \wrd{1161}{A~}
                        \wrd{3588}{oni}
                        \wrd{2036}{powiedzieli}
                        \wrd{846}{mu:}
                        \wrd{1722}{w~}
                        \wrd{965}{Betlejem,}
                        \underline{\wrd{2449}{w Judei,}}
                        \wrd{3779}{tak}
                        \wrd{1063}{bowiem}
                        \doubleline{\wrd{1125}{jest napisane}}
                        \wrd{1223}{przez}
                        \wrd{4396}{proroka:}
                \bverse{6}
                        \wrd{2532}{I~}
                        \wrd{4771}{ty,}
                        \wrd{965}{Betlejem,}
                        \wrd{1093}{ziemio}
                        \wrd{2448}{Judy,}
                        \underline{\wrd{3760}{wcale nie}}
                        \wrd{1488}{jesteś}
                        \wrd{1646}{najmniejsze}
                        \wrd{1722}{spośród}
                        \wrd{2232}{rządców}
                        \wrd{2448}{Judy,}
                        \wrd{1063}{ponieważ}
                        \wrd{1537}{z~}
                        \wrd{4675}{ciebie}
                        \wrd{1831}{wyjdzie}
                        \wrd{2233}{rządzący,}
                        \wrd{3748}{który}
                        \doubleline{\wrd{4165}{paść będzie}}
                        \wrd{2992}{lud}
                        \wrd{3450}{mój–}
                        \wrd{2474}{Izraela.}
                \bverse{7}
                        \wrd{5119}{Wtedy}
                        \wrd{2264}{Herod,}
                        \wrd{2977}{potajemnie}
                        \wrd{2564}{wezwawszy}
                        \wrd{3097}{magów,}
                        \underline{\wrd{198}{dokładnie dowiedział się}}
                        \wrd{3844}{od}
                        \wrd{846}{nich}
                        \wrd{}{(o)}
                        \wrd{5550}{czasie}
                        \doubleline{\wrd{5316}{ukazującej się}}
                        \wrd{792}{gwiazdy.}
                \bverse{8}
                        \wrd{2532}{I~}
                        \wrd{3992}{wysławszy}
                        \wrd{846}{ich}
                        \wrd{1519}{do}
                        \wrd{965}{Betlejem,}
                        \wrd{2036}{powiedział:}
                        \wrd{4198}{dotrzyjcie,}
                        \wrd{199}{dokładnie}
                        \wrd{1833}{wypytajcie}
                        \wrd{4012}{o~}
                        \wrd{3813}{dziecko.}
                        \wrd{1161}{A~}
                        \wrd{1875}{gdy}
                        \wrd{}{(je)}
                        \wrd{2147}{znajdziecie,}
                        \wrd{518}{oznajmijcie}
                        \wrd{3427}{mi,}
                        \wrd{3704}{żebym}
                        \underline{\wrd{2504}{i ja}}
                        \wrd{2064}{przyszedł}
                        \wrd{}{(i)}
                        \doubleline{\wrd{4352}{oddał}}
                        \wrd{846}{mu}
                        \doubleline{\wrd{4352}{pokłon.}}
                \bverse{9}
                        \wrd{1161}{A~}
                        \wrd{3588}{oni,}
                        \underline{\wrd{191}{gdy wysłuchali}}
                        \wrd{935}{króla,}
                        \wrd{4198}{wyruszyli;}
                        \wrd{2532}{a~}
                        \wrd{2400}{oto}
                        \wrd{792}{gwiazda,}
                        \wrd{3739}{którą}
                        \wrd{1492}{dostrzegli}
                        \wrd{1722}{na}
                        \wrd{395}{wschodzie,}
                        \wrd{4254}{wyprzedzała}
                        \wrd{846}{ich,}
                        \wrd{2193}{aż}
                        \wrd{2064}{przybyła}
                        \wrd{}{(i)}
                        \wrd{2476}{stanęła}
                        \wrd{1883}{ponad}
                        \wrd{}{(miejscem),}
                        \wrd{3757}{gdzie}
                        \wrd{2258}{było}
                        \wrd{3813}{dziecko.}
                \bverse{10}
                        \wrd{1161}{A~}
                        \underline{\wrd{1492}{gdy dostrzegli}}
                        \wrd{792}{gwiazdę,}
                        \doubleline{\wrd{5463}{uradowali się}}
                        \wrd{5479}{radością}
                        \wrd{4970}{bardzo}
                        \wrd{3173}{wielką.}
                \bverse{11}
                        \wrd{2532}{I~}
                        \underline{\wrd{2064}{gdy przyszli}}
                        \wrd{1519}{do}
                        \wrd{3614}{domu,}
                        \wrd{3708}{zobaczyli}
                        \wrd{3813}{dziecko,}
                        \doubleline{\wrd{3326}{razem z}}
                        \wrd{3137}{Marią,}
                        \wrd{846}{jego}
                        \wrd{3384}{matką,}
                        \wrd{2532}{i~}
                        \wrd{4098}{upadłszy,}
                        \underline{\wrd{4352}{oddali}}
                        \wrd{846}{mu}
                        \underline{\wrd{4352}{pokłon,}}
                        \wrd{2532}{a~}
                        \wrd{455}{otworzywszy}
                        \wrd{846}{swoje}
                        \wrd{2344}{skarby,}
                        \wrd{4374}{ofiarowali}
                        \wrd{846}{mu}
                        \wrd{1435}{dary:}
                        \wrd{5557}{złoto}
                        \wrd{2532}{i~}
                        \wrd{3030}{kadzidło,}
                        \wrd{2532}{i~}
                        \wrd{4666}{mirrę.}
                \bverse{12}
                        \wrd{2532}{A~}
                        \underline{\wrd{5537}{gdy}}
                        \wrd{2596}{podczas}
                        \wrd{3677}{snu}
                        \underline{\wrd{5537}{otrzymali pouczenie,}}
                        \wrd{}{(aby)}
                        \wrd{3361}{nie}
                        \wrd{344}{powracać}
                        \wrd{4314}{do}
                        \wrd{2264}{Heroda,}
                        \wrd{1223}{poprzez}
                        \wrd{243}{inną}
                        \wrd{3598}{drogę}
                        \wrd{402}{odeszli}
                        \wrd{1519}{na}
                        \wrd{846}{swoje}
                        \wrd{5561}{terytorium.}
                \bverse{13}
                        \wrd{1161}{A~}
                        \underline{\wrd{402}{gdy}}
                        \wrd{846}{oni}
                        \underline{\wrd{402}{odeszli,}}
                        \wrd{2400}{oto}
                        \wrd{32}{anioł}
                        \wrd{2962}{Pana}
                        \doubleline{\wrd{5316}{ukazał się}}
                        \wrd{2596}{podczas}
                        \wrd{3677}{snu}
                        \wrd{2501}{Józefowi,}
                        \wrd{3004}{mówiąc:}
                        \wrd{1453}{powstań,}
                        \wrd{3880}{weź}
                        \wrd{3813}{dziecko}
                        \wrd{2532}{i~}
                        \wrd{846}{jego}
                        \wrd{3384}{matkę,}
                        \wrd{2532}{i~}
                        \wrd{5343}{uciekaj}
                        \wrd{1519}{do}
                        \wrd{125}{Egiptu,}
                        \wrd{2532}{i~}
                        \wrd{2468}{bądź}
                        \wrd{1563}{tam,}
                        \wrd{2193}{aż}
                        \wrd{302}{[-]}
                        \wrd{4671}{ci}
                        \wrd{2036}{powiem.}
                        \wrd{3195}{Zamierza}
                        \wrd{1063}{bowiem}
                        \wrd{2264}{Herod}
                        \wrd{2212}{odszukać}
                        \wrd{3813}{dziecko,}
                        \wrd{}{(żeby)}
                        \wrd{846}{je}
                        \wrd{622}{stracić.}
                \bverse{14}
                        \wrd{1161}{A~}
                        \wrd{3588}{on,}
                        \underline{\wrd{1453}{gdy powstał,}}
                        \wrd{3880}{wziął}
                        \wrd{3571}{nocą}
                        \wrd{3588}{to}
                        \wrd{3813}{dziecko}
                        \wrd{2532}{oraz}
                        \wrd{846}{jego}
                        \wrd{3384}{matkę}
                        \wrd{2532}{i~}
                        \wrd{402}{odszedł}
                        \wrd{1519}{do}
                        \wrd{125}{Egiptu.}
                \bverse{15}
                        \wrd{2532}{I~}
                        \wrd{2258}{był}
                        \wrd{1563}{tam}
                        \underline{\wrd{2193}{aż do}}
                        \wrd{5054}{końca}
                        \wrd{}{(życia)}
                        \wrd{2264}{Heroda,}
                        \wrd{2443}{aby}
                        \doubleline{\wrd{4137}{wypełniło się}}
                        \underline{\wrd{3588}{to, co}}
                        \doubleline{\wrd{4483}{zostało powiedziane}}
                        \wrd{5259}{przez}
                        \wrd{2962}{Pana}
                        \underline{\wrd{1223}{za pośrednictwem}}
                        \wrd{4396}{proroka,}
                        \wrd{3004}{mówiącego:}
                        \wrd{1537}{z~}
                        \wrd{125}{Egiptu}
                        \wrd{2564}{wywołałem}
                        \wrd{5207}{syna}
                        \wrd{3450}{mego.}
                        \wrd{}{(Oz 11:1)}
                \bverse{16}
                        \wrd{5119}{Wtedy}
                        \wrd{2264}{Herod,}
                        \underline{\wrd{1492}{gdy dostrzegł,}}
                        \wrd{3754}{że}
                        \doubleline{\wrd{1702}{został wyszydzony}}
                        \wrd{5259}{przez}
                        \wrd{3097}{magów,}
                        \wrd{3029}{bardzo}
                        \underline{\wrd{2373}{się rozgniewał}}
                        \wrd{2532}{i~}
                        \wrd{649}{posławszy}
                        \wrd{}{(swoich ludzi),}
                        \wrd{337}{zgładził}
                        \wrd{3956}{wszystkich}
                        \wrd{3816}{chłopców}
                        \wrd{1722}{w~}
                        \wrd{965}{Betlejem}
                        \wrd{2532}{oraz}
                        \wrd{1722}{we}
                        \wrd{3956}{wszelkich}
                        \wrd{846}{jego}
                        \wrd{3725}{granicach,}
                        \wrd{575}{od}
                        \wrd{1332}{dwulatków}
                        \wrd{}{(począwszy)}
                        \wrd{2532}{oraz}
                        \wrd{2736}{poniżej}
                        \wrd{}{(tego wieku),}
                        \doubleline{\wrd{2596}{zgodnie z}}
                        \wrd{5550}{czasem,}
                        \wrd{}{(o)}
                        \wrd{3739}{którym}
                        \underline{\wrd{198}{dokładnie się dowiedział}}
                        \wrd{3844}{od}
                        \wrd{3097}{magów.}
                \bverse{17}
                        \wrd{5119}{Wtedy}
                        \underline{\wrd{4137}{wypełniło się}}
                        \wrd{}{(to, co)}
                        \doubleline{\wrd{4483}{zostało powiedziane}}
                        \wrd{5259}{przez}
                        \wrd{4396}{proroka}
                        \wrd{2408}{Jeremiasza,}
                        \wrd{3004}{mówiącego:}
                \bverse{18}
                        \wrd{1722}{W~}
                        \wrd{4471}{Rama}
                        \underline{\wrd{191}{był słyszany}}
                        \wrd{5456}{głos:}
                        \wrd{2355}{lament}
                        \wrd{2532}{i~}
                        \wrd{2805}{płacz,}
                        \wrd{2532}{i~}
                        \wrd{4183}{wielkie}
                        \wrd{3602}{narzekanie;}
                        \wrd{4478}{Rachel}
                        \wrd{2799}{opłakuje}
                        \wrd{846}{swoje}
                        \wrd{5043}{dzieci}
                        \wrd{2532}{i~}
                        \wrd{3756}{nie}
                        \wrd{2309}{chce}
                        \doubleline{\wrd{3870}{zostać pocieszona,}}
                        \wrd{3754}{ponieważ}
                        \wrd{}{(ich już)}
                        \wrd{3756}{nie}
                        \wrd{1526}{ma.}
                        \wrd{}{(Jr 31:15)}
                \bverse{19}
                        \wrd{1161}{A~}
                        \underline{\wrd{5053}{gdy}}
                        \wrd{2264}{Herod}
                        \underline{\wrd{5053}{doszedł do końca}}
                        \wrd{}{(życia),}
                        \wrd{2400}{oto}
                        \wrd{32}{anioł}
                        \wrd{2962}{Pana}
                        \wrd{2596}{poprzez}
                        \wrd{3677}{sen}
                        \doubleline{\wrd{5316}{ukazał się}}
                        \wrd{2501}{Józefowi}
                        \wrd{1722}{w~}
                        \wrd{125}{Egipcie,}
                \bverse{20}
                        \wrd{3004}{mówiąc:}
                        \wrd{1453}{Powstań,}
                        \wrd{3880}{weź}
                        \wrd{3813}{dziecko}
                        \wrd{2532}{i~}
                        \wrd{846}{jego}
                        \wrd{3384}{matkę}
                        \wrd{2532}{i~}
                        \wrd{4198}{wyrusz}
                        \wrd{1519}{do}
                        \wrd{1093}{ziemi}
                        \wrd{2474}{Izraela.}
                        \wrd{2348}{Umarli}
                        \wrd{1063}{bowiem}
                        \underline{\wrd{3588}{ci, którzy}}
                        \wrd{2212}{szukają}
                        \wrd{5590}{duszy}
                        \wrd{3813}{dziecka.}
                \bverse{21}
                        \wrd{1161}{A~}
                        \wrd{3588}{On,}
                        \underline{\wrd{1453}{gdy powstał,}}
                        \wrd{3880}{wziął}
                        \wrd{3813}{dziecko}
                        \wrd{2532}{i~}
                        \wrd{846}{jego}
                        \wrd{3384}{matkę,}
                        \wrd{2532}{i~}
                        \wrd{2064}{przybył}
                        \wrd{1519}{do}
                        \wrd{1093}{ziemi}
                        \wrd{2474}{Izraela.}
                \bverse{22}
                        \wrd{1161}{Ale}
                        \underline{\wrd{191}{gdy usłyszał,}}
                        \wrd{3754}{że}
                        \wrd{745}{Archelaos}
                        \doubleline{\wrd{936}{jest królem}}
                        \wrd{1909}{w~}
                        \wrd{2449}{Judei–}
                        \wrd{473}{zamiast}
                        \wrd{2264}{Heroda,}
                        \wrd{846}{swojego}
                        \wrd{3962}{ojca–}
                        \underline{\wrd{5399}{przestraszył się}}
                        \wrd{1563}{tam}
                        \wrd{565}{wejść;}
                        \wrd{1161}{a~}
                        \doubleline{\wrd{5537}{otrzymawszy pouczenie}}
                        \wrd{2596}{podczas}
                        \wrd{3677}{snu,}
                        \wrd{402}{odszedł}
                        \wrd{1519}{ku}
                        \wrd{3313}{obszarom}
                        \wrd{1056}{Galilei.}
                \bverse{23}
                        \wrd{2532}{A~}
                        \underline{\wrd{2064}{gdy}}
                        \wrd{}{(tam)}
                        \underline{\wrd{2064}{przybył,}}
                        \wrd{2730}{zamieszkał}
                        \wrd{1519}{w~}
                        \wrd{4172}{mieście,}
                        \wrd{3004}{zwanym}
                        \wrd{3478}{Nazaret,}
                        \wrd{3704}{aby}
                        \doubleline{\wrd{4137}{wypełniło się}}
                        \underline{\wrd{3588}{to, co}}
                        \doubleline{\wrd{4483}{zostało powiedziane}}
                        \underline{\wrd{1223}{za pośrednictwem}}
                        \wrd{4396}{proroków,}
                        \wrd{3754}{że}
                        \doubleline{\wrd{2564}{będzie nazwany}}
                        \wrd{3480}{Nazarejczykiem.} \section{} \IndTwo
                \bverse{1}
                        \wrd{1161}{A~}
                        \wrd{1722}{w~}
                        \wrd{1565}{owe}
                        \wrd{2250}{dni}
                        \wrd{3854}{przybył}
                        \wrd{2491}{Jan}
                        \wrd{910}{Chrzciciel,}
                        \underline{\wrd{2784}{który głosił}}
                        \wrd{1722}{na}
                        \wrd{2048}{pustkowiu}
                        \wrd{2449}{Judei}
                \bverse{2}
                        \wrd{2532}{i~}
                        \wrd{3004}{mówił:}
                        \underline{\wrd{3340}{zmieniajcie myślenie,}}
                        \wrd{1063}{ponieważ}
                        \doubleline{\wrd{1448}{zbliżyło się}}
                        \wrd{932}{królestwo}
                        \wrd{3772}{niebios.}
                \bverse{3}
                        \wrd{2076}{Jest}
                        \wrd{3778}{on}
                        \wrd{1063}{bowiem}
                        \wrd{4483}{przepowiedziany}
                        \wrd{5259}{przez}
                        \wrd{4396}{proroka}
                        \wrd{2268}{Izajasza,}
                        \underline{\wrd{3004}{który mówi:}}
                        \wrd{5456}{Głos}
                        \wrd{994}{wołającego}
                        \wrd{1722}{na}
                        \wrd{2048}{pustkowiu:}
                        \wrd{2090}{przygotujcie}
                        \wrd{3598}{drogę}
                        \wrd{2962}{Pana;}
                        \wrd{2117}{prostymi}
                        \wrd{4160}{czyńcie}
                        \wrd{846}{Jego}
                        \wrd{5147}{ścieżki.}
                \bverse{4}
                        \wrd{846}{Sam}
                        \wrd{1161}{zaś}
                        \wrd{2491}{Jan}
                        \wrd{2192}{miał}
                        \wrd{846}{swe}
                        \wrd{1742}{odzienie}
                        \wrd{575}{z~}
                        \wrd{2359}{sierści}
                        \wrd{2574}{wielbłąda}
                        \wrd{2532}{oraz}
                        \wrd{1193}{skórzany}
                        \wrd{2223}{pas}
                        \wrd{4012}{wokół}
                        \wrd{846}{swoich}
                        \wrd{3751}{bioder,}
                        \wrd{1161}{a~}
                        \wrd{5160}{pokarmem}
                        \wrd{846}{jego}
                        \wrd{2258}{była}
                        \wrd{200}{szarańcza}
                        \wrd{2532}{i~}
                        \wrd{66}{dziki}
                        \wrd{3192}{miód.}
                \bverse{5}
                        \wrd{5119}{Wtedy}
                        \wrd{1607}{wychodziła}
                        \wrd{4314}{do}
                        \wrd{846}{niego}
                        \wrd{2414}{Jerozolima}
                        \wrd{2532}{i~}
                        \wrd{3956}{cała}
                        \wrd{2449}{Judea,}
                        \wrd{2532}{i~}
                        \wrd{3956}{cała}
                        \underline{\wrd{4066}{sąsiednia okolica}}
                        \wrd{2446}{Jordanu;}
                \bverse{6}
                        \wrd{2532}{i~}
                        \underline{\wrd{907}{byli zanurzani}}
                        \wrd{5259}{przez}
                        \wrd{846}{niego}
                        \wrd{1722}{w~}
                        \wrd{2446}{Jordanie–}
                        \doubleline{\wrd{1843}{ci, którzy wyznawali}}
                        \wrd{846}{swoje}
                        \wrd{266}{grzechy.}
                \bverse{7}
                        \wrd{1161}{A~}
                        \underline{\wrd{1492}{gdy dostrzegł}}
                        \wrd{4183}{licznych}
                        \wrd{5330}{faryzeuszów}
                        \wrd{2532}{i~}
                        \wrd{4523}{saduceuszów,}
                        \wrd{2064}{przychodzących}
                        \wrd{1909}{do}
                        \wrd{}{(dokonywanego przez)}
                        \wrd{846}{niego}
                        \wrd{908}{zanurzenia,}
                        \wrd{2036}{powiedział}
                        \wrd{846}{im:}
                        \wrd{1081}{potomstwo}
                        \wrd{2191}{żmij,}
                        \wrd{5101}{kto}
                        \wrd{5213}{wam}
                        \wrd{5263}{pokazał,}
                        \doubleline{\wrd{5343}{jak uciec}}
                        \wrd{575}{od}
                        \underline{\wrd{3195}{mającego nadejść}}
                        \wrd{3709}{gniewu?}
                \bverse{8}
                        \wrd{4160}{Zrodźcie}
                        \wrd{3767}{zatem}
                        \wrd{2590}{owoc,}
                        \wrd{514}{godny}
                        \underline{\wrd{3341}{zmiany myślenia.}}
                \bverse{9}
                        \wrd{2532}{I~}
                        \wrd{3361}{nie}
                        \underline{\wrd{1380}{sądźcie, że słuszne}}
                        \wrd{}{(jest)}
                        \wrd{3004}{mówić}
                        \wrd{1722}{wobec}
                        \wrd{1438}{siebie:}
                        \wrd{2192}{mamy}
                        \wrd{3962}{ojca–}
                        \wrd{11}{Abrahama,}
                        \wrd{1063}{gdyż}
                        \wrd{3004}{mówię}
                        \wrd{5213}{wam,}
                        \wrd{3754}{że}
                        \wrd{2316}{Bóg}
                        \doubleline{\wrd{1410}{jest w stanie}}
                        \wrd{1537}{z~}
                        \wrd{5130}{tych}
                        \wrd{3037}{kamieni}
                        \wrd{1453}{wzbudzić}
                        \wrd{5043}{dzieci}
                        \wrd{11}{Abrahamowi.}
                \bverse{10}
                        \wrd{1161}{Ale}
                        \underline{\wrd{2235}{już właśnie}}
                        \wrd{2532}{i~}
                        \wrd{513}{siekiera}
                        \wrd{4314}{do}
                        \wrd{4491}{korzenia}
                        \wrd{1186}{drzew}
                        \doubleline{\wrd{2749}{jest przystawiona.}}
                        \wrd{3956}{Każde}
                        \wrd{3767}{więc}
                        \wrd{1186}{drzewo,}
                        \underline{\wrd{3361}{które nie}}
                        \wrd{4160}{rodzi}
                        \wrd{2570}{dobrego}
                        \wrd{2590}{owocu,}
                        \doubleline{\wrd{1581}{zostaje wycięte}}
                        \wrd{2532}{i~}
                        \wrd{906}{wrzucane}
                        \wrd{1519}{w~}
                        \wrd{4442}{ogień.}
                \bverse{11}
                        \wrd{1473}{Ja}
                        \wrd{3303}{[-]}
                        \wrd{5209}{was}
                        \wrd{907}{zanurzam}
                        \wrd{1722}{w~}
                        \wrd{5204}{wodzie}
                        \wrd{1519}{ku}
                        \underline{\wrd{3341}{zmianie myślenia,}}
                        \wrd{1161}{ale}
                        \doubleline{\wrd{3588}{ten, który}}
                        \wrd{2064}{przychodzi}
                        \wrd{3694}{za}
                        \wrd{3450}{mną,}
                        \wrd{2076}{jest}
                        \wrd{2478}{potężniejszy}
                        \underline{\wrd{3450}{ode mnie;}}
                        \wrd{3756}{nie}
                        \wrd{1510}{jestem}
                        \doubleline{\wrd{2425}{dostatecznie ważny,}}
                        \wrd{}{(by)}
                        \wrd{3739}{mu}
                        \wrd{5266}{sandały}
                        \wrd{941}{podnieść.}
                        \wrd{846}{On}
                        \wrd{5209}{was}
                        \wrd{907}{zanurzy}
                        \wrd{1722}{w~}
                        \wrd{4151}{Duchu}
                        \wrd{40}{Świętym}
                \bverse{12}
                        \wrd{3739}{którego}
                        \underline{\wrd{4425}{szufla do odwiewania}}
                        \doubleline{\wrd{1722}{jest w}}
                        \wrd{846}{jego}
                        \wrd{5495}{ręku;}
                        \wrd{2532}{i~}
                        \wrd{1245}{oczyści}
                        \wrd{846}{swoje}
                        \wrd{257}{klepisko,}
                        \wrd{2532}{i~}
                        \wrd{4863}{zbierze}
                        \wrd{846}{swą}
                        \wrd{4621}{pszenicę}
                        \wrd{1519}{do}
                        \wrd{596}{magazynu,}
                        \wrd{1161}{zaś}
                        \wrd{892}{plewy}
                        \wrd{2618}{spali}
                        \underline{\wrd{762}{w~}}
                        \wrd{4442}{ogniu,}
                        \underline{\wrd{762}{niemożliwym do ugaszenia.}}
                \bverse{13}
                        \wrd{5119}{Wtedy}
                        \wrd{3854}{przybył}
                        \wrd{2424}{Jezus}
                        \wrd{575}{z~}
                        \wrd{1056}{Galilei}
                        \wrd{1909}{nad}
                        \wrd{2446}{Jordan}
                        \wrd{4314}{do}
                        \wrd{2491}{Jana,}
                        \underline{\wrd{907}{by dać się}}
                        \wrd{5259}{przez}
                        \wrd{846}{niego}
                        \underline{\wrd{907}{zanurzyć.}}
                \bverse{14}
                        \wrd{1161}{Zaś}
                        \wrd{2491}{Jan}
                        \wrd{1254}{powstrzymywał}
                        \wrd{846}{go,}
                        \wrd{3004}{mówiąc:}
                        \wrd{1473}{ja}
                        \wrd{2192}{mam}
                        \wrd{5532}{potrzebę,}
                        \wrd{}{(by)}
                        \underline{\wrd{907}{być zanurzonym}}
                        \wrd{5259}{przez}
                        \wrd{4675}{ciebie,}
                        \wrd{2532}{a~}
                        \wrd{4771}{ty}
                        \wrd{2064}{przychodzisz}
                        \wrd{4314}{do}
                        \wrd{3165}{mnie?}
                \bverse{15}
                        \wrd{1161}{A~}
                        \wrd{2424}{Jezus,}
                        \wrd{611}{odpowiadając,}
                        \wrd{2036}{powiedział}
                        \wrd{4314}{do}
                        \wrd{846}{niego:}
                        \wrd{863}{dopuść}
                        \wrd{737}{teraz,}
                        \wrd{3779}{tak}
                        \wrd{1063}{bowiem}
                        \wrd{4241}{stosowne}
                        \wrd{2076}{jest}
                        \wrd{4137}{wypełnić}
                        \wrd{2254}{nam}
                        \wrd{3956}{wszelką}
                        \wrd{1343}{sprawiedliwość.}
                        \wrd{5119}{Wtedy}
                        \wrd{846}{go}
                        \wrd{863}{dopuścił.}
                \bverse{16}
                        \wrd{2424}{Jezus}
                        \wrd{2532}{zaś,}
                        \underline{\wrd{907}{gdy został zanurzony,}}
                        \wrd{2117}{natychmiast}
                        \wrd{305}{wyszedł}
                        \wrd{575}{z~}
                        \wrd{5204}{wody.}
                        \wrd{2532}{I~}
                        \wrd{2400}{oto}
                        \doubleline{\wrd{455}{otwarte zostały}}
                        \wrd{846}{mu}
                        \wrd{3772}{niebiosa}
                        \wrd{2532}{i~}
                        \wrd{1492}{dostrzegł}
                        \wrd{4151}{ducha}
                        \wrd{2316}{Boga,}
                        \wrd{2597}{zstępującego}
                        \underline{\wrd{5616}{jak gdyby}}
                        \wrd{4058}{gołąb}
                        \wrd{2532}{i~}
                        \wrd{2064}{przychodzącego}
                        \wrd{1909}{na}
                        \wrd{846}{niego.}
                \bverse{17}
                        \wrd{2532}{I~}
                        \wrd{2400}{oto}
                        \wrd{5456}{głos}
                        \wrd{1537}{z~}
                        \wrd{3772}{nieba}
                        \wrd{3004}{mówił:}
                        \wrd{3778}{Ten}
                        \wrd{2076}{jest}
                        \wrd{3450}{moim}
                        \wrd{27}{umiłowanym}
                        \wrd{5207}{synem,}
                        \wrd{1722}{w~}
                        \wrd{3739}{którym}
                        \underline{\wrd{2106}{mam upodobanie.}} \section{} \IndTwo
                \bverse{1}
                        \wrd{5119}{Wtedy}
                        \wrd{2424}{Jezus}
                        \underline{\wrd{321}{został wyprowadzony}}
                        \wrd{1519}{na}
                        \wrd{2048}{pustkowie}
                        \wrd{5259}{przez}
                        \wrd{4151}{ducha,}
                        \wrd{}{(by)}
                        \doubleline{\wrd{3985}{poddawanym próbie}}
                        \wrd{5259}{przez}
                        \wrd{1228}{diabła.}
                \bverse{2}
                        \wrd{2532}{I~}
                        \underline{\wrd{3522}{gdy przepościł}}
                        \wrd{5062}{czterdzieści}
                        \wrd{2250}{dni}
                        \wrd{2532}{i~}
                        \wrd{5062}{czterdzieści}
                        \wrd{3571}{nocy,}
                        \wrd{5305}{wreszcie}
                        \doubleline{\wrd{3983}{odczuł głód.}}
                \bverse{3}
                        \wrd{2532}{I~}
                        \underline{\wrd{4334}{gdy}}
                        \doubleline{\wrd{3985}{poddający próbie}}
                        \underline{\wrd{4334}{podszedł}}
                        \underline{\wrd{846}{do niego,}}
                        \wrd{2036}{powiedział:}
                        \wrd{1487}{jeżeli}
                        \wrd{1488}{jesteś}
                        \wrd{5207}{synem}
                        \wrd{2316}{Boga}
                        \wrd{2036}{powiedz,}
                        \wrd{2443}{aby}
                        \wrd{3778}{te}
                        \wrd{3037}{kamienie}
                        \doubleline{\wrd{1096}{stały się}}
                        \wrd{740}{chlebami.}
                \bverse{4}
                        \wrd{1161}{A~}
                        \wrd{3588}{on,}
                        \wrd{611}{odpowiadając,}
                        \wrd{2036}{powiedział:}
                        \underline{\wrd{1125}{Jest napisane:}}
                        \wrd{3756}{nie}
                        \doubleline{\wrd{1909}{z powodu}}
                        \wrd{3441}{samego}
                        \wrd{740}{chleba}
                        \wrd{444}{człowiek}
                        \underline{\wrd{2198}{będzie żył,}}
                        \wrd{235}{lecz}
                        \doubleline{\wrd{1909}{z powodu}}
                        \wrd{3956}{każdego}
                        \wrd{4487}{słowa,}
                        \wrd{1607}{wychodzącego}
                        \wrd{1223}{przez}
                        \wrd{4750}{usta}
                        \wrd{2316}{Boga.}
                \bverse{5}
                        \wrd{5119}{Wtedy}
                        \wrd{3880}{wziął}
                        \wrd{846}{go}
                        \wrd{1228}{diabeł}
                        \wrd{1519}{do}
                        \wrd{40}{świętego}
                        \wrd{4172}{miasta}
                        \wrd{2532}{i~}
                        \wrd{2476}{postawił}
                        \wrd{846}{go}
                        \wrd{1909}{na}
                        \wrd{4419}{skraju}
                        \wrd{2411}{świątyni}
                \bverse{6}
                        \wrd{2532}{i~}
                        \wrd{3004}{mówił}
                        \wrd{846}{mu:}
                        \wrd{1487}{jeśli}
                        \wrd{1488}{jesteś}
                        \wrd{5207}{synem}
                        \wrd{2316}{Boga,}
                        \wrd{906}{zrzuć}
                        \underline{\wrd{4572}{się sam}}
                        \doubleline{\wrd{2736}{na dół,}}
                        \underline{\wrd{1125}{jest}}
                        \wrd{1063}{bowiem}
                        \underline{\wrd{1125}{napisane,}}
                        \wrd{3754}{że}
                        \wrd{846}{swym}
                        \wrd{32}{aniołom}
                        \wrd{1781}{rozkaże}
                        \wrd{4012}{odnośnie}
                        \wrd{4675}{ciebie;}
                        \wrd{2532}{i~}
                        \wrd{1909}{na}
                        \wrd{5495}{rękach}
                        \wrd{4571}{cię}
                        \wrd{142}{uniosą,}
                        \doubleline{\wrd{3379}{abyś nie}}
                        \wrd{4350}{uderzył}
                        \wrd{4675}{swą}
                        \wrd{4228}{nogą}
                        \wrd{4314}{o~}
                        \wrd{3037}{kamień.}
                \bverse{7}
                        \wrd{2424}{Jezus}
                        \wrd{846}{mu}
                        \wrd{5346}{powiedział:}
                        \underline{\wrd{1125}{jest}}
                        \wrd{3825}{także}
                        \underline{\wrd{1125}{napisane:}}
                        \wrd{3756}{nie}
                        \doubleline{\wrd{1598}{będziesz wystawiał na próbę}}
                        \wrd{2962}{Pana,}
                        \wrd{4675}{twego}
                        \wrd{2316}{Boga.}
                \bverse{8}
                        \wrd{3880}{Wziął}
                        \wrd{846}{go}
                        \wrd{3825}{znów}
                        \wrd{1228}{diabeł}
                        \wrd{1519}{na}
                        \wrd{3029}{bardzo}
                        \wrd{5308}{wysoką}
                        \wrd{3735}{górę}
                        \wrd{2532}{i~}
                        \wrd{1166}{pokazał}
                        \wrd{846}{mu}
                        \wrd{3956}{wszystkie}
                        \wrd{932}{królestwa}
                        \wrd{2889}{świata}
                        \wrd{2532}{oraz}
                        \wrd{846}{ich}
                        \wrd{1391}{chwałę.}
                \bverse{9}
                        \wrd{2532}{I~}
                        \wrd{3004}{powiedział}
                        \wrd{846}{mu:}
                        \wrd{1325}{dam}
                        \wrd{5023}{to}
                        \wrd{3956}{wszystko}
                        \wrd{4671}{tobie,}
                        \wrd{1437}{jeśli}
                        \wrd{4098}{upadniesz}
                        \wrd{}{(i)}
                        \underline{\wrd{4352}{oddasz}}
                        \wrd{3427}{mi}
                        \underline{\wrd{4352}{pokłon.}}
                \bverse{10}
                        \wrd{5119}{Wtedy}
                        \wrd{2424}{Jezus}
                        \wrd{3004}{powiedział}
                        \wrd{846}{mu:}
                        \wrd{5217}{odejdź}
                        \wrd{3694}{ode}
                        \wrd{3450}{mnie,}
                        \wrd{4567}{szatanie,}
                        \wrd{1063}{ponieważ}
                        \underline{\wrd{1125}{jest napisane:}}
                        \wrd{2962}{Panu,}
                        \wrd{4675}{twemu}
                        \wrd{2316}{Bogu,}
                        \doubleline{\wrd{4352}{będziesz oddawał pokłon}}
                        \wrd{2532}{i~}
                        \wrd{846}{jemu}
                        \wrd{3441}{samemu}
                        \underline{\wrd{3000}{będziesz służył.}}
                \bverse{11}
                        \wrd{5119}{Wówczas}
                        \wrd{863}{opuścił}
                        \wrd{846}{go}
                        \wrd{1228}{diabeł,}
                        \wrd{2532}{a~}
                        \wrd{2400}{oto}
                        \wrd{32}{aniołowie}
                        \wrd{4334}{podeszli}
                        \wrd{2532}{i~}
                        \wrd{1247}{usługiwali}
                        \wrd{846}{mu.}
                \bverse{12}
                        \wrd{1161}{A~}
                        \underline{\wrd{191}{gdy}}
                        \wrd{2424}{Jezus}
                        \underline{\wrd{191}{usłyszał,}}
                        \wrd{3754}{że}
                        \wrd{2491}{Jan}
                        \doubleline{\wrd{3860}{został wydany,}}
                        \wrd{402}{odszedł}
                        \wrd{1519}{ku}
                        \wrd{1056}{Galilei.}
                \bverse{13}
                        \wrd{2532}{I~}
                        \wrd{2641}{opuściwszy}
                        \wrd{3478}{Nazaret,}
                        \wrd{2064}{przybył}
                        \wrd{}{(by)}
                        \wrd{2730}{zamieszkać}
                        \wrd{1519}{w~}
                        \wrd{2584}{Kafarnaum}
                        \wrd{3864}{nadmorskim,}
                        \wrd{1722}{w~}
                        \wrd{3725}{granicach}
                        \wrd{2194}{Zabulona}
                        \wrd{2532}{i~}
                        \wrd{3508}{Neftalego,}
                \bverse{14}
                        \wrd{2443}{aby}
                        \underline{\wrd{4137}{wypełniło się}}
                        \doubleline{\wrd{3588}{to, co}}
                        \underline{\wrd{4483}{zostało powiedziane}}
                        \doubleline{\wrd{1223}{za pośrednictwem}}
                        \wrd{4396}{proroka}
                        \wrd{2268}{Izajasza,}
                        \wrd{3004}{mówiącego:}
                \bverse{15}
                        \wrd{1093}{Ziemia}
                        \wrd{2194}{Zabulona}
                        \wrd{2532}{i~}
                        \wrd{1093}{ziemia}
                        \wrd{3508}{Neftalego,}
                        \wrd{3598}{droga}
                        \wrd{2281}{morska}
                        \underline{\wrd{4008}{po drugiej stronie}}
                        \wrd{2446}{Jordanu,}
                        \wrd{1056}{Galilea}
                        \wrd{1484}{narodów.}
                \bverse{16}
                        \wrd{2992}{Lud,}
                        \wrd{2521}{siedzący}
                        \wrd{1722}{w~}
                        \wrd{4655}{ciemności,}
                        \wrd{3708}{ujrzał}
                        \wrd{3173}{wielkie}
                        \wrd{5457}{światło,}
                        \wrd{2532}{a~}
                        \wrd{846}{im,}
                        \wrd{2521}{osiadłym}
                        \wrd{1722}{w~}
                        \wrd{5561}{krainie}
                        \wrd{2532}{i~}
                        \wrd{4639}{cieniu}
                        \wrd{2288}{śmierci,}
                        \wrd{393}{wzeszło}
                        \wrd{5457}{światło.}
                \bverse{17}
                        \wrd{575}{Od}
                        \wrd{5119}{wtedy}
                        \wrd{2424}{Jezus}
                        \wrd{756}{zaczął}
                        \wrd{2784}{głosić}
                        \wrd{2532}{i~}
                        \wrd{3004}{mówić:}
                        \underline{\wrd{3340}{zmieniajcie myślenie,}}
                        \wrd{1063}{gdyż}
                        \doubleline{\wrd{1448}{zbliżyło się}}
                        \wrd{932}{królestwo}
                        \wrd{3772}{niebios.}
                \bverse{18}
                        \wrd{4043}{Idąc}
                        \wrd{1161}{zaś}
                        \wrd{3844}{wzdłuż}
                        \wrd{2281}{morza}
                        \wrd{1056}{Galilejskiego,}
                        \wrd{1492}{dostrzegł}
                        \wrd{1417}{dwóch}
                        \wrd{80}{braci:}
                        \wrd{4613}{Szymona,}
                        \wrd{3004}{nazywanego}
                        \wrd{4074}{Piotrem,}
                        \wrd{2532}{oraz}
                        \wrd{406}{Andrzeja,}
                        \wrd{846}{jego}
                        \wrd{80}{brata,}
                        \wrd{906}{zarzucających}
                        \wrd{293}{niewód}
                        \wrd{1519}{w~}
                        \wrd{2281}{morze.}
                        \wrd{2258}{Byli}
                        \wrd{1063}{bowiem}
                        \wrd{231}{rybakami.}
                \bverse{19}
                        \wrd{2532}{I~}
                        \wrd{3004}{powiedział}
                        \wrd{846}{im:}
                        \wrd{1205}{chodźcie}
                        \wrd{3694}{za}
                        \wrd{3450}{mną,}
                        \wrd{2532}{a~}
                        \wrd{4160}{uczynię}
                        \wrd{5209}{was}
                        \wrd{231}{rybakami}
                        \wrd{444}{ludzi.}
                \bverse{20}
                        \wrd{1161}{A~}
                        \wrd{3588}{oni,}
                        \wrd{2112}{natychmiast}
                        \wrd{863}{zostawiwszy}
                        \wrd{1350}{sieci,}
                        \underline{\wrd{190}{poszli za}}
                        \wrd{846}{nim.}
                \bverse{21}
                        \wrd{2532}{I~}
                        \underline{\wrd{4260}{gdy przeszedł}}
                        \wrd{1564}{stamtąd}
                        \underline{\wrd{4260}{dalej,}}
                        \wrd{1492}{dostrzegł}
                        \wrd{1417}{dwóch}
                        \wrd{243}{innych}
                        \wrd{80}{braci:}
                        \wrd{2385}{Jakuba,}
                        \wrd{}{(syna)}
                        \wrd{3588}{owego}
                        \wrd{2199}{Zebedeusza}
                        \wrd{2532}{i~}
                        \wrd{2491}{Jana,}
                        \wrd{846}{jego}
                        \wrd{80}{brata,}
                        \wrd{}{(jak)}
                        \wrd{1722}{w~}
                        \wrd{4143}{łodzi}
                        \wrd{3326}{z~}
                        \wrd{3962}{ojcem}
                        \wrd{846}{swym,}
                        \wrd{2199}{Zebedeuszem,}
                        \wrd{2675}{przygotowywali}
                        \wrd{846}{swoje}
                        \wrd{1350}{sieci.}
                        \wrd{2532}{I~}
                        \wrd{2564}{wezwał}
                        \wrd{846}{ich.}
                \bverse{22}
                        \wrd{3588}{Oni}
                        \wrd{1161}{zaś}
                        \wrd{2112}{natychmiast,}
                        \wrd{863}{zostawiwszy}
                        \wrd{4143}{łódź}
                        \wrd{2532}{i~}
                        \wrd{3962}{ojca}
                        \wrd{846}{swego,}
                        \underline{\wrd{190}{zaczęli iść za}}
                        \wrd{846}{nim.}
                \bverse{23}
                        \wrd{2532}{I~}
                        \wrd{4013}{obchodził}
                        \wrd{2424}{Jezus}
                        \wrd{3650}{całą}
                        \wrd{1056}{Galileę,}
                        \wrd{1321}{ucząc}
                        \wrd{1722}{w~}
                        \wrd{846}{ich}
                        \wrd{4864}{synagogach,}
                        \wrd{2532}{i~}
                        \wrd{2784}{głosił}
                        \underline{\wrd{2098}{dobrą nowinę}}
                        \wrd{932}{królestwa,}
                        \wrd{2532}{i~}
                        \wrd{2323}{uzdrawiał}
                        \wrd{3956}{wszelką}
                        \wrd{3554}{chorobę}
                        \wrd{2532}{i~}
                        \wrd{3956}{każdą}
                        \wrd{3119}{niemoc}
                        \wrd{1722}{wśród}
                        \wrd{2992}{ludzi.}
                \bverse{24}
                        \wrd{2532}{I~}
                        \underline{\wrd{565}{rozeszła się}}
                        \wrd{189}{wieść}
                        \doubleline{\wrd{846}{o nim}}
                        \wrd{1519}{po}
                        \wrd{3650}{całej}
                        \wrd{4947}{Syrii}
                        \wrd{2532}{i~}
                        \wrd{4374}{znosili}
                        \wrd{846}{mu}
                        \wrd{3956}{wszystkich,}
                        \wrd{2560}{źle}
                        \underline{\wrd{2192}{się mających,}}
                        \wrd{4912}{przyciśniętych}
                        \wrd{4164}{rozmaitymi}
                        \wrd{3554}{chorobami}
                        \wrd{2532}{i~}
                        \wrd{931}{męczarniami,}
                        \wrd{2532}{i~}
                        \wrd{1139}{opętanych,}
                        \wrd{2532}{i~}
                        \wrd{4583}{epileptyków,}
                        \wrd{2532}{i~}
                        \wrd{3885}{sparaliżowanych,}
                        \wrd{2532}{i~}
                        \wrd{2323}{uzdrowił}
                        \wrd{846}{ich.}
                \bverse{25}
                        \wrd{2532}{I~}
                        \underline{\wrd{190}{szły za}}
                        \wrd{846}{nim}
                        \wrd{4183}{liczne}
                        \wrd{3793}{tłumy}
                        \wrd{575}{z~}
                        \wrd{1056}{Galilei}
                        \wrd{2532}{i~}
                        \wrd{1179}{Dekapolis,}
                        \wrd{2532}{i~}
                        \wrd{2414}{Jerozolimy,}
                        \wrd{2532}{i~}
                        \wrd{2449}{Judei,}
                        \wrd{2532}{i~}
                        \doubleline{\wrd{4008}{z drugiej strony}}
                        \wrd{2446}{Jordanu.} \section{} \IndTwo
                \bverse{1}
                        \wrd{1161}{Gdy}
                        \wrd{1492}{zobaczył}
                        \wrd{3793}{tłumy,}
                        \wrd{305}{wszedł}
                        \wrd{1519}{na}
                        \wrd{3735}{górę.}
                        \wrd{2532}{I~}
                        \underline{\wrd{2523}{gdy}}
                        \wrd{846}{sam}
                        \underline{\wrd{2523}{usiadł,}}
                        \doubleline{\wrd{4334}{podeszli do}}
                        \wrd{846}{niego}
                        \wrd{846}{jego}
                        \wrd{3101}{uczniowie.}
                \bverse{2}
                        \wrd{2532}{I~}
                        \wrd{455}{otworzywszy}
                        \wrd{4750}{usta}
                        \wrd{846}{swoje,}
                        \wrd{1321}{uczył}
                        \wrd{846}{ich,}
                        \wrd{3004}{mówiąc:}
                \bverse{3}
                        \wrd{3107}{Szczęśliwi}
                        \wrd{4434}{ubodzy}
                        \wrd{4151}{duchem,}
                        \wrd{3754}{ponieważ}
                        \wrd{846}{ich}
                        \wrd{2076}{jest}
                        \wrd{932}{królestwo}
                        \wrd{3772}{niebios.}
                \bverse{4}
                        \wrd{3107}{Szczęśliwi}
                        \underline{\wrd{3996}{smucący się,}}
                        \wrd{3754}{bo}
                        \wrd{846}{oni}
                        \doubleline{\wrd{3870}{będą pocieszeni.}}
                \bverse{5}
                        \wrd{3107}{Szczęśliwi}
                        \wrd{4239}{łagodni,}
                        \wrd{3754}{ponieważ}
                        \wrd{846}{oni}
                        \wrd{2816}{odziedziczą}
                        \wrd{1093}{ziemię.}
                \bverse{6}
                        \wrd{3107}{Szczęśliwi}
                        \underline{\wrd{3983}{odczuwający głód}}
                        \wrd{1343}{sprawiedliwości}
                        \wrd{2532}{i~}
                        \wrd{1372}{pragnący}
                        \wrd{}{(jej),}
                        \wrd{3754}{bo}
                        \wrd{846}{oni}
                        \doubleline{\wrd{5526}{będą nakarmieni.}}
                \bverse{7}
                        \wrd{3107}{Szczęśliwi}
                        \wrd{1655}{litościwi,}
                        \wrd{3754}{ponieważ}
                        \wrd{846}{oni}
                        \underline{\wrd{1653}{litości dostąpią.}}
                \bverse{8}
                        \wrd{3107}{Szczęśliwi}
                        \wrd{2513}{czyści}
                        \wrd{2588}{sercem,}
                        \wrd{3754}{bo}
                        \wrd{846}{oni}
                        \wrd{3700}{zobaczą}
                        \wrd{2316}{Boga.}
                \bverse{9}
                        \wrd{3107}{Szczęśliwi}
                        \underline{\wrd{1518}{pokój czyniący,}}
                        \wrd{3754}{bo}
                        \wrd{846}{oni}
                        \doubleline{\wrd{2564}{będą nazwani}}
                        \wrd{5207}{synami}
                        \wrd{2316}{Boga.}
                \bverse{10}
                        \wrd{3107}{Szczęśliwi}
                        \wrd{1377}{prześladowani}
                        \underline{\wrd{1752}{z powodu}}
                        \wrd{1343}{sprawiedliwości,}
                        \wrd{3754}{ponieważ}
                        \wrd{846}{ich}
                        \wrd{2076}{jest}
                        \wrd{932}{królestwo}
                        \wrd{3772}{niebios.}
                \bverse{11}
                        \wrd{3107}{Szczęśliwi}
                        \wrd{2075}{jesteście,}
                        \wrd{3752}{gdy}
                        \underline{\wrd{3679}{robią}}
                        \wrd{5209}{wam}
                        \underline{\wrd{3679}{wyrzuty,}}
                        \wrd{2532}{i~}
                        \wrd{1377}{prześladują,}
                        \wrd{2532}{i~}
                        \wrd{2036}{mówią}
                        \wrd{3956}{wszelkie}
                        \wrd{4190}{złe}
                        \wrd{4487}{słowa}
                        \wrd{2596}{przeciwko}
                        \wrd{5216}{wam,}
                        \wrd{5574}{kłamiąc}
                        \doubleline{\wrd{1752}{z powodu}}
                        \wrd{1700}{mnie.}
                \bverse{12}
                        \underline{\wrd{5463}{Radujcie się}}
                        \wrd{2532}{i~}
                        \doubleline{\wrd{21}{niezmiernie się cieszcie,}}
                        \wrd{3754}{ponieważ}
                        \wrd{3408}{zapłata}
                        \wrd{5216}{wasza}
                        \wrd{4183}{wielka}
                        \wrd{1722}{w~}
                        \wrd{3772}{niebiosach.}
                        \wrd{3779}{Tak}
                        \wrd{1063}{bowiem}
                        \wrd{1377}{prześladowali}
                        \wrd{4396}{proroków}
                        \wrd{4253}{przed}
                        \wrd{5216}{wami.}
                \bverse{13}
                        \wrd{5210}{Wy}
                        \wrd{2075}{jesteście}
                        \wrd{217}{solą}
                        \wrd{1093}{ziemi.}
                        \wrd{1437}{Gdyby}
                        \wrd{1161}{zaś}
                        \wrd{217}{sól}
                        \underline{\wrd{3471}{stała się bezużyteczna,}}
                        \wrd{1722}{[-]}
                        \wrd{5101}{czym}
                        \doubleline{\wrd{233}{będzie posolona?}}
                        \wrd{1519}{Na}
                        \wrd{3762}{nic}
                        \underline{\wrd{2480}{się}}
                        \wrd{2089}{już}
                        \wrd{}{(nie)}
                        \underline{\wrd{2480}{przydaje.}}
                        \wrd{1487}{Czy}
                        \wrd{3361}{nie}
                        \wrd{}{(na)}
                        \wrd{906}{wyrzucenie}
                        \doubleline{\wrd{1854}{na zewnątrz}}
                        \wrd{2532}{i~}
                        \wrd{2662}{zdeptanie}
                        \wrd{5259}{przez}
                        \wrd{444}{ludzi?}
                \bverse{14}
                        \wrd{5210}{Wy}
                        \wrd{2075}{jesteście}
                        \wrd{5457}{światłem}
                        \wrd{2889}{świata.}
                        \wrd{3756}{Nie}
                        \underline{\wrd{1410}{jest w stanie}}
                        \doubleline{\wrd{2928}{ukryć się}}
                        \wrd{4172}{miasto,}
                        \wrd{2749}{położone}
                        \wrd{1883}{na}
                        \wrd{3735}{górze.}
                \bverse{15}
                        \underline{\wrd{3761}{I nie}}
                        \wrd{2545}{zapala}
                        \wrd{}{(nikt)}
                        \wrd{3088}{lampy,}
                        \wrd{2532}{[-]}
                        \wrd{5087}{umieszczając}
                        \wrd{846}{ją}
                        \wrd{5259}{pod}
                        \wrd{3426}{korcem,}
                        \wrd{235}{lecz}
                        \wrd{1909}{na}
                        \wrd{3087}{podstawce;}
                        \wrd{2532}{i~}
                        \wrd{2989}{świeci}
                        \wrd{3956}{wszystkim}
                        \wrd{1722}{w~}
                        \wrd{3614}{domu.}
                \bverse{16}
                        \wrd{3779}{Tak}
                        \underline{\wrd{2989}{niech rozświeci się}}
                        \wrd{5216}{wasze}
                        \wrd{5457}{światło}
                        \wrd{1715}{przed}
                        \wrd{444}{ludźmi,}
                        \wrd{3704}{aby}
                        \wrd{1492}{dostrzegli}
                        \wrd{5216}{wasze}
                        \wrd{2570}{piękne}
                        \wrd{2041}{czyny}
                        \wrd{2532}{i~}
                        \wrd{1392}{uwielbili}
                        \wrd{3962}{Ojca}
                        \wrd{5216}{waszego}
                        \wrd{1722}{w~}
                        \wrd{3772}{niebiosach.}
                \bverse{17}
                        \wrd{3361}{Nie}
                        \wrd{3543}{sądźcie,}
                        \wrd{3754}{że}
                        \wrd{2064}{przyszedłem}
                        \wrd{2647}{zniszczyć}
                        \wrd{3551}{Torę}
                        \wrd{2228}{lub}
                        \wrd{4396}{Proroków.}
                        \wrd{3756}{Nie}
                        \wrd{2064}{przyszedłem}
                        \wrd{2647}{zniszczyć,}
                        \wrd{235}{lecz}
                        \wrd{4137}{wypełnić.}
                \bverse{18}
                        \wrd{281}{Zaprawdę}
                        \wrd{1063}{bowiem}
                        \wrd{3004}{mówię}
                        \wrd{5213}{wam:}
                        \underline{\wrd{2193}{aż do}}
                        \wrd{302}{[-]}
                        \wrd{3928}{przeminięcia}
                        \wrd{3772}{nieba}
                        \wrd{2532}{i~}
                        \wrd{1093}{ziemi,}
                        \wrd{1520}{jedna}
                        \wrd{2503}{jota}
                        \wrd{2228}{lub}
                        \wrd{1520}{jedna}
                        \wrd{2762}{kreska}
                        \wrd{3756}{nie}
                        \wrd{3361}{[-]}
                        \wrd{3928}{przeminie}
                        \wrd{575}{z~}
                        \wrd{3551}{Tory,}
                        \wrd{2193}{aż}
                        \wrd{302}{[-]}
                        \wrd{3956}{wszystko}
                        \doubleline{\wrd{1096}{się stanie.}}
                \bverse{19}
                        \wrd{1437}{Gdyby}
                        \wrd{3767}{więc}
                        \wrd{3739}{ktoś}
                        \wrd{3089}{rozwiązał}
                        \wrd{1520}{jedno}
                        \underline{\wrd{5130}{z tych}}
                        \wrd{1646}{najmniejszych}
                        \wrd{1785}{przykazań}
                        \wrd{2532}{i~}
                        \wrd{1321}{uczył}
                        \wrd{3779}{tak}
                        \wrd{444}{ludzi,}
                        \wrd{1646}{najmniejszym}
                        \doubleline{\wrd{2564}{będzie nazwany}}
                        \wrd{1722}{w~}
                        \wrd{932}{królestwie}
                        \wrd{3772}{niebios.}
                        \wrd{3739}{Kto}
                        \wrd{1161}{zaś}
                        \wrd{302}{[-]}
                        \underline{\wrd{4160}{by czynił}}
                        \wrd{2532}{i~}
                        \wrd{1321}{uczył,}
                        \wrd{3778}{ten}
                        \wrd{3173}{wielkim}
                        \doubleline{\wrd{2564}{będzie nazwany}}
                        \wrd{1722}{w~}
                        \wrd{932}{królestwie}
                        \wrd{3772}{niebios.}
                \bverse{20}
                        \wrd{1063}{Albowiem}
                        \wrd{3004}{mówię}
                        \wrd{5213}{wam,}
                        \wrd{3754}{że}
                        \wrd{1437}{jeśliby}
                        \wrd{5216}{wasza}
                        \wrd{1343}{sprawiedliwość}
                        \wrd{3361}{nie}
                        \underline{\wrd{4052}{była pełniejsza,}}
                        \underline{\wrd{4119}{była pełniejsza,}}
                        \wrd{}{(niż)}
                        \doubleline{\wrd{1122}{znawców Pisma}}
                        \wrd{2532}{i~}
                        \wrd{5330}{faryzeuszów,}
                        \wrd{3756}{nie}
                        \wrd{3361}{[-]}
                        \wrd{1525}{wejdziecie}
                        \wrd{1519}{do}
                        \wrd{932}{królestwa}
                        \wrd{3772}{niebios.}
                \bverse{21}
                        \wrd{191}{Słyszeliście,}
                        \wrd{3754}{że}
                        \wrd{4483}{powiedziano}
                        \wrd{744}{przodkom:}
                        \wrd{3756}{nie}
                        \underline{\wrd{5407}{będziesz zabijał,}}
                        \wrd{1161}{a~}
                        \wrd{3739}{kto}
                        \wrd{}{(by)}
                        \wrd{302}{[-]}
                        \wrd{5407}{zabił,}
                        \wrd{1777}{podlegać}
                        \wrd{2071}{będzie}
                        \wrd{2920}{sądowi.}
                \bverse{22}
                        \wrd{1473}{Ja}
                        \wrd{1161}{zaś}
                        \wrd{3004}{mówię}
                        \wrd{5213}{wam,}
                        \wrd{3754}{że}
                        \wrd{3956}{każdy,}
                        \wrd{3588}{kto}
                        \underline{\wrd{3710}{się gniewa}}
                        \doubleline{\wrd{1500}{bez powodu}}
                        \wrd{}{(na)}
                        \wrd{80}{brata}
                        \wrd{846}{swego,}
                        \wrd{1777}{podlegać}
                        \wrd{2071}{będzie}
                        \wrd{2920}{sądowi.}
                        \wrd{1161}{A~}
                        \wrd{3739}{kto}
                        \wrd{302}{[-]}
                        \underline{\wrd{2036}{by powiedział}}
                        \wrd{846}{swemu}
                        \wrd{80}{bratu:}
                        \doubleline{\wrd{4469}{„pusta głowo”,}}
                        \wrd{1777}{podlegać}
                        \wrd{2071}{będzie}
                        \wrd{4892}{sanhedrynowi.}
                        \wrd{1161}{A~}
                        \wrd{3739}{kto}
                        \wrd{302}{[-]}
                        \underline{\wrd{2036}{by powiedział:}}
                        \wrd{3474}{„bezbożniku”,}
                        \wrd{1777}{podlegać}
                        \wrd{2071}{będzie}
                        \wrd{1519}{[-]}
                        \wrd{4442}{ogniowi}
                        \wrd{1067}{Gehenny.}
                \bverse{23}
                        \wrd{1437}{Jeśli}
                        \wrd{3767}{więc}
                        \wrd{4374}{przyniesiesz}
                        \wrd{1435}{dar}
                        \wrd{4675}{swój}
                        \wrd{1909}{na}
                        \wrd{2379}{ołtarz}
                        \wrd{2532}{i~}
                        \wrd{1563}{tam}
                        \underline{\wrd{3415}{przypomnisz sobie,}}
                        \wrd{3754}{że}
                        \wrd{4675}{twój}
                        \wrd{80}{brat}
                        \wrd{2192}{ma}
                        \wrd{5100}{coś}
                        \wrd{2596}{przeciwko}
                        \wrd{4675}{tobie,}
                \bverse{24}
                        \wrd{863}{pozostaw}
                        \wrd{1563}{tam}
                        \wrd{1435}{dar}
                        \wrd{4675}{swój}
                        \wrd{1715}{przed}
                        \wrd{2379}{ołtarzem,}
                        \wrd{2532}{i~}
                        \wrd{5217}{idź}
                        \wrd{4412}{najpierw}
                        \underline{\wrd{1259}{pojednać się ze}}
                        \wrd{4675}{swoim}
                        \wrd{80}{bratem,}
                        \wrd{2532}{a~}
                        \wrd{5119}{wtedy}
                        \wrd{2064}{przyjdź,}
                        \wrd{4374}{przynosząc}
                        \wrd{1435}{dar}
                        \wrd{4675}{swój.}
                \bverse{25}
                        \underline{\wrd{2468}{Odmień relacje}}
                        \underline{\wrd{2132}{Odmień relacje}}
                        \doubleline{\wrd{476}{z przeciwnikiem}}
                        \wrd{4675}{swoim}
                        \wrd{5035}{szybko,}
                        \wrd{2193}{dopóki}
                        \wrd{3755}{[-]}
                        \wrd{1488}{jesteś}
                        \wrd{3326}{z~}
                        \wrd{846}{nim}
                        \wrd{1722}{w~}
                        \wrd{3598}{drodze,}
                        \underline{\wrd{3379}{aby}}
                        \wrd{476}{przeciwnik}
                        \underline{\wrd{3379}{nie}}
                        \wrd{3860}{wydał}
                        \wrd{4571}{cię}
                        \wrd{2923}{sędziemu,}
                        \wrd{2532}{a~}
                        \wrd{2923}{sędzia}
                        \wrd{3860}{wydałby}
                        \wrd{4571}{cię}
                        \wrd{5257}{podwładnemu,}
                        \wrd{2532}{i~}
                        \doubleline{\wrd{906}{wrzucono by cię}}
                        \wrd{1519}{do}
                        \wrd{5438}{więzienia.}
                \bverse{26}
                        \wrd{281}{Zaprawdę}
                        \wrd{3004}{powiadam}
                        \wrd{4671}{ci:}
                        \wrd{3756}{Nie}
                        \wrd{3361}{[-]}
                        \wrd{1831}{wyjdziesz}
                        \wrd{1564}{stamtąd,}
                        \wrd{2193}{dopóki}
                        \wrd{302}{[-]}
                        \underline{\wrd{591}{nie oddasz}}
                        \wrd{2078}{ostatniej}
                        \doubleline{\wrd{2835}{czwartej części asa.}}
                \bverse{27}
                        \wrd{191}{Słyszeliście,}
                        \wrd{3754}{że}
                        \wrd{4483}{powiedziano}
                        \wrd{3756}{Nie}
                        \wrd{3431}{cudzołóż.}
                \bverse{28}
                        \wrd{1161}{Ale}
                        \wrd{1473}{ja}
                        \wrd{5213}{wam}
                        \wrd{3004}{mówię,}
                        \wrd{3754}{że}
                        \wrd{3956}{każdy,}
                        \underline{\wrd{991}{kto patrzy na}}
                        \wrd{1135}{kobietę,}
                        \wrd{4314}{aby}
                        \wrd{846}{jej}
                        \wrd{1937}{pożądać,}
                        \wrd{2235}{już}
                        \doubleline{\wrd{3431}{popełnił}}
                        \underline{\wrd{846}{z nią}}
                        \doubleline{\wrd{3431}{cudzołóstwo}}
                        \wrd{1722}{w~}
                        \wrd{846}{swoim}
                        \wrd{2588}{sercu.}
                \bverse{29}
                        \wrd{1161}{A~}
                        \wrd{1487}{jeżeli}
                        \wrd{4675}{twoje}
                        \wrd{1188}{prawe}
                        \wrd{3788}{oko}
                        \underline{\wrd{4624}{sprawia, że upadasz,}}
                        \underline{\wrd{4571}{sprawia, że upadasz,}}
                        \wrd{1807}{wyrwij}
                        \wrd{846}{je}
                        \wrd{2532}{i~}
                        \wrd{906}{odrzuć}
                        \doubleline{\wrd{575}{z dala od}}
                        \wrd{4675}{siebie.}
                        \underline{\wrd{4851}{Pożyteczniej}}
                        \wrd{1063}{bowiem}
                        \underline{\wrd{4851}{jest}}
                        \doubleline{\wrd{4671}{dla ciebie,}}
                        \wrd{2443}{aby}
                        \wrd{622}{stracić}
                        \wrd{1520}{jeden}
                        \underline{\wrd{4675}{z twoich}}
                        \wrd{3196}{członków,}
                        \wrd{2532}{a~}
                        \wrd{3361}{nie}
                        \wrd{}{(żeby)}
                        \wrd{3650}{całe}
                        \wrd{4675}{twoje}
                        \wrd{4983}{ciało}
                        \doubleline{\wrd{906}{było wrzucone}}
                        \wrd{1519}{do}
                        \wrd{1067}{Gehenny.}
                \bverse{30}
                        \wrd{2532}{A~}
                        \wrd{1487}{jeżeli}
                        \wrd{4675}{twoja}
                        \wrd{1188}{prawa}
                        \wrd{5495}{ręka}
                        \underline{\wrd{4624}{sprawia, że upadasz,}}
                        \underline{\wrd{4571}{sprawia, że upadasz,}}
                        \wrd{1581}{wytnij}
                        \wrd{846}{ją}
                        \wrd{2532}{i~}
                        \wrd{906}{odrzuć}
                        \doubleline{\wrd{575}{z dala od}}
                        \wrd{4675}{siebie.}
                        \underline{\wrd{4851}{Pożyteczniej}}
                        \wrd{1063}{bowiem}
                        \underline{\wrd{4851}{jest}}
                        \doubleline{\wrd{4671}{dla ciebie,}}
                        \wrd{2443}{aby}
                        \wrd{622}{stracić}
                        \wrd{1520}{jeden}
                        \underline{\wrd{4675}{z twoich}}
                        \wrd{3196}{członków,}
                        \wrd{2532}{a~}
                        \wrd{3361}{nie}
                        \wrd{}{(żeby)}
                        \wrd{3650}{całe}
                        \wrd{4675}{twoje}
                        \wrd{4983}{ciało}
                        \doubleline{\wrd{906}{było wrzucone}}
                        \wrd{1519}{do}
                        \wrd{1067}{Gehenny.}
                \bverse{31}
                        \wrd{4483}{Powiedziano}
                        \wrd{1161}{też,}
                        \wrd{3754}{że}
                        \wrd{3739}{kto}
                        \wrd{302}{by}
                        \wrd{630}{uwolnił}
                        \wrd{846}{swoją}
                        \wrd{1135}{żonę,}
                        \underline{\wrd{1325}{niech}}
                        \wrd{846}{jej}
                        \underline{\wrd{1325}{da}}
                        \doubleline{\wrd{647}{akt uwolnienia.}}
                \bverse{32}
                        \wrd{1161}{Lecz}
                        \wrd{1473}{ja}
                        \wrd{5213}{wam}
                        \wrd{3004}{mówię,}
                        \wrd{3754}{że}
                        \wrd{3739}{kto}
                        \wrd{302}{by}
                        \wrd{630}{uwolnił}
                        \wrd{846}{swoją}
                        \wrd{1135}{żonę–}
                        \underline{\wrd{3924}{z wyjątkiem,}}
                        \doubleline{\wrd{3056}{gdy mowa o}}
                        \wrd{4202}{nierządzie–}
                        \wrd{4160}{czyni}
                        \wrd{846}{ją}
                        \wrd{3429}{cudzołożną.}
                        \wrd{2532}{A~}
                        \wrd{1437}{gdyby}
                        \wrd{3739}{ktoś}
                        \wrd{630}{uwolnioną}
                        \underline{\wrd{1060}{wziął za żonę–}}
                        \wrd{3429}{cudzołoży.}
                \bverse{33}
                        \wrd{191}{Słyszeliście}
                        \wrd{3825}{też,}
                        \wrd{3754}{że}
                        \wrd{4483}{powiedziano}
                        \wrd{744}{przodkom:}
                        \wrd{3756}{Nie}
                        \underline{\wrd{1964}{będziesz fałszywie przysięgał,}}
                        \wrd{1161}{ale}
                        \wrd{591}{oddasz}
                        \wrd{2962}{Panu}
                        \wrd{4675}{swe}
                        \wrd{3727}{przysięgi.}
                \bverse{34}
                        \wrd{1161}{Ale}
                        \wrd{1473}{ja}
                        \wrd{5213}{wam}
                        \wrd{3004}{mówię:}
                        \underline{\wrd{3654}{W ogóle}}
                        \wrd{3361}{nie}
                        \wrd{3660}{przysięgajcie–}
                        \wrd{3383}{ani}
                        \wrd{1722}{na}
                        \wrd{3772}{niebiosa,}
                        \wrd{3754}{ponieważ}
                        \wrd{2076}{są}
                        \wrd{2362}{tronem}
                        \wrd{2316}{Boga.}
                \bverse{35}
                        \wrd{3383}{Ani}
                        \wrd{1722}{na}
                        \wrd{1093}{ziemię,}
                        \wrd{3754}{ponieważ}
                        \wrd{2076}{jest}
                        \wrd{5286}{podnóżkiem}
                        \wrd{4228}{nóg}
                        \wrd{846}{jego.}
                        \wrd{3383}{Ani}
                        \wrd{1519}{na}
                        \wrd{2414}{Jerozolimę,}
                        \wrd{3754}{ponieważ}
                        \wrd{2076}{jest}
                        \wrd{4172}{miastem}
                        \wrd{3173}{wielkiego}
                        \wrd{935}{króla.}
                \bverse{36}
                        \underline{\wrd{3383}{Ani}}
                        \wrd{1722}{na}
                        \wrd{4675}{swoją}
                        \wrd{2776}{głowę}
                        \underline{\wrd{3383}{nie}}
                        \doubleline{\wrd{3660}{będziesz przysięgał,}}
                        \wrd{3754}{ponieważ}
                        \wrd{3756}{nie}
                        \underline{\wrd{1410}{jesteś w stanie}}
                        \wrd{1520}{jednego}
                        \wrd{2359}{włosa}
                        \wrd{4160}{uczynić}
                        \wrd{3022}{białym}
                        \wrd{2228}{albo}
                        \wrd{3189}{czarnym.}
                \bverse{37}
                        \wrd{1161}{Ale}
                        \wrd{5216}{wasza}
                        \wrd{3056}{mowa}
                        \underline{\wrd{2077}{niech będzie:}}
                        \wrd{3483}{Tak–}
                        \wrd{3483}{tak,}
                        \wrd{3756}{nie–}
                        \wrd{3756}{nie.}
                        \wrd{1161}{A~}
                        \wrd{3588}{co}
                        \wrd{4053}{ponadto,}
                        \wrd{5130}{to}
                        \wrd{1537}{od}
                        \wrd{4190}{złego}
                        \wrd{2076}{pochodzi.}
                \bverse{38}
                        \wrd{191}{Słyszeliście,}
                        \wrd{3754}{że}
                        \wrd{4483}{powiedziano:}
                        \wrd{3788}{„Oko}
                        \wrd{473}{za}
                        \wrd{3788}{oko”}
                        \wrd{2532}{oraz}
                        \wrd{3599}{„ząb}
                        \wrd{473}{za}
                        \wrd{3599}{ząb”.}
                \bverse{39}
                        \wrd{1161}{Ale}
                        \wrd{1473}{ja}
                        \wrd{5213}{wam}
                        \wrd{3004}{mówię:}
                        \wrd{3361}{Nie}
                        \underline{\wrd{436}{przeciwstawiajcie się}}
                        \wrd{4190}{złemu,}
                        \doubleline{\wrd{235}{ale jeśli}}
                        \wrd{3748}{ktoś}
                        \wrd{4474}{uderzy}
                        \wrd{4571}{cię}
                        \wrd{1909}{w~}
                        \wrd{1188}{prawy}
                        \wrd{4600}{policzek,}
                        \underline{\wrd{4762}{zwróć ku}}
                        \wrd{846}{niemu}
                        \wrd{2532}{i~}
                        \wrd{243}{drugi.}
                \bverse{40}
                        \wrd{2532}{A~}
                        \underline{\wrd{3588}{temu, kto}}
                        \doubleline{\wrd{2309}{chce się}}
                        \underline{\wrd{4671}{z tobą}}
                        \wrd{2919}{sądzić}
                        \wrd{2532}{i~}
                        \wrd{2983}{wziąć}
                        \wrd{4675}{twoją}
                        \doubleline{\wrd{5509}{szatę spodnią,}}
                        \wrd{863}{zostaw}
                        \wrd{846}{mu}
                        \wrd{2532}{i~}
                        \underline{\wrd{2440}{szatę wierzchnią.}}
                \bverse{41}
                        \wrd{2532}{I~}
                        \wrd{3748}{ktokolwiek}
                        \wrd{4571}{cię}
                        \wrd{29}{zmusza}
                        \wrd{}{(do)}
                        \wrd{1520}{jednej}
                        \wrd{3400}{mili,}
                        \wrd{5217}{przejdź}
                        \wrd{3326}{z~}
                        \wrd{846}{nim}
                        \wrd{1417}{dwie.}
                \bverse{42}
                        \underline{\wrd{3588}{Temu, kto}}
                        \wrd{4571}{cię}
                        \wrd{154}{prosi,}
                        \wrd{1325}{daj,}
                        \wrd{2532}{a~}
                        \doubleline{\wrd{3588}{od tego, kto}}
                        \wrd{2309}{chce}
                        \wrd{575}{od}
                        \wrd{4675}{ciebie}
                        \wrd{1155}{pożyczyć,}
                        \wrd{3361}{nie}
                        \underline{\wrd{654}{odwracaj się.}}
                \bverse{43}
                        \wrd{191}{Słyszeliście,}
                        \wrd{3754}{że}
                        \wrd{4483}{powiedziano:}
                        \underline{\wrd{25}{Będziesz miłował}}
                        \wrd{4675}{swego}
                        \wrd{4139}{bliźniego,}
                        \wrd{2532}{a~}
                        \wrd{4675}{swego}
                        \wrd{2190}{nieprzyjaciela}
                        \doubleline{\wrd{3404}{będziesz nienawidził?}}
                \bverse{44}
                        \wrd{1161}{A~}
                        \wrd{1473}{ja}
                        \wrd{5213}{wam}
                        \wrd{3004}{mówię:}
                        \wrd{25}{Miłujcie}
                        \wrd{5216}{waszych}
                        \wrd{2190}{nieprzyjaciół,}
                        \wrd{2127}{błogosławcie}
                        \underline{\wrd{3588}{tym, którzy}}
                        \wrd{5209}{wam}
                        \wrd{2672}{złorzeczą,}
                        \wrd{2573}{dobrze}
                        \wrd{4160}{czyńcie}
                        \doubleline{\wrd{3588}{tym, którzy}}
                        \wrd{5209}{was}
                        \wrd{3404}{nienawidzą}
                        \wrd{2532}{oraz}
                        \underline{\wrd{4336}{módlcie się}}
                        \wrd{5228}{za}
                        \doubleline{\wrd{3588}{tych, którzy}}
                        \wrd{5209}{was}
                        \wrd{1908}{oczerniają}
                        \wrd{2532}{i~}
                        \wrd{1377}{prześladują}
                        \wrd{5209}{was,}
                \bverse{45}
                        \wrd{3704}{abyście}
                        \underline{\wrd{1096}{stali się}}
                        \wrd{5207}{synami}
                        \wrd{5216}{waszego}
                        \wrd{3962}{Ojca}
                        \wrd{1722}{w~}
                        \wrd{3772}{niebiosach.}
                        \wrd{3754}{Ponieważ}
                        \wrd{846}{jego}
                        \wrd{2246}{słońce}
                        \wrd{393}{wschodzi}
                        \wrd{1909}{nad}
                        \wrd{4190}{złymi}
                        \wrd{2532}{i~}
                        \wrd{18}{dobrymi,}
                        \wrd{2532}{i~}
                        \doubleline{\wrd{1026}{deszcz zsyła}}
                        \wrd{1909}{na}
                        \wrd{1342}{sprawiedliwych}
                        \wrd{2532}{i~}
                        \wrd{94}{niesprawiedliwych.}
                \bverse{46}
                        \wrd{1437}{Gdybyście}
                        \wrd{1063}{bowiem}
                        \wrd{25}{miłowali}
                        \underline{\wrd{3588}{tych, którzy}}
                        \wrd{5209}{was}
                        \wrd{25}{miłują,}
                        \wrd{5101}{jaką}
                        \wrd{2192}{macie}
                        \wrd{3408}{zapłatę?}
                        \doubleline{\wrd{3780}{Czy nie}}
                        \wrd{4160}{czynią}
                        \wrd{846}{tego}
                        \wrd{2532}{także}
                        \underline{\wrd{5057}{poborcy podatków?}}
                \bverse{47}
                        \wrd{2532}{I~}
                        \wrd{1437}{jeśli}
                        \wrd{782}{pozdrawiacie}
                        \wrd{3440}{tylko}
                        \wrd{5216}{waszych}
                        \wrd{5384}{przyjaciół,}
                        \wrd{5101}{cóż}
                        \wrd{4053}{ponadto}
                        \wrd{4160}{czynicie?}
                        \underline{\wrd{3780}{Czy nie}}
                        \wrd{4160}{czynią}
                        \wrd{3779}{tego}
                        \wrd{2532}{także}
                        \doubleline{\wrd{5057}{poborcy podatków?}}
                \bverse{48}
                        \wrd{5210}{Wy}
                        \wrd{3767}{zatem}
                        \wrd{2071}{bądźcie}
                        \wrd{5046}{doskonali,}
                        \wrd{5618}{jak}
                        \wrd{5046}{doskonały}
                        \wrd{2076}{jest}
                        \wrd{5216}{wasz}
                        \wrd{3962}{Ojciec,}
                        \underline{\wrd{3588}{który jest}}
                        \wrd{1722}{w~}
                        \wrd{3772}{niebiosach.} \section{} \IndTwo
                \bverse{1}
                        \underline{\wrd{4337}{Pilnujcie się,}}
                        \wrd{}{(by)}
                        \wrd{5216}{waszego}
                        \doubleline{\wrd{1654}{gestu miłosierdzia}}
                        \wrd{3361}{nie}
                        \wrd{4160}{czynić}
                        \wrd{1715}{przed}
                        \wrd{444}{ludźmi}
                        \wrd{}{(po to),}
                        \wrd{4314}{aby}
                        \wrd{846}{was}
                        \wrd{2300}{widzieli,}
                        \wrd{1161}{bo}
                        \wrd{1487}{inaczej}
                        \wrd{3361}{inaczej}
                        \wrd{3756}{nie}
                        \underline{\wrd{2192}{będziecie mieli}}
                        \wrd{3408}{zapłaty}
                        \wrd{3844}{u~}
                        \wrd{5216}{waszego}
                        \wrd{3962}{Ojca–}
                        \wrd{3588}{tego}
                        \wrd{1722}{w~}
                        \wrd{3772}{niebiosach.}
                \bverse{2}
                        \wrd{3767}{Dlatego,}
                        \wrd{3752}{gdy}
                        \wrd{4160}{czynisz}
                        \underline{\wrd{1654}{gest miłosierdzia,}}
                        \wrd{3361}{nie}
                        \wrd{4537}{trąb}
                        \wrd{1715}{przed}
                        \wrd{4675}{sobą,}
                        \doubleline{\wrd{5618}{jak to}}
                        \wrd{4160}{robią}
                        \wrd{5273}{hipokryci}
                        \wrd{1722}{w~}
                        \wrd{4864}{synagogach}
                        \wrd{2532}{i~}
                        \wrd{1722}{na}
                        \wrd{4505}{ulicach,}
                        \wrd{3704}{aby}
                        \underline{\wrd{1392}{być chwalonymi}}
                        \wrd{5259}{przez}
                        \wrd{444}{ludzi.}
                        \wrd{281}{Zaprawdę}
                        \wrd{3004}{powiadam}
                        \wrd{5213}{wam:}
                        \wrd{568}{otrzymują}
                        \wrd{846}{swoją}
                        \wrd{3408}{zapłatę.}
                \bverse{3}
                        \wrd{1161}{Gdy}
                        \wrd{4675}{ty}
                        \wrd{4160}{czynisz}
                        \underline{\wrd{1654}{gest miłosierdzia,}}
                        \doubleline{\wrd{1097}{niech}}
                        \wrd{3361}{nie}
                        \doubleline{\wrd{1097}{wie}}
                        \wrd{4675}{twoja}
                        \wrd{710}{lewa}
                        \wrd{}{(ręka),}
                        \wrd{5101}{co}
                        \wrd{4160}{czyni}
                        \wrd{4675}{twoja}
                        \wrd{1188}{prawa,}
                \bverse{4}
                        \wrd{3704}{aby}
                        \wrd{4675}{twój}
                        \underline{\wrd{1654}{gest miłosierdzia}}
                        \wrd{5600}{pozostał}
                        \wrd{1722}{w~}
                        \wrd{2927}{ukryciu,}
                        \wrd{2532}{a~}
                        \wrd{4675}{twój}
                        \wrd{3962}{Ojciec,}
                        \doubleline{\wrd{991}{który widzi}}
                        \wrd{1722}{w~}
                        \wrd{2927}{ukryciu,}
                        \wrd{846}{on}
                        \wrd{591}{odda}
                        \wrd{4671}{ci}
                        \wrd{1722}{w~}
                        \underline{\wrd{5318}{sposób jawny.}}
                \bverse{5}
                        \wrd{2532}{A~}
                        \wrd{3752}{gdy}
                        \underline{\wrd{4336}{się modlicie,}}
                        \wrd{3756}{nie}
                        \wrd{2071}{bądźcie}
                        \wrd{5618}{jak}
                        \wrd{5273}{hipokryci.}
                        \wrd{3754}{Ponieważ}
                        \wrd{5368}{kochają}
                        \doubleline{\wrd{4336}{się modlić,}}
                        \wrd{2476}{stojąc}
                        \wrd{1722}{w~}
                        \wrd{4864}{synagogach}
                        \wrd{2532}{i~}
                        \wrd{1722}{na}
                        \wrd{1137}{narożnikach}
                        \wrd{4113}{ulic,}
                        \wrd{3704}{aby}
                        \wrd{302}{[-]}
                        \underline{\wrd{5316}{pokazać się}}
                        \wrd{444}{ludziom.}
                        \wrd{281}{Zaprawdę}
                        \wrd{3004}{powiadam}
                        \wrd{5213}{wam,}
                        \wrd{3754}{że}
                        \wrd{568}{odbierają}
                        \wrd{846}{swoją}
                        \wrd{3408}{zapłatę.}
                \bverse{6}
                        \wrd{1161}{Ale}
                        \wrd{3752}{gdy}
                        \wrd{4771}{ty}
                        \underline{\wrd{4336}{się modlisz,}}
                        \wrd{1525}{wejdź}
                        \wrd{1519}{do}
                        \wrd{4675}{swego}
                        \wrd{5009}{pokoju}
                        \wrd{2532}{i~}
                        \wrd{2808}{zamknij}
                        \wrd{2374}{drzwi}
                        \wrd{4675}{swoje,}
                        \doubleline{\wrd{4336}{modląc się}}
                        \underline{\wrd{4675}{do twego}}
                        \wrd{3962}{Ojca,}
                        \doubleline{\wrd{3588}{który jest}}
                        \wrd{1722}{w~}
                        \wrd{2927}{ukryciu.}
                        \wrd{2532}{A~}
                        \wrd{4675}{twój}
                        \wrd{3962}{Ojciec,}
                        \underline{\wrd{991}{który widzi}}
                        \wrd{1722}{w~}
                        \wrd{2927}{ukryciu,}
                        \wrd{591}{odda}
                        \wrd{4671}{ci}
                        \wrd{1722}{w~}
                        \doubleline{\wrd{5318}{sposób jawny.}}
                \bverse{7}
                        \wrd{1161}{A~}
                        \underline{\wrd{4336}{modląc się,}}
                        \wrd{3361}{nie}
                        \wrd{945}{paplajcie}
                        \wrd{5618}{jak}
                        \wrd{1482}{poganie,}
                        \wrd{1063}{ponieważ}
                        \wrd{1380}{sądzą,}
                        \wrd{3754}{że}
                        \doubleline{\wrd{1722}{z powodu}}
                        \wrd{846}{swojej}
                        \wrd{4180}{wielomówności}
                        \underline{\wrd{1522}{będą wysłuchani.}}
                \bverse{8}
                        \wrd{3767}{Więc}
                        \wrd{3361}{nie}
                        \underline{\wrd{3666}{upodabniajcie się}}
                        \doubleline{\wrd{846}{do nich,}}
                        \wrd{1063}{bowiem}
                        \wrd{5216}{wasz}
                        \wrd{3962}{Ojciec}
                        \wrd{1492}{dostrzega,}
                        \wrd{3739}{jakie}
                        \wrd{2192}{macie}
                        \wrd{5532}{potrzeby,}
                        \wrd{4253}{zanim}
                        \wrd{5209}{wy}
                        \wrd{846}{go}
                        \wrd{154}{poprosicie.}
                \bverse{9}
                        \wrd{5210}{Wy}
                        \wrd{3767}{więc}
                        \wrd{3779}{tak}
                        \underline{\wrd{4336}{się módlcie:}}
                        \wrd{2257}{Nasz}
                        \wrd{3962}{Ojcze}
                        \wrd{1722}{w~}
                        \wrd{3772}{niebiosach,}
                        \doubleline{\wrd{37}{niech będzie uświęcone}}
                        \wrd{4675}{twoje}
                        \wrd{3686}{imię.}
                \bverse{10}
                        \underline{\wrd{2064}{Niech przyjdzie}}
                        \wrd{4675}{twoje}
                        \wrd{932}{królestwo.}
                        \doubleline{\wrd{1096}{Niech się dzieje}}
                        \wrd{4675}{twoja}
                        \wrd{2307}{wola:}
                        \wrd{5613}{jak}
                        \wrd{1722}{w~}
                        \wrd{3772}{niebiosach,}
                        \wrd{}{(tak)}
                        \wrd{2532}{i~}
                        \wrd{1909}{na}
                        \wrd{1093}{ziemi.}
                \bverse{11}
                        \wrd{1325}{Daj}
                        \wrd{2254}{nam}
                        \wrd{4594}{dzisiaj}
                        \wrd{2257}{naszego}
                        \wrd{1967}{codziennego}
                        \wrd{740}{chleba.}
                \bverse{12}
                        \wrd{2532}{I~}
                        \wrd{863}{odpuść}
                        \wrd{2254}{nam}
                        \wrd{3783}{długi}
                        \wrd{2257}{nasze,}
                        \wrd{5613}{jak}
                        \wrd{2532}{i~}
                        \wrd{2249}{my}
                        \wrd{863}{odpuszczamy}
                        \wrd{2257}{naszym}
                        \wrd{3781}{dłużnikom.}
                \bverse{13}
                        \wrd{2532}{I~}
                        \wrd{3361}{nie}
                        \wrd{1533}{wprowadzaj}
                        \wrd{2248}{nas}
                        \wrd{1519}{w~}
                        \wrd{3986}{doświadczenie,}
                        \wrd{235}{ale}
                        \wrd{4506}{wybaw}
                        \wrd{2248}{nas}
                        \wrd{575}{od}
                        \wrd{4190}{zła,}
                        \wrd{3754}{ponieważ}
                        \wrd{4675}{twoje}
                        \wrd{2076}{jest}
                        \wrd{932}{królestwo}
                        \wrd{2532}{i~}
                        \wrd{1411}{moc,}
                        \wrd{2532}{i~}
                        \wrd{1391}{chwała}
                        \wrd{1519}{na}
                        \wrd{165}{wieczność.}
                        \wrd{281}{Amen.}
                \bverse{14}
                        \wrd{1437}{Jeśli}
                        \wrd{1063}{bowiem}
                        \wrd{863}{odpuścicie}
                        \wrd{444}{ludziom}
                        \wrd{846}{ich}
                        \wrd{3900}{upadki,}
                        \wrd{863}{odpuści}
                        \wrd{2532}{i~}
                        \wrd{5213}{wam}
                        \wrd{3962}{Ojciec}
                        \wrd{5216}{wasz}
                        \wrd{3770}{niebiański.}
                \bverse{15}
                        \wrd{1437}{Gdybyście}
                        \wrd{1161}{jednak}
                        \wrd{3361}{nie}
                        \wrd{863}{odpuścili}
                        \wrd{444}{ludziom}
                        \wrd{846}{ich}
                        \wrd{3900}{upadków,}
                        \underline{\wrd{3761}{to i}}
                        \wrd{3962}{Ojciec}
                        \wrd{5216}{wasz}
                        \underline{\wrd{3761}{nie}}
                        \wrd{863}{odpuści}
                        \wrd{3900}{upadków}
                        \wrd{5216}{waszych.}
                \bverse{16}
                        \wrd{1161}{A~}
                        \wrd{3752}{gdy}
                        \wrd{3522}{pościcie,}
                        \wrd{3361}{nie}
                        \underline{\wrd{1096}{stawajcie się}}
                        \wrd{4659}{smutni}
                        \wrd{5618}{jak}
                        \wrd{5273}{hipokryci;}
                        \wrd{853}{niszczą}
                        \wrd{1063}{bowiem}
                        \wrd{846}{swoje}
                        \wrd{4383}{oblicza,}
                        \wrd{3704}{aby}
                        \doubleline{\wrd{5316}{pokazać się}}
                        \wrd{444}{ludziom,}
                        \underline{\wrd{3522}{że poszczą.}}
                        \wrd{281}{Zaprawdę}
                        \wrd{3004}{powiadam}
                        \wrd{5213}{wam,}
                        \wrd{3754}{że}
                        \wrd{568}{otrzymują}
                        \wrd{3408}{zapłatę}
                        \wrd{846}{swoją.}
                \bverse{17}
                        \wrd{1161}{Ale}
                        \wrd{4771}{ty,}
                        \underline{\wrd{3522}{gdy pościsz,}}
                        \wrd{218}{namaść}
                        \wrd{4675}{swoją}
                        \wrd{2776}{głowę}
                        \wrd{2532}{i~}
                        \wrd{3538}{umyj}
                        \wrd{4675}{swoje}
                        \wrd{4383}{oblicze;}
                \bverse{18}
                        \wrd{3704}{Aby}
                        \wrd{3361}{nie}
                        \wrd{444}{ludziom}
                        \wrd{5316}{ukazać,}
                        \wrd{}{(że)}
                        \wrd{3522}{pościsz,}
                        \wrd{235}{lecz}
                        \wrd{4675}{twojemu}
                        \wrd{3962}{Ojcu,}
                        \underline{\wrd{3588}{który jest}}
                        \wrd{1722}{w~}
                        \wrd{2927}{ukryciu;}
                        \wrd{2532}{a~}
                        \wrd{4675}{twój}
                        \wrd{3962}{Ojciec,}
                        \wrd{3588}{który}
                        \wrd{991}{widzi}
                        \wrd{1722}{w~}
                        \wrd{2927}{ukryciu,}
                        \wrd{591}{odda}
                        \wrd{4671}{ci}
                \bverse{19}
                        \wrd{3361}{Nie}
                        \wrd{2343}{gromadźcie}
                        \wrd{5213}{sobie}
                        \wrd{2344}{skarbów}
                        \wrd{1909}{na}
                        \wrd{1093}{ziemi,}
                        \wrd{3699}{gdzie}
                        \wrd{4597}{mól}
                        \wrd{2532}{i~}
                        \wrd{1035}{rdza}
                        \wrd{853}{niszczą,}
                        \wrd{2532}{i~}
                        \wrd{3699}{gdzie}
                        \wrd{2812}{złodzieje}
                        \underline{\wrd{1358}{podkopują się}}
                        \wrd{2532}{i~}
                        \wrd{2813}{kradną;}
                \bverse{20}
                        \wrd{1161}{Ale}
                        \wrd{2343}{gromadźcie}
                        \wrd{5213}{sobie}
                        \wrd{2344}{skarby}
                        \wrd{1722}{w~}
                        \wrd{3772}{niebiosach,}
                        \wrd{3699}{gdzie}
                        \wrd{3777}{ani}
                        \wrd{4597}{mól,}
                        \wrd{3777}{ani}
                        \wrd{1035}{rdza}
                        \wrd{}{(nie)}
                        \wrd{853}{niszczą}
                        \wrd{2532}{i~}
                        \wrd{3699}{gdzie}
                        \wrd{2812}{złodzieje}
                        \wrd{3756}{nie}
                        \underline{\wrd{1358}{podkopują się}}
                        \doubleline{\wrd{3761}{i nie}}
                        \wrd{2813}{kradną.}
                \bverse{21}
                        \wrd{3699}{Gdzie}
                        \wrd{1063}{bowiem}
                        \wrd{2076}{jest}
                        \wrd{2344}{skarb}
                        \wrd{5216}{wasz,}
                        \wrd{1563}{tam}
                        \wrd{2071}{będzie}
                        \wrd{2532}{i~}
                        \wrd{2588}{serce}
                        \wrd{5216}{wasze.}
                \bverse{22}
                        \wrd{3088}{Lampą}
                        \wrd{}{(dla)}
                        \wrd{4983}{ciała}
                        \wrd{2076}{jest}
                        \wrd{3788}{oko.}
                        \wrd{1437}{Jeśli}
                        \wrd{3767}{więc}
                        \wrd{4675}{twoje}
                        \wrd{3788}{oko}
                        \wrd{5600}{jest}
                        \wrd{573}{szczere,}
                        \wrd{3650}{całe}
                        \wrd{4983}{ciało}
                        \wrd{4675}{twoje}
                        \wrd{2071}{jest}
                        \wrd{5460}{jasne.}
                \bverse{23}
                        \wrd{1437}{Jeśli}
                        \wrd{1161}{zaś}
                        \wrd{4675}{twoje}
                        \wrd{3788}{oko}
                        \wrd{5600}{jest}
                        \wrd{4190}{złe,}
                        \wrd{3650}{całe}
                        \wrd{4675}{twoje}
                        \wrd{4983}{ciało}
                        \wrd{2071}{jest}
                        \wrd{4652}{ciemne.}
                        \wrd{1487}{Jeśli}
                        \wrd{3767}{więc}
                        \wrd{5457}{światło}
                        \wrd{1722}{w~}
                        \wrd{4671}{tobie}
                        \wrd{2076}{jest}
                        \wrd{4655}{ciemnością,}
                        \underline{\wrd{4214}{jakże wielka}}
                        \wrd{}{(to)}
                        \wrd{4655}{ciemność.}
                \bverse{24}
                        \underline{\wrd{3762}{Nikt nie}}
                        \doubleline{\wrd{1410}{jest w stanie}}
                        \wrd{1417}{dwom}
                        \wrd{2962}{panom}
                        \wrd{1398}{służyć,}
                        \wrd{1063}{gdyż}
                        \wrd{2228}{albo}
                        \wrd{1520}{jednego}
                        \wrd{3404}{znienawidzi,}
                        \wrd{2532}{a~}
                        \wrd{2087}{drugiego}
                        \wrd{25}{umiłuje,}
                        \wrd{2228}{albo}
                        \wrd{1520}{jednego}
                        \underline{\wrd{472}{będzie się trzymał,}}
                        \wrd{2532}{a~}
                        \wrd{2087}{drugim}
                        \wrd{2706}{pogardzi.}
                        \wrd{3756}{Nie}
                        \doubleline{\wrd{1410}{jesteście w stanie}}
                        \wrd{1398}{służyć}
                        \wrd{2316}{Bogu}
                        \wrd{2532}{i~}
                        \wrd{3126}{mamonie.}
                \bverse{25}
                        \wrd{1223}{Dlatego}
                        \wrd{5124}{Dlatego}
                        \wrd{3004}{mówię}
                        \wrd{5213}{wam:}
                        \wrd{3361}{Nie}
                        \underline{\wrd{3309}{troszczcie się o}}
                        \wrd{5216}{waszą}
                        \wrd{5590}{duszę,}
                        \wrd{5101}{co}
                        \doubleline{\wrd{5315}{będziecie jeść}}
                        \wrd{2532}{albo}
                        \wrd{5101}{co}
                        \underline{\wrd{4095}{będziecie pić,}}
                        \doubleline{\wrd{3366}{ani o}}
                        \wrd{5216}{wasze}
                        \wrd{4983}{ciało,}
                        \wrd{5101}{co}
                        \wrd{}{(nań)}
                        \wrd{1746}{przyoblec.}
                        \underline{\wrd{3780}{Czy}}
                        \wrd{5590}{dusza}
                        \underline{\wrd{3780}{nie}}
                        \wrd{2076}{jest}
                        \doubleline{\wrd{4119}{czymś więcej,}}
                        \wrd{}{(niż)}
                        \wrd{5160}{pokarm,}
                        \wrd{2532}{a~}
                        \wrd{4983}{ciało,}
                        \wrd{}{(niż)}
                        \wrd{1742}{odzienie?}
                \bverse{26}
                        \wrd{1689}{Popatrzcie}
                        \wrd{1519}{na}
                        \wrd{4071}{ptaki}
                        \wrd{3772}{niebios,}
                        \wrd{3754}{że}
                        \wrd{3756}{nie}
                        \wrd{4687}{sieją}
                        \wrd{3761}{ani}
                        \wrd{}{(nie)}
                        \wrd{2325}{żną,}
                        \wrd{3761}{ani}
                        \wrd{}{(nie)}
                        \wrd{4863}{zbierają}
                        \wrd{1519}{do}
                        \wrd{596}{magazynów,}
                        \wrd{2532}{a~}
                        \wrd{3962}{Ojciec}
                        \wrd{5216}{wasz}
                        \wrd{3770}{niebiański}
                        \wrd{5142}{żywi}
                        \wrd{846}{je.}
                        \underline{\wrd{3756}{Czy}}
                        \wrd{5210}{wy}
                        \underline{\wrd{3756}{nie}}
                        \wrd{}{(jesteście)}
                        \wrd{3123}{dużo}
                        \wrd{1308}{ważniejsi,}
                        \wrd{}{(niż)}
                        \wrd{846}{one?}
                \bverse{27}
                        \wrd{1161}{A~}
                        \wrd{5101}{kto}
                        \wrd{1537}{z~}
                        \wrd{5216}{was,}
                        \underline{\wrd{3309}{troszcząc się,}}
                        \doubleline{\wrd{1410}{jest w stanie}}
                        \wrd{4369}{dodać}
                        \wrd{1909}{do}
                        \wrd{846}{swego}
                        \wrd{2244}{wzrostu}
                        \wrd{1520}{jeden}
                        \wrd{4083}{łokieć?}
                \bverse{28}
                        \wrd{2532}{A~}
                        \wrd{4012}{o~}
                        \wrd{1742}{odzienie}
                        \wrd{5101}{dlaczego}
                        \underline{\wrd{3309}{się troszczycie?}}
                        \doubleline{\wrd{2648}{Przypatrzcie się}}
                        \wrd{2918}{liliom}
                        \wrd{68}{polnym,}
                        \wrd{4459}{jak}
                        \wrd{837}{rosną.}
                        \wrd{3756}{Nie}
                        \underline{\wrd{2872}{trudzą się,}}
                        \wrd{3761}{ani}
                        \wrd{}{(nie)}
                        \wrd{3514}{przędą;}
                \bverse{29}
                        \wrd{1161}{A~}
                        \wrd{3004}{mówię}
                        \wrd{5213}{wam,}
                        \wrd{3754}{że}
                        \wrd{3761}{nawet}
                        \wrd{4672}{Salomon}
                        \wrd{1722}{w~}
                        \wrd{3956}{całej}
                        \wrd{1391}{chwale}
                        \wrd{846}{swojej}
                        \wrd{}{(nie)}
                        \underline{\wrd{5613}{był}}
                        \wrd{}{(tak)}
                        \wrd{4016}{odziany,}
                        \underline{\wrd{5613}{jak}}
                        \wrd{1520}{jedna}
                        \doubleline{\wrd{5130}{z nich.}}
                \bverse{30}
                        \wrd{1487}{Jeśli}
                        \wrd{1161}{więc}
                        \wrd{5528}{trawę}
                        \wrd{68}{polną,}
                        \underline{\wrd{5607}{która}}
                        \wrd{4594}{dziś}
                        \underline{\wrd{5607}{jest,}}
                        \wrd{2532}{a~}
                        \wrd{839}{jutro}
                        \wrd{1519}{do}
                        \wrd{2823}{pieca}
                        \doubleline{\wrd{906}{będzie wrzucona,}}
                        \wrd{2316}{Bóg}
                        \wrd{3779}{tak}
                        \wrd{294}{odziewa,}
                        \wrd{}{(czyż)}
                        \wrd{3756}{nie}
                        \wrd{4183}{dużo}
                        \wrd{3123}{bardziej}
                        \wrd{5209}{was,}
                        \wrd{}{(ludzie)}
                        \underline{\wrd{3640}{małej wiary?}}
                \bverse{31}
                        \wrd{3361}{Nie}
                        \underline{\wrd{3309}{troszczcie się}}
                        \wrd{3767}{więc,}
                        \wrd{3004}{mówiąc:}
                        \wrd{5101}{Cóż}
                        \doubleline{\wrd{5315}{będziemy jeść?}}
                        \wrd{2228}{Albo:}
                        \wrd{5101}{Co}
                        \underline{\wrd{4095}{będziemy pić?}}
                        \wrd{2228}{Albo:}
                        \doubleline{\wrd{5101}{W co}}
                        \underline{\wrd{4016}{się odziejemy?}}
                \bverse{32}
                        \wrd{1063}{Bowiem}
                        \wrd{5023}{tego}
                        \wrd{3956}{wszystkiego}
                        \wrd{1484}{narody}
                        \wrd{1934}{pragną.}
                        \wrd{1063}{Bo}
                        \wrd{3962}{Ojciec}
                        \wrd{5216}{wasz}
                        \wrd{3770}{niebiański}
                        \wrd{1492}{dostrzega,}
                        \wrd{3754}{że}
                        \wrd{5535}{potrzebujecie}
                        \wrd{5130}{tego}
                        \wrd{537}{wszystkiego.}
                \bverse{33}
                        \wrd{1161}{Ale}
                        \wrd{2212}{szukajcie}
                        \underline{\wrd{4412}{przede wszystkim}}
                        \wrd{932}{królestwa}
                        \wrd{2316}{bożego}
                        \wrd{2532}{i~}
                        \wrd{846}{jego}
                        \wrd{1343}{sprawiedliwości,}
                        \wrd{2532}{a~}
                        \wrd{5023}{to}
                        \wrd{3956}{wszystko}
                        \doubleline{\wrd{4369}{będzie}}
                        \wrd{5213}{wam}
                        \doubleline{\wrd{4369}{dodane.}}
                \bverse{34}
                        \wrd{3361}{Nie}
                        \underline{\wrd{3309}{troszczcie się}}
                        \wrd{3767}{więc}
                        \wrd{1519}{o~}
                        \wrd{839}{jutro,}
                        \wrd{1063}{bowiem}
                        \wrd{839}{jutro}
                        \doubleline{\wrd{3309}{będzie się troszczyło}}
                        \underline{\wrd{1438}{o siebie.}}
                        \wrd{713}{Dosyć}
                        \wrd{}{(ma)}
                        \wrd{2250}{dzień}
                        \wrd{2549}{zła}
                        \wrd{846}{swego.} \section{} \IndTwo
                \bverse{1}
                        \wrd{3361}{Nie}
                        \wrd{2919}{sądźcie,}
                        \wrd{2443}{abyście}
                        \wrd{3361}{nie}
                        \underline{\wrd{2919}{byli osądzeni.}}
                \bverse{2}
                        \wrd{1722}{Jakim}
                        \wrd{3739}{Jakim}
                        \wrd{1063}{bowiem}
                        \wrd{2917}{sądem}
                        \wrd{2919}{sądzicie,}
                        \wrd{}{(takim)}
                        \underline{\wrd{2919}{będziecie osądzeni,}}
                        \wrd{2532}{i~}
                        \wrd{1722}{jaką}
                        \wrd{3739}{jaką}
                        \wrd{3358}{miarą}
                        \wrd{3354}{mierzycie,}
                        \wrd{}{(taką)}
                        \doubleline{\wrd{3354}{będzie}}
                        \wrd{5213}{wam}
                        \doubleline{\wrd{3354}{odmierzone.}}
                \bverse{3}
                        \wrd{1161}{A~}
                        \wrd{5101}{czemu}
                        \wrd{991}{widzisz}
                        \wrd{2595}{źdźbło}
                        \wrd{1722}{w~}
                        \wrd{3788}{oku}
                        \wrd{80}{brata}
                        \wrd{4675}{swego,}
                        \wrd{1161}{a~}
                        \wrd{1722}{w~}
                        \wrd{4674}{swoim}
                        \wrd{3788}{oku}
                        \wrd{1385}{belki}
                        \wrd{3756}{nie}
                        \wrd{2657}{dostrzegasz?}
                \bverse{4}
                        \wrd{2228}{Albo}
                        \wrd{4459}{jak}
                        \wrd{2046}{powiesz}
                        \wrd{80}{bratu}
                        \wrd{4675}{swojemu:}
                        \wrd{863}{Pozwól,}
                        \underline{\wrd{1544}{niech wyrzucę}}
                        \wrd{2595}{źdźbło}
                        \wrd{575}{z~}
                        \wrd{3788}{oka}
                        \wrd{4675}{twego,}
                        \wrd{2532}{a~}
                        \wrd{2400}{oto}
                        \wrd{1385}{belka}
                        \wrd{}{(jest)}
                        \wrd{1722}{w~}
                        \wrd{3788}{oku}
                        \wrd{4675}{twoim?}
                \bverse{5}
                        \wrd{5273}{Hipokryto,}
                        \wrd{1544}{wyrzuć}
                        \underline{\wrd{4412}{przede wszystkim}}
                        \wrd{1385}{belkę}
                        \wrd{1537}{z~}
                        \wrd{3788}{oka}
                        \wrd{4675}{swojego,}
                        \wrd{2532}{a~}
                        \wrd{5119}{wtedy}
                        \wrd{1227}{przejrzysz,}
                        \wrd{}{(aby)}
                        \wrd{1544}{wyrzucić}
                        \wrd{2595}{źdźbło}
                        \wrd{1537}{z~}
                        \wrd{3788}{oka}
                        \wrd{80}{brata}
                        \wrd{4675}{swojego.}
                \bverse{6}
                        \wrd{3361}{Nie}
                        \wrd{1325}{dawajcie}
                        \wrd{2965}{psom}
                        \underline{\wrd{3588}{tego, co}}
                        \wrd{40}{święte,}
                        \doubleline{\wrd{3366}{ani nie}}
                        \wrd{906}{rzucajcie}
                        \wrd{5216}{swoich}
                        \wrd{3135}{pereł}
                        \wrd{1715}{przed}
                        \wrd{5519}{świnie,}
                        \underline{\wrd{3379}{aby nie}}
                        \wrd{2662}{zdeptały}
                        \wrd{846}{ich}
                        \wrd{846}{swoimi}
                        \wrd{1722}{[-]}
                        \wrd{4228}{nogami,}
                        \wrd{2532}{a~}
                        \doubleline{\wrd{4762}{obróciwszy się–}}
                        \wrd{}{(nie)}
                        \wrd{4486}{roztrzaskały}
                        \wrd{5209}{was.}
                \bverse{7}
                        \wrd{154}{Proście,}
                        \wrd{2532}{a~}
                        \underline{\wrd{1325}{będzie}}
                        \wrd{5213}{wam}
                        \underline{\wrd{1325}{dane;}}
                        \wrd{2212}{szukajcie,}
                        \wrd{2532}{a~}
                        \wrd{2147}{znajdziecie;}
                        \wrd{2925}{pukajcie,}
                        \wrd{2532}{a~}
                        \doubleline{\wrd{455}{będzie}}
                        \wrd{5213}{wam}
                        \doubleline{\wrd{455}{otworzone.}}
                \bverse{8}
                        \wrd{3956}{Każdy}
                        \wrd{1063}{bowiem}
                        \wrd{154}{proszący–}
                        \wrd{2983}{dostaje,}
                        \wrd{2532}{a~}
                        \wrd{2212}{szukający–}
                        \wrd{2147}{znajduje,}
                        \wrd{2532}{a~}
                        \wrd{2925}{pukającemu}
                        \underline{\wrd{455}{będzie otworzone.}}
                \bverse{9}
                        \wrd{2228}{Albo}
                        \wrd{5101}{kto}
                        \wrd{1537}{z~}
                        \wrd{5216}{was}
                        \wrd{2076}{jest}
                        \wrd{444}{człowiekiem,}
                        \wrd{3739}{który,}
                        \wrd{1437}{gdy}
                        \wrd{846}{jego}
                        \wrd{5207}{syn}
                        \wrd{154}{poprosi}
                        \wrd{}{(o)}
                        \wrd{740}{chleb,}
                        \wrd{3361}{[-]}
                        \wrd{3037}{kamień}
                        \wrd{846}{mu}
                        \wrd{1929}{poda?}
                \bverse{10}
                        \wrd{2532}{Lub}
                        \wrd{1437}{gdy}
                        \wrd{}{(o)}
                        \wrd{2486}{rybę}
                        \wrd{154}{poprosi,}
                        \wrd{3361}{[-]}
                        \wrd{3789}{węża}
                        \wrd{846}{mu}
                        \wrd{1929}{poda?}
                \bverse{11}
                        \wrd{1487}{Jeśli}
                        \wrd{3767}{więc}
                        \wrd{5210}{wy,}
                        \wrd{5607}{będąc}
                        \wrd{4190}{złymi,}
                        \wrd{1492}{umiecie}
                        \wrd{1325}{dawać}
                        \wrd{18}{dobre}
                        \wrd{1390}{dary}
                        \wrd{5216}{swoim}
                        \wrd{5043}{dzieciom,}
                        \underline{\wrd{4214}{o ile}}
                        \wrd{3123}{bardziej}
                        \wrd{3962}{Ojciec}
                        \wrd{5216}{wasz}
                        \wrd{1722}{w~}
                        \wrd{3772}{niebiosach}
                        \wrd{1325}{da}
                        \wrd{154}{proszącym}
                        \wrd{846}{go}
                        \wrd{}{(to, co)}
                        \wrd{18}{dobre.}
                \bverse{12}
                        \wrd{3956}{Wszystko}
                        \wrd{3767}{więc,}
                        \wrd{3745}{co}
                        \wrd{302}{[-]}
                        \underline{\wrd{2309}{byście chcieli,}}
                        \wrd{2443}{aby}
                        \wrd{5213}{wam}
                        \wrd{444}{ludzie}
                        \wrd{4160}{czynili,}
                        \wrd{3779}{tak}
                        \wrd{2532}{i~}
                        \wrd{5210}{wy}
                        \wrd{846}{im}
                        \wrd{4160}{czyńcie.}
                        \wrd{3778}{Taka}
                        \wrd{2076}{jest}
                        \wrd{1063}{bowiem}
                        \wrd{3551}{Tora}
                        \wrd{2532}{i~}
                        \wrd{4396}{Prorocy.}
                \bverse{13}
                        \wrd{1525}{Wejdźcie}
                        \wrd{1223}{przez}
                        \wrd{4728}{wąską}
                        \wrd{4439}{bramę.}
                        \wrd{3754}{Bo}
                        \wrd{4116}{szeroka}
                        \wrd{}{(jest)}
                        \wrd{4439}{brama}
                        \wrd{2532}{i~}
                        \wrd{2149}{szeroka}
                        \wrd{3598}{droga,}
                        \wrd{3588}{która}
                        \wrd{520}{prowadzi}
                        \wrd{1519}{na}
                        \wrd{684}{zniszczenie,}
                        \wrd{2532}{i~}
                        \wrd{4183}{liczni}
                        \wrd{1526}{są}
                        \wrd{}{(ci, którzy)}
                        \wrd{1223}{przez}
                        \wrd{846}{nią}
                        \wrd{1525}{wchodzą.}
                \bverse{14}
                        \wrd{}{(A)}
                        \underline{\wrd{5101}{cóż za}}
                        \wrd{4728}{wąska}
                        \wrd{}{(jest)}
                        \wrd{4439}{brama}
                        \wrd{2532}{i~}
                        \wrd{2346}{ciasna}
                        \wrd{3598}{droga,}
                        \wrd{3588}{która}
                        \wrd{520}{prowadzi}
                        \wrd{1519}{do}
                        \wrd{2222}{życia,}
                        \wrd{2532}{a~}
                        \wrd{3641}{nieliczni}
                        \wrd{1526}{są,}
                        \wrd{}{(którzy)}
                        \wrd{846}{ją}
                        \wrd{2147}{znajdują.}
                \bverse{15}
                        \wrd{1161}{Ale}
                        \underline{\wrd{4337}{wystrzegajcie się}}
                        \wrd{575}{[-]}
                        \doubleline{\wrd{5578}{fałszywych proroków,}}
                        \wrd{3748}{którzy}
                        \wrd{2064}{przychodzą}
                        \wrd{4314}{do}
                        \wrd{5209}{was}
                        \wrd{1722}{w~}
                        \wrd{1742}{odzieniu}
                        \wrd{4263}{owiec,}
                        \wrd{1161}{a~}
                        \wrd{2081}{wewnątrz}
                        \wrd{1526}{są}
                        \wrd{727}{rabującymi}
                        \wrd{3074}{wilkami.}
                \bverse{16}
                        \wrd{575}{Po}
                        \wrd{846}{ich}
                        \wrd{2590}{owocach}
                        \wrd{846}{ich}
                        \wrd{1921}{poznacie.}
                        \wrd{3385}{Czy}
                        \underline{\wrd{4816}{zbiera się}}
                        \wrd{575}{z~}
                        \wrd{173}{ciernia}
                        \doubleline{\wrd{4718}{kiście winogron,}}
                        \wrd{2228}{albo}
                        \wrd{575}{z~}
                        \wrd{5146}{ostu–}
                        \wrd{4810}{figi?}
                \bverse{17}
                        \wrd{3779}{Tak}
                        \wrd{3956}{każde}
                        \wrd{18}{dobre}
                        \wrd{1186}{drzewo}
                        \wrd{4160}{rodzi}
                        \wrd{2570}{piękne}
                        \wrd{2590}{owoce,}
                        \wrd{1161}{a~}
                        \wrd{4550}{zgniłe}
                        \wrd{1186}{drzewo–}
                        \wrd{4160}{rodzi}
                        \wrd{4190}{złe}
                        \wrd{2590}{owoce.}
                \bverse{18}
                        \wrd{3756}{Nie}
                        \underline{\wrd{1410}{jest w stanie}}
                        \wrd{18}{dobre}
                        \wrd{1186}{drzewo}
                        \wrd{4160}{rodzić}
                        \wrd{4190}{złych}
                        \wrd{2590}{owoców,}
                        \wrd{3761}{ani}
                        \wrd{4550}{zgniłe}
                        \wrd{1186}{drzewo–}
                        \wrd{4160}{rodzić}
                        \wrd{2570}{pięknych}
                        \wrd{2590}{owoców.}
                \bverse{19}
                        \wrd{3956}{Każde}
                        \wrd{1186}{drzewo,}
                        \underline{\wrd{4160}{które}}
                        \wrd{3361}{nie}
                        \underline{\wrd{4160}{rodzi}}
                        \wrd{2570}{pięknego}
                        \wrd{2590}{owocu,}
                        \doubleline{\wrd{1581}{jest wycinane}}
                        \wrd{2532}{i~}
                        \wrd{906}{wrzucane}
                        \wrd{1519}{w~}
                        \wrd{4442}{ogień.}
                \bverse{20}
                        \underline{\wrd{686}{A zatem}}
                        \wrd{1065}{[-]}
                        \wrd{575}{z~}
                        \wrd{846}{ich}
                        \wrd{2590}{owoców}
                        \wrd{1921}{poznacie}
                        \wrd{846}{ich.}
                \bverse{21}
                        \wrd{3756}{Nie}
                        \wrd{3956}{każdy,}
                        \underline{\wrd{3004}{który}}
                        \wrd{3427}{mi}
                        \underline{\wrd{3004}{mówi:}}
                        \wrd{2962}{„Panie,}
                        \wrd{2962}{Panie”,}
                        \wrd{1525}{wejdzie}
                        \wrd{1519}{do}
                        \wrd{932}{królestwa}
                        \wrd{3772}{niebios,}
                        \wrd{235}{ale}
                        \doubleline{\wrd{4160}{ten, który czyni}}
                        \wrd{2307}{wolę}
                        \wrd{3450}{mojego}
                        \wrd{3962}{Ojca}
                        \wrd{1722}{w~}
                        \wrd{3772}{niebiosach.}
                \bverse{22}
                        \wrd{4183}{Wielu}
                        \wrd{2046}{powie}
                        \wrd{3427}{mi}
                        \wrd{1722}{w~}
                        \wrd{1565}{tamtym}
                        \wrd{2250}{dniu:}
                        \wrd{2962}{„Panie,}
                        \wrd{2962}{Panie,}
                        \wrd{3756}{nie}
                        \wrd{}{(w)}
                        \wrd{4674}{twoim}
                        \wrd{3686}{imieniu}
                        \wrd{4395}{prorokowaliśmy?}
                        \wrd{2532}{I~}
                        \wrd{4674}{twoim}
                        \wrd{3686}{imieniem}
                        \wrd{1140}{demony}
                        \wrd{1544}{wyrzuciliśmy!}
                        \wrd{2532}{I~}
                        \wrd{4674}{twoim}
                        \wrd{3686}{imieniem}
                        \wrd{4160}{uczyniliśmy}
                        \wrd{4183}{liczne}
                        \wrd{1411}{cuda!”}
                \bverse{23}
                        \wrd{2532}{A~}
                        \wrd{5119}{wtedy}
                        \wrd{3670}{wyznam}
                        \wrd{846}{im,}
                        \wrd{3754}{że:}
                        \underline{\wrd{3763}{„Nigdy}}
                        \wrd{5209}{was}
                        \underline{\wrd{3763}{nie}}
                        \wrd{1097}{poznałem.}
                        \wrd{672}{Odejdźcie}
                        \wrd{575}{ode}
                        \wrd{1700}{mnie,}
                        \wrd{2038}{czyniący}
                        \wrd{458}{nieprawość”.}
                \bverse{24}
                        \wrd{3956}{Każdy}
                        \wrd{3767}{więc,}
                        \wrd{3748}{kto}
                        \wrd{5128}{tych}
                        \wrd{3056}{słów}
                        \wrd{3450}{moich}
                        \wrd{191}{słucha}
                        \wrd{2532}{i~}
                        \wrd{4160}{czyni}
                        \wrd{846}{je,}
                        \wrd{846}{ten}
                        \underline{\wrd{3666}{będzie podobny}}
                        \wrd{}{(do)}
                        \wrd{5429}{mądrego}
                        \wrd{435}{męża,}
                        \wrd{3748}{który}
                        \wrd{3618}{zbudował}
                        \wrd{846}{swój}
                        \wrd{3614}{dom}
                        \wrd{1909}{na}
                        \wrd{4073}{skale.}
                \bverse{25}
                        \wrd{2532}{I~}
                        \wrd{2597}{zstąpił}
                        \wrd{1028}{deszcz,}
                        \wrd{2532}{i~}
                        \wrd{2064}{przybyły}
                        \wrd{4215}{rzeki,}
                        \wrd{2532}{i~}
                        \wrd{4154}{zawiały}
                        \wrd{417}{wiatry,}
                        \wrd{2532}{i~}
                        \wrd{4363}{wpadły}
                        \wrd{}{(na)}
                        \wrd{1565}{ten}
                        \wrd{3614}{dom,}
                        \wrd{2532}{ale}
                        \wrd{3756}{nie}
                        \wrd{4098}{upadł,}
                        \underline{\wrd{2311}{utwierdzony był}}
                        \wrd{1063}{bowiem}
                        \wrd{1909}{na}
                        \wrd{4073}{skale.}
                \bverse{26}
                        \wrd{2532}{A~}
                        \wrd{3956}{każdy,}
                        \underline{\wrd{191}{kto słucha}}
                        \wrd{5128}{tych}
                        \wrd{3056}{słów}
                        \wrd{3450}{moich,}
                        \wrd{2532}{ale}
                        \wrd{3361}{nie}
                        \wrd{4160}{czyni}
                        \wrd{846}{ich,}
                        \doubleline{\wrd{3666}{będzie podobny}}
                        \wrd{}{(do)}
                        \wrd{3474}{głupiego}
                        \wrd{435}{męża,}
                        \wrd{3748}{który}
                        \wrd{3618}{zbudował}
                        \wrd{846}{swój}
                        \wrd{3614}{dom}
                        \wrd{1909}{na}
                        \wrd{285}{piasku.}
                \bverse{27}
                        \wrd{2532}{I~}
                        \wrd{2597}{zstąpił}
                        \wrd{1028}{deszcz,}
                        \wrd{2532}{i~}
                        \wrd{2064}{przybyły}
                        \wrd{4215}{rzeki,}
                        \wrd{2532}{i~}
                        \wrd{4154}{zawiały}
                        \wrd{417}{wiatry,}
                        \wrd{2532}{i~}
                        \wrd{4350}{uderzyły}
                        \wrd{}{(na)}
                        \wrd{1565}{ten}
                        \wrd{3614}{dom,}
                        \wrd{2532}{i~}
                        \wrd{4098}{upadł,}
                        \wrd{2532}{a~}
                        \wrd{4431}{upadek}
                        \wrd{846}{jego}
                        \wrd{2258}{był}
                        \wrd{3173}{wielki.}
                \bverse{28}
                        \wrd{2532}{I~}
                        \underline{\wrd{1096}{stało się,}}
                        \wrd{}{(że)}
                        \wrd{3753}{gdy}
                        \wrd{2424}{Jezus}
                        \wrd{4931}{dokończył}
                        \wrd{5128}{tych}
                        \wrd{3056}{słów,}
                        \doubleline{\wrd{1605}{zdumiewały się}}
                        \wrd{3793}{tłumy}
                        \wrd{1909}{nad}
                        \wrd{846}{jego}
                        \wrd{1322}{nauką.}
                \bverse{29}
                        \wrd{2258}{Albowiem}
                        \wrd{1063}{Albowiem}
                        \wrd{1321}{uczył}
                        \wrd{846}{ich,}
                        \wrd{5613}{jak}
                        \wrd{2192}{posiadający}
                        \wrd{1849}{moc,}
                        \wrd{2532}{a~}
                        \wrd{3756}{nie}
                        \wrd{5613}{jak}
                        \underline{\wrd{1122}{znawcy Pisma.}} \section{} \IndTwo
                \bverse{1}
                        \wrd{1161}{A~}
                        \underline{\wrd{2597}{gdy}}
                        \wrd{846}{on}
                        \underline{\wrd{2597}{zszedł}}
                        \wrd{575}{z~}
                        \wrd{3735}{góry,}
                        \doubleline{\wrd{190}{szły za}}
                        \wrd{846}{nim}
                        \wrd{4183}{liczne}
                        \wrd{3793}{tłumy.}
                \bverse{2}
                        \wrd{2532}{A~}
                        \wrd{2400}{oto}
                        \wrd{3015}{trędowaty}
                        \wrd{2064}{podszedł,}
                        \underline{\wrd{4352}{oddał}}
                        \wrd{846}{mu}
                        \underline{\wrd{4352}{pokłon,}}
                        \wrd{}{(i)}
                        \wrd{3004}{powiedział:}
                        \wrd{2962}{Panie,}
                        \wrd{1437}{gdybyś}
                        \wrd{2309}{zechciał,}
                        \doubleline{\wrd{1410}{jesteś w stanie}}
                        \wrd{3165}{mnie}
                        \wrd{2511}{oczyścić.}
                \bverse{3}
                        \wrd{2532}{I~}
                        \wrd{1614}{wyciągnąwszy}
                        \wrd{5495}{rękę,}
                        \wrd{2424}{Jezus}
                        \wrd{680}{dotknął}
                        \wrd{846}{go,}
                        \wrd{3004}{mówiąc:}
                        \wrd{2309}{Chcę,}
                        \underline{\wrd{2511}{zostań oczyszczony.}}
                        \wrd{2532}{I~}
                        \wrd{2112}{natychmiast}
                        \doubleline{\wrd{2511}{został oczyszczony}}
                        \wrd{846}{jego}
                        \wrd{3014}{trąd.}
                \bverse{4}
                        \wrd{2532}{I~}
                        \wrd{3004}{powiedział}
                        \wrd{846}{mu}
                        \wrd{2424}{Jezus:}
                        \wrd{3708}{Uważaj,}
                        \wrd{3367}{nikomu}
                        \wrd{}{(nie)}
                        \wrd{2036}{mów,}
                        \wrd{235}{ale}
                        \wrd{5217}{odejdź,}
                        \wrd{1166}{pokaż}
                        \wrd{4572}{się}
                        \wrd{2409}{kapłanowi}
                        \wrd{2532}{i~}
                        \wrd{4374}{zanieś}
                        \wrd{1435}{dar,}
                        \wrd{3739}{który}
                        \wrd{4367}{nakazał}
                        \wrd{3475}{Mojżesz,}
                        \wrd{1519}{na}
                        \wrd{3142}{świadectwo}
                        \underline{\wrd{846}{dla nich.}}
                \bverse{5}
                        \wrd{1161}{A~}
                        \underline{\wrd{1525}{gdy}}
                        \wrd{846}{on}
                        \underline{\wrd{1525}{wszedł}}
                        \wrd{1519}{do}
                        \wrd{2584}{Kafarnaum,}
                        \wrd{4334}{podszedł}
                        \doubleline{\wrd{846}{do niego}}
                        \wrd{1543}{setnik,}
                        \wrd{3870}{prosząc}
                        \wrd{846}{go}
                \bverse{6}
                        \wrd{2532}{i~}
                        \wrd{3004}{mówiąc:}
                        \wrd{2962}{Panie,}
                        \wrd{3450}{mój}
                        \wrd{3816}{chłopiec}
                        \wrd{906}{leży}
                        \wrd{1722}{w~}
                        \wrd{3614}{domu}
                        \wrd{3885}{sparaliżowany,}
                        \underline{\wrd{928}{poddawany}}
                        \wrd{1171}{strasznej}
                        \underline{\wrd{928}{próbie.}}
                \bverse{7}
                        \wrd{2532}{A~}
                        \wrd{2424}{Jezus}
                        \wrd{3004}{rzekł}
                        \wrd{846}{mu:}
                        \wrd{1473}{Ja}
                        \wrd{2064}{przyjdę}
                        \wrd{}{(i)}
                        \wrd{846}{go}
                        \wrd{2323}{uzdrowię.}
                \bverse{8}
                        \wrd{2532}{A~}
                        \wrd{1543}{setnik,}
                        \wrd{611}{odpowiadając,}
                        \wrd{5346}{powiedział:}
                        \wrd{2962}{Panie,}
                        \wrd{3756}{nie}
                        \wrd{1510}{jestem}
                        \wrd{2425}{wystarczająco}
                        \wrd{}{(godzien),}
                        \wrd{2443}{abyś}
                        \wrd{1525}{wszedł}
                        \wrd{5259}{pod}
                        \wrd{3450}{mój}
                        \wrd{4721}{dach,}
                        \wrd{235}{ale}
                        \wrd{2036}{powiedz}
                        \wrd{3440}{tylko}
                        \wrd{3056}{słowo,}
                        \wrd{2532}{a~}
                        \wrd{3450}{mój}
                        \wrd{3816}{chłopiec}
                        \underline{\wrd{2390}{zostanie uzdrowiony.}}
                \bverse{9}
                        \wrd{1063}{Bowiem}
                        \wrd{2532}{i~}
                        \wrd{1473}{ja}
                        \wrd{}{(choć)}
                        \wrd{1510}{jestem}
                        \wrd{444}{człowiekiem}
                        \wrd{5259}{pod}
                        \wrd{1849}{władzą,}
                        \wrd{2192}{mam}
                        \wrd{5259}{pod}
                        \wrd{1683}{sobą}
                        \wrd{4757}{żołnierzy}
                        \wrd{2532}{i~}
                        \wrd{3004}{mówię}
                        \wrd{5129}{temu:}
                        \wrd{4198}{Ruszaj.}
                        \wrd{2532}{I~}
                        \wrd{4198}{wyrusza.}
                        \wrd{2532}{A~}
                        \wrd{243}{innemu:}
                        \wrd{2064}{Przyjdź.}
                        \wrd{2532}{I~}
                        \wrd{2064}{przychodzi.}
                        \wrd{2532}{A~}
                        \wrd{3450}{mojemu}
                        \wrd{1401}{niewolnikowi:}
                        \wrd{4160}{Uczyń}
                        \wrd{5124}{to.}
                        \wrd{2532}{I~}
                        \wrd{4160}{czyni.}
                \bverse{10}
                        \wrd{1161}{A~}
                        \wrd{2424}{Jezus,}
                        \underline{\wrd{191}{gdy}}
                        \wrd{}{(to)}
                        \underline{\wrd{191}{usłyszał,}}
                        \doubleline{\wrd{2296}{zdziwił się}}
                        \wrd{2532}{i~}
                        \wrd{2036}{powiedział}
                        \underline{\wrd{190}{idącym za}}
                        \wrd{}{(nim):}
                        \wrd{281}{Zaprawdę}
                        \wrd{3004}{mówię}
                        \wrd{5213}{wam,}
                        \doubleline{\wrd{3761}{że nie}}
                        \wrd{2147}{znalazłem}
                        \wrd{1722}{w~}
                        \wrd{2474}{Izraelu}
                        \underline{\wrd{5118}{tak wielkiej}}
                        \wrd{4102}{wiary.}
                \bverse{11}
                        \wrd{1161}{I~}
                        \wrd{3004}{mówię}
                        \wrd{5213}{wam,}
                        \wrd{3754}{że}
                        \wrd{2240}{przyjdą}
                        \wrd{4183}{liczni}
                        \wrd{575}{ze}
                        \wrd{395}{wschodu}
                        \wrd{2532}{i~}
                        \wrd{1424}{zachodu,}
                        \wrd{2532}{i~}
                        \underline{\wrd{347}{położą się}}
                        \wrd{}{(przy stole)}
                        \wrd{1722}{w~}
                        \wrd{932}{królestwie}
                        \wrd{3772}{niebios}
                        \wrd{3326}{z~}
                        \wrd{11}{Abrahamem,}
                        \wrd{2532}{i~}
                        \wrd{2464}{Izaakiem,}
                        \wrd{2532}{i~}
                        \wrd{2384}{Jakubem.}
                \bverse{12}
                        \wrd{1161}{A~}
                        \wrd{5207}{synowie}
                        \wrd{932}{królestwa}
                        \underline{\wrd{1544}{zostaną wyrzuceni}}
                        \wrd{1519}{w~}
                        \wrd{1857}{zewnętrzną}
                        \wrd{4655}{ciemność.}
                        \wrd{1563}{Tam}
                        \wrd{2071}{będzie}
                        \wrd{2805}{płacz}
                        \wrd{2532}{i~}
                        \wrd{1030}{zgrzytanie}
                        \wrd{3599}{zębów.}
                \bverse{13}
                        \wrd{2532}{I~}
                        \wrd{2036}{powiedział}
                        \wrd{2424}{Jezus}
                        \wrd{1543}{setnikowi:}
                        \wrd{5217}{Odejdź,}
                        \wrd{2532}{a~}
                        \wrd{5613}{jak}
                        \wrd{4100}{uwierzyłeś,}
                        \wrd{}{(tak)}
                        \underline{\wrd{1096}{niech}}
                        \wrd{4671}{ci}
                        \underline{\wrd{1096}{się stanie.}}
                        \wrd{2532}{I~}
                        \wrd{1722}{w~}
                        \wrd{1565}{tej}
                        \wrd{5610}{chwili}
                        \doubleline{\wrd{2390}{został uzdrowiony}}
                        \wrd{846}{jego}
                        \wrd{3816}{chłopiec.}
                \bverse{14}
                        \wrd{2532}{A~}
                        \wrd{2424}{Jezus,}
                        \underline{\wrd{2064}{gdy przybył}}
                        \wrd{1519}{do}
                        \wrd{3614}{domu}
                        \wrd{4074}{Piotra,}
                        \wrd{1492}{zobaczył}
                        \wrd{846}{jego}
                        \wrd{3994}{teściową,}
                        \wrd{906}{leżącą}
                        \wrd{2532}{i~}
                        \wrd{4445}{gorączkującą.}
                \bverse{15}
                        \wrd{2532}{I~}
                        \wrd{680}{dotknął}
                        \wrd{846}{jej}
                        \wrd{5495}{rękę,}
                        \wrd{2532}{i~}
                        \wrd{863}{opuściła}
                        \wrd{846}{ją}
                        \wrd{4446}{gorączka.}
                        \wrd{2532}{I~}
                        \wrd{1453}{wstała,}
                        \wrd{2532}{i~}
                        \wrd{1247}{służyła}
                        \wrd{846}{im.}
                \bverse{16}
                        \wrd{1161}{A~}
                        \wrd{}{(gdy)}
                        \wrd{1096}{nastał}
                        \wrd{3798}{wieczór,}
                        \wrd{4374}{przyprowadzali}
                        \wrd{846}{mu}
                        \wrd{4183}{licznych}
                        \wrd{1139}{opętanych.}
                        \wrd{2532}{I~}
                        \wrd{3056}{słowem}
                        \wrd{1544}{wyrzucił}
                        \wrd{4151}{duchy,}
                        \wrd{2532}{oraz}
                        \wrd{2323}{uzdrowił}
                        \wrd{3956}{wszystkich,}
                        \underline{\wrd{2192}{którzy się}}
                        \wrd{2560}{źle}
                        \underline{\wrd{2192}{mieli,}}
                \bverse{17}
                        \wrd{3704}{aby}
                        \underline{\wrd{4137}{wypełniło się}}
                        \doubleline{\wrd{3588}{to, co}}
                        \underline{\wrd{4483}{zostało powiedziane}}
                        \wrd{1223}{przez}
                        \wrd{4396}{proroka}
                        \wrd{2268}{Izajasza,}
                        \doubleline{\wrd{3004}{który mówił:}}
                        \wrd{846}{On}
                        \wrd{2983}{wziął}
                        \wrd{2257}{nasze}
                        \wrd{769}{słabości}
                        \wrd{2532}{oraz}
                        \wrd{3554}{choroby}
                        \wrd{941}{zabrał.}
                \bverse{18}
                        \wrd{1161}{A~}
                        \wrd{2424}{Jezus,}
                        \underline{\wrd{1492}{gdy zobaczył}}
                        \wrd{4183}{liczny}
                        \wrd{3793}{tłum}
                        \wrd{4012}{wokół}
                        \wrd{846}{siebie,}
                        \wrd{2753}{kazał}
                        \wrd{565}{odpłynąć}
                        \wrd{1519}{na}
                        \doubleline{\wrd{4008}{drugą stronę.}}
                \bverse{19}
                        \wrd{2532}{A~}
                        \wrd{1520}{jeden}
                        \wrd{}{(ze)}
                        \underline{\wrd{1122}{znawców Pisma}}
                        \wrd{4334}{podszedł}
                        \wrd{}{(i)}
                        \wrd{2036}{powiedział}
                        \wrd{846}{mu:}
                        \wrd{1320}{Nauczycielu,}
                        \doubleline{\wrd{190}{pójdę za}}
                        \wrd{4671}{tobą,}
                        \wrd{3699}{gdziekolwiek}
                        \wrd{1437}{gdziekolwiek}
                        \wrd{}{(ty)}
                        \wrd{565}{pójdziesz.}
                \bverse{20}
                        \wrd{2532}{I~}
                        \wrd{3004}{rzekł}
                        \wrd{846}{mu}
                        \wrd{2424}{Jezus:}
                        \wrd{258}{Lisy}
                        \wrd{2192}{mają}
                        \wrd{5454}{nory,}
                        \wrd{2532}{a~}
                        \wrd{4071}{ptaki}
                        \wrd{3772}{niebiańskie–}
                        \wrd{2682}{gniazda,}
                        \wrd{1161}{lecz}
                        \wrd{5207}{Syn}
                        \wrd{444}{Człowieczy}
                        \wrd{3756}{nie}
                        \wrd{2192}{ma}
                        \wrd{4226}{gdzie}
                        \wrd{2776}{głowy}
                        \wrd{2827}{skłonić.}
                \bverse{21}
                        \wrd{1161}{A~}
                        \wrd{846}{Jego}
                        \wrd{2087}{drugi}
                        \wrd{3101}{uczeń}
                        \wrd{2036}{powiedział}
                        \wrd{846}{mu:}
                        \wrd{2962}{Panie,}
                        \wrd{2010}{pozwól}
                        \wrd{3427}{mi}
                        \wrd{4412}{najpierw}
                        \wrd{565}{odejść}
                        \wrd{2532}{i~}
                        \wrd{2290}{pogrzebać}
                        \wrd{3450}{mego}
                        \wrd{3962}{ojca.}
                \bverse{22}
                        \wrd{1161}{A~}
                        \wrd{2424}{Jezus}
                        \wrd{2036}{powiedział}
                        \wrd{846}{mu:}
                        \underline{\wrd{190}{Idź za}}
                        \wrd{3427}{mną}
                        \wrd{2532}{i~}
                        \wrd{863}{zostaw}
                        \wrd{3498}{zmarłym}
                        \wrd{2290}{grzebanie}
                        \wrd{1438}{ich}
                        \wrd{3498}{zmarłych.}
                \bverse{23}
                        \wrd{2532}{A~}
                        \underline{\wrd{1684}{kiedy}}
                        \wrd{846}{on}
                        \underline{\wrd{1684}{wszedł}}
                        \wrd{1519}{do}
                        \wrd{4143}{łodzi,}
                        \wrd{190}{towarzyszyli}
                        \wrd{846}{mu}
                        \wrd{846}{jego}
                        \wrd{3101}{uczniowie.}
                \bverse{24}
                        \wrd{2532}{I~}
                        \wrd{2400}{oto}
                        \wrd{3173}{wielki}
                        \wrd{4578}{sztorm}
                        \wrd{1096}{powstał}
                        \wrd{1722}{na}
                        \wrd{2281}{morzu,}
                        \underline{\wrd{5620}{tak że}}
                        \wrd{4143}{łódź}
                        \doubleline{\wrd{2572}{była przykrywana}}
                        \wrd{5259}{przez}
                        \wrd{2949}{fale.}
                        \wrd{846}{On}
                        \wrd{1161}{zaś}
                        \wrd{2518}{spał.}
                \bverse{25}
                        \wrd{2532}{A~}
                        \wrd{3101}{uczniowie}
                        \wrd{4334}{podszedłszy,}
                        \wrd{1453}{obudzili}
                        \wrd{846}{go,}
                        \wrd{3004}{mówiąc:}
                        \wrd{2962}{Panie!}
                        \wrd{4982}{Ratuj}
                        \wrd{2248}{nas!}
                        \wrd{622}{Giniemy!}
                \bverse{26}
                        \wrd{2532}{I~}
                        \wrd{3004}{powiedział}
                        \wrd{846}{im:}
                        \wrd{5101}{Dlaczego}
                        \wrd{2075}{jesteście}
                        \wrd{1169}{lękliwi,}
                        \underline{\wrd{3640}{małej wiary?}}
                        \wrd{5119}{Wtedy}
                        \doubleline{\wrd{1453}{podniósł się,}}
                        \wrd{2008}{zgromił}
                        \wrd{417}{wiatry}
                        \wrd{2532}{i~}
                        \wrd{2281}{morze,}
                        \wrd{2532}{i~}
                        \wrd{1096}{nastała}
                        \wrd{3173}{wielka}
                        \wrd{1055}{cisza.}
                \bverse{27}
                        \wrd{1161}{A~}
                        \wrd{444}{ludzie}
                        \underline{\wrd{2296}{zdziwili się,}}
                        \wrd{3004}{mówiąc:}
                        \wrd{4217}{Kim}
                        \wrd{3778}{on}
                        \wrd{2076}{jest,}
                        \wrd{3754}{że}
                        \wrd{2532}{i~}
                        \wrd{417}{wiatry,}
                        \wrd{2532}{i~}
                        \wrd{2281}{morze}
                        \doubleline{\wrd{5219}{są}}
                        \wrd{846}{mu}
                        \doubleline{\wrd{5219}{posłuszne?}}
                \bverse{28}
                        \wrd{2532}{A~}
                        \wrd{846}{on,}
                        \underline{\wrd{2064}{gdy przybył}}
                        \wrd{1519}{na}
                        \doubleline{\wrd{4008}{drugą stronę–}}
                        \wrd{1519}{na}
                        \wrd{5561}{terytorium}
                        \wrd{1086}{Gergezeńczyków–}
                        \underline{\wrd{5221}{wyszli}}
                        \wrd{846}{mu}
                        \underline{\wrd{5221}{na spotkanie}}
                        \wrd{1417}{dwaj}
                        \wrd{1139}{opętani,}
                        \wrd{}{(którzy)}
                        \wrd{1831}{wyszli}
                        \wrd{1537}{z~}
                        \wrd{3419}{grobów–}
                        \wrd{3029}{bardzo}
                        \wrd{5467}{uciążliwi,}
                        \doubleline{\wrd{5620}{tak że}}
                        \wrd{5100}{nikt}
                        \wrd{3361}{nie}
                        \underline{\wrd{2480}{był w stanie}}
                        \wrd{3928}{przejść}
                        \wrd{1223}{przez}
                        \wrd{1565}{tamtą}
                        \wrd{3598}{drogę.}
                \bverse{29}
                        \wrd{2532}{I~}
                        \wrd{2400}{oto}
                        \wrd{2896}{krzyczeli,}
                        \wrd{3004}{mówiąc:}
                        \wrd{5101}{Co}
                        \wrd{2254}{nam}
                        \wrd{2532}{do}
                        \wrd{4671}{ciebie,}
                        \wrd{2424}{Jezusie,}
                        \wrd{5207}{Synu}
                        \wrd{2316}{Boga?}
                        \wrd{2064}{Przyszedłeś}
                        \wrd{5602}{tu}
                        \wrd{4253}{przed}
                        \wrd{2540}{czasem}
                        \underline{\wrd{928}{poddawać}}
                        \wrd{2248}{nas}
                        \underline{\wrd{928}{próbie?}}
                \bverse{30}
                        \wrd{1161}{A~}
                        \wrd{3112}{daleko}
                        \wrd{575}{od}
                        \wrd{846}{nich}
                        \wrd{2258}{była}
                        \wrd{34}{trzoda}
                        \wrd{4183}{licznych,}
                        \underline{\wrd{1006}{pasących się}}
                        \wrd{5519}{świń.}
                \bverse{31}
                        \wrd{1161}{A~}
                        \wrd{1142}{demony}
                        \wrd{3870}{prosiły}
                        \wrd{846}{go,}
                        \wrd{3004}{mówiąc:}
                        \wrd{1487}{Jeśli}
                        \wrd{2248}{nas}
                        \wrd{1544}{wyrzucasz,}
                        \wrd{2010}{pozwól}
                        \wrd{2254}{nam}
                        \wrd{565}{odejść}
                        \wrd{1519}{w~}
                        \wrd{34}{trzodę}
                        \wrd{5519}{świń.}
                \bverse{32}
                        \wrd{2532}{I~}
                        \wrd{2036}{powiedział}
                        \wrd{846}{im:}
                        \wrd{5217}{Odejdźcie.}
                        \wrd{1161}{A~}
                        \underline{\wrd{1831}{gdy wyszły,}}
                        \wrd{565}{odeszły}
                        \wrd{1519}{w~}
                        \wrd{34}{trzodę}
                        \wrd{5519}{świń.}
                        \wrd{2532}{I~}
                        \wrd{2400}{oto}
                        \wrd{3956}{cała}
                        \wrd{34}{trzoda}
                        \wrd{5519}{świń}
                        \wrd{3729}{ruszyła}
                        \wrd{2596}{z~}
                        \wrd{2911}{urwiska}
                        \wrd{1519}{w~}
                        \wrd{2281}{morze.}
                        \wrd{2532}{I~}
                        \wrd{599}{zginęły}
                        \wrd{1722}{w~}
                        \wrd{5204}{wodach.}
                \bverse{33}
                        \wrd{1161}{A~}
                        \wrd{1006}{pasterze}
                        \wrd{5343}{uciekli}
                        \wrd{2532}{i~}
                        \underline{\wrd{565}{gdy odeszli}}
                        \wrd{1519}{do}
                        \wrd{4172}{miasta,}
                        \wrd{518}{opowiedzieli}
                        \wrd{3956}{wszystko,}
                        \wrd{2532}{także}
                        \wrd{3588}{o~}
                        \wrd{1139}{opętanych.}
                \bverse{34}
                        \wrd{2532}{I~}
                        \wrd{2400}{oto}
                        \wrd{3956}{całe}
                        \wrd{4172}{miasto}
                        \wrd{1831}{wyszło}
                        \wrd{2424}{Jezusowi}
                        \wrd{1519}{na}
                        \wrd{4877}{spotkanie}
                        \wrd{2532}{i~}
                        \underline{\wrd{1492}{gdy}}
                        \wrd{846}{go}
                        \underline{\wrd{1492}{zobaczyli,}}
                        \wrd{3870}{prosili,}
                        \wrd{3704}{żeby}
                        \wrd{3327}{odszedł}
                        \wrd{}{(z dala)}
                        \wrd{575}{od}
                        \wrd{846}{ich}
                        \wrd{3725}{granic.} \section{} \IndTwo
                \bverse{1}
                        \wrd{2532}{A~}
                        \underline{\wrd{1684}{gdy wszedł}}
                        \wrd{1519}{do}
                        \wrd{4143}{łodzi,}
                        \doubleline{\wrd{1276}{przeprawił się}}
                        \wrd{2532}{i~}
                        \wrd{2064}{przybył}
                        \wrd{1519}{do}
                        \wrd{2398}{swojego}
                        \wrd{4172}{miasta.}
                \bverse{2}
                        \wrd{2532}{I~}
                        \wrd{2400}{oto}
                        \wrd{4374}{przynieśli}
                        \wrd{846}{mu}
                        \wrd{3885}{sparaliżowanego,}
                        \wrd{906}{leżącego}
                        \wrd{1909}{na}
                        \wrd{2825}{łożu.}
                        \wrd{2532}{A~}
                        \wrd{2424}{Jezus,}
                        \underline{\wrd{1492}{gdy zobaczył}}
                        \wrd{846}{ich}
                        \wrd{4102}{wiarę,}
                        \wrd{2036}{powiedział}
                        \wrd{3885}{sparaliżowanemu:}
                        \wrd{2293}{Odwagi,}
                        \wrd{5043}{dziecko,}
                        \doubleline{\wrd{863}{odpuszczone są}}
                        \wrd{4671}{tobie}
                        \wrd{266}{grzechy}
                        \wrd{4675}{twoje.}
                \bverse{3}
                        \wrd{2532}{A~}
                        \wrd{2400}{oto}
                        \wrd{5100}{niektórzy}
                        \underline{\wrd{1122}{ze znawców Pisma}}
                        \wrd{2036}{pomyśleli}
                        \wrd{1722}{w~}
                        \wrd{1438}{sobie:}
                        \wrd{3778}{On}
                        \wrd{987}{znieważa.}
                \bverse{4}
                        \wrd{2532}{A~}
                        \wrd{2424}{Jezus,}
                        \underline{\wrd{1492}{gdy zobaczył}}
                        \wrd{846}{ich}
                        \wrd{1761}{myśli,}
                        \wrd{2036}{powiedział:}
                        \wrd{2443}{Dlaczego}
                        \wrd{2444}{Dlaczego}
                        \wrd{5210}{wymyślacie}
                        \wrd{1760}{wymyślacie}
                        \wrd{4190}{zło}
                        \wrd{1722}{w~}
                        \wrd{5216}{waszych}
                        \wrd{2588}{sercach?}
                \bverse{5}
                        \wrd{5101}{Co}
                        \wrd{1063}{bowiem}
                        \wrd{2076}{jest}
                        \wrd{2123}{łatwiejsze,}
                        \wrd{2036}{powiedzieć:}
                        \underline{\wrd{863}{Odpuszczone są}}
                        \wrd{4675}{twoje}
                        \wrd{266}{grzechy,}
                        \wrd{2228}{czy}
                        \wrd{2036}{powiedzieć:}
                        \wrd{1453}{Powstań}
                        \wrd{2532}{i~}
                        \wrd{4043}{chodź?}
                \bverse{6}
                        \wrd{1161}{Ale}
                        \wrd{2443}{żebyście}
                        \wrd{1492}{wiedzieli,}
                        \wrd{3754}{że}
                        \wrd{5207}{Syn}
                        \wrd{444}{Człowieczy}
                        \wrd{2192}{ma}
                        \wrd{1909}{na}
                        \wrd{1093}{ziemi}
                        \wrd{1849}{moc}
                        \wrd{863}{odpuszczania}
                        \wrd{266}{grzechów–}
                        \wrd{5119}{wtedy}
                        \wrd{3004}{powiedział}
                        \wrd{3885}{sparaliżowanemu–}
                        \wrd{1453}{Powstań,}
                        \wrd{142}{zabierz}
                        \wrd{4675}{swe}
                        \wrd{2825}{łoże}
                        \wrd{2532}{i~}
                        \wrd{5217}{odejdź}
                        \wrd{1519}{do}
                        \wrd{4675}{swojego}
                        \wrd{3624}{domu.}
                \bverse{7}
                        \wrd{2532}{A~}
                        \underline{\wrd{1453}{gdy wstał,}}
                        \wrd{565}{poszedł}
                        \wrd{1519}{do}
                        \wrd{846}{swego}
                        \wrd{3624}{domu.}
                \bverse{8}
                        \wrd{1161}{A~}
                        \wrd{3793}{tłumy,}
                        \underline{\wrd{1492}{gdy}}
                        \wrd{}{(to)}
                        \underline{\wrd{1492}{zobaczyły,}}
                        \doubleline{\wrd{2296}{dziwiły się}}
                        \wrd{2532}{i~}
                        \underline{\wrd{1392}{oddały chwałę}}
                        \wrd{2316}{Bogu,}
                        \doubleline{\wrd{1325}{że dał}}
                        \wrd{444}{ludziom}
                        \wrd{5108}{taką}
                        \wrd{1849}{moc.}
                \bverse{9}
                        \wrd{2532}{A~}
                        \wrd{3855}{odchodząc}
                        \wrd{1564}{stamtąd,}
                        \wrd{2424}{Jezus}
                        \wrd{1492}{dostrzegł}
                        \wrd{444}{człowieka,}
                        \wrd{2521}{siedzącego}
                        \wrd{1909}{w~}
                        \underline{\wrd{5058}{miejscu pobierania podatków,}}
                        \doubleline{\wrd{3004}{którego nazywano}}
                        \wrd{3156}{Mateuszem,}
                        \wrd{2532}{i~}
                        \wrd{3004}{powiedział}
                        \wrd{846}{mu:}
                        \underline{\wrd{190}{Idź za}}
                        \wrd{3427}{mną.}
                        \wrd{2532}{A~}
                        \doubleline{\wrd{450}{gdy wstał,}}
                        \underline{\wrd{190}{zaczął iść za}}
                        \wrd{846}{nim.}
                \bverse{10}
                        \wrd{2532}{A~}
                        \underline{\wrd{846}{gdy on}}
                        \doubleline{\wrd{345}{leżał przy stole}}
                        \wrd{1722}{w~}
                        \wrd{3614}{domu,}
                        \underline{\wrd{1096}{stało się,}}
                        \wrd{2532}{a~}
                        \wrd{2400}{oto}
                        \wrd{4183}{liczni}
                        \doubleline{\wrd{5057}{poborcy podatków}}
                        \wrd{2532}{i~}
                        \wrd{268}{grzesznicy,}
                        \wrd{2064}{przyszedłszy,}
                        \underline{\wrd{4873}{leżeli przy stole}}
                        \doubleline{\wrd{2424}{z Jezusem}}
                        \wrd{2532}{oraz}
                        \wrd{846}{jego}
                        \wrd{3101}{uczniami.}
                \bverse{11}
                        \wrd{2532}{A~}
                        \underline{\wrd{1492}{gdy zobaczyli}}
                        \wrd{}{(to)}
                        \wrd{5330}{faryzeusze,}
                        \wrd{2036}{mówili}
                        \wrd{846}{jego}
                        \wrd{3101}{uczniom:}
                        \wrd{1223}{Dlaczego}
                        \wrd{5101}{Dlaczego}
                        \wrd{5216}{wasz}
                        \wrd{1320}{nauczyciel}
                        \wrd{2068}{je}
                        \wrd{3326}{z~}
                        \doubleline{\wrd{5057}{poborcami podatków}}
                        \wrd{2532}{i~}
                        \wrd{268}{grzesznikami?}
                \bverse{12}
                        \wrd{1161}{A~}
                        \wrd{2424}{Jezus,}
                        \underline{\wrd{191}{gdy}}
                        \wrd{}{(to)}
                        \underline{\wrd{191}{usłyszał,}}
                        \wrd{2036}{powiedział}
                        \wrd{846}{im:}
                        \wrd{3756}{Nie}
                        \wrd{2480}{zdrowi}
                        \wrd{5532}{potrzebują}
                        \wrd{2192}{potrzebują}
                        \wrd{2395}{lekarza,}
                        \wrd{235}{ale}
                        \doubleline{\wrd{2192}{ci, co się}}
                        \wrd{2560}{źle}
                        \doubleline{\wrd{2192}{mają.}}
                \bverse{13}
                        \wrd{4198}{Idźcie}
                        \wrd{1161}{i~}
                        \underline{\wrd{3129}{nauczcie się,}}
                        \wrd{5101}{co}
                        \wrd{}{(to)}
                        \wrd{2076}{znaczy:}
                        \wrd{1656}{Miłosierdzia}
                        \wrd{2309}{chcę,}
                        \wrd{2532}{a~}
                        \wrd{3756}{nie}
                        \wrd{2378}{ofiary.}
                        \wrd{3756}{Nie}
                        \wrd{2064}{przyszedłem}
                        \wrd{1063}{bowiem}
                        \wrd{2564}{powołać}
                        \wrd{1519}{do}
                        \doubleline{\wrd{3341}{zmiany myślenia}}
                        \wrd{1342}{sprawiedliwych,}
                        \wrd{235}{ale}
                        \wrd{268}{grzeszników.}
                \bverse{14}
                        \wrd{5119}{Wtedy}
                        \underline{\wrd{4334}{przyszli do}}
                        \wrd{846}{niego}
                        \wrd{3101}{uczniowie}
                        \wrd{2491}{Jana}
                        \wrd{}{(i)}
                        \wrd{3004}{mówili:}
                        \wrd{1223}{Dlaczego}
                        \wrd{5101}{Dlaczego}
                        \wrd{2249}{my}
                        \wrd{2532}{i~}
                        \wrd{5330}{faryzeusze}
                        \wrd{4183}{dużo}
                        \wrd{3522}{pościmy,}
                        \wrd{1161}{a~}
                        \wrd{4675}{twoi}
                        \wrd{3101}{uczniowie}
                        \wrd{3756}{nie}
                        \wrd{3522}{poszczą?}
                \bverse{15}
                        \wrd{2532}{A~}
                        \wrd{2424}{Jezus}
                        \wrd{2036}{powiedział}
                        \wrd{846}{im:}
                        \wrd{3361}{Nie}
                        \underline{\wrd{1410}{są w stanie}}
                        \wrd{5207}{synowie}
                        \doubleline{\wrd{3567}{domu weselnego}}
                        \underline{\wrd{3996}{być w żałobie,}}
                        \wrd{1909}{jak}
                        \wrd{3745}{długo}
                        \wrd{2076}{jest}
                        \wrd{3326}{z~}
                        \wrd{846}{nimi}
                        \doubleline{\wrd{3566}{pan młody.}}
                        \wrd{1161}{Ale}
                        \wrd{2064}{przyjdą}
                        \wrd{2250}{dni,}
                        \wrd{3752}{kiedy}
                        \underline{\wrd{522}{zostanie odebrany}}
                        \wrd{575}{im}
                        \wrd{846}{im}
                        \doubleline{\wrd{3566}{pan młody}}
                        \wrd{2532}{i~}
                        \wrd{5119}{wtedy}
                        \underline{\wrd{3522}{będą pościć.}}
                \bverse{16}
                        \wrd{1161}{Przecież}
                        \wrd{3762}{nikt}
                        \wrd{}{(nie)}
                        \wrd{1911}{nakłada}
                        \wrd{1915}{naszywki}
                        \wrd{}{(z)}
                        \underline{\wrd{4470}{kawałka}}
                        \wrd{46}{surowej}
                        \underline{\wrd{4470}{tkaniny}}
                        \wrd{1909}{na}
                        \wrd{3820}{starą}
                        \doubleline{\wrd{2440}{szatę wierzchnią,}}
                        \wrd{1063}{bowiem}
                        \wrd{}{(to)}
                        \wrd{846}{jej}
                        \wrd{4138}{uzupełnienie}
                        \underline{\wrd{142}{zrywa się}}
                        \wrd{575}{z~}
                        \doubleline{\wrd{2440}{szaty wierzchniej,}}
                        \wrd{2532}{a~}
                        \wrd{4978}{rozdarcie}
                        \underline{\wrd{1096}{staje się}}
                        \wrd{5501}{gorsze.}
                \bverse{17}
                        \underline{\wrd{3761}{Ani nie}}
                        \doubleline{\wrd{906}{wlewa się}}
                        \wrd{3501}{młodego}
                        \wrd{3631}{wina}
                        \wrd{1519}{w~}
                        \wrd{3820}{stare}
                        \underline{\wrd{779}{skórzane bukłaki.}}
                        \wrd{1487}{Bo}
                        \wrd{1161}{inaczej}
                        \wrd{3361}{inaczej}
                        \doubleline{\wrd{779}{skórzane bukłaki}}
                        \underline{\wrd{4486}{się roztrzaskują}}
                        \wrd{2532}{i~}
                        \wrd{3631}{wino}
                        \doubleline{\wrd{1632}{się wylewa,}}
                        \wrd{2532}{a~}
                        \underline{\wrd{779}{skórzane bukłaki}}
                        \doubleline{\wrd{622}{się niszczą.}}
                        \wrd{235}{Ale}
                        \underline{\wrd{906}{wlewa się}}
                        \wrd{3501}{młode}
                        \wrd{3631}{wino}
                        \wrd{1519}{w~}
                        \wrd{2537}{nowe}
                        \doubleline{\wrd{779}{skórzane bukłaki,}}
                        \wrd{2532}{a~}
                        \wrd{297}{obydwa}
                        \underline{\wrd{4933}{się zachowują.}}
                \bverse{18}
                        \wrd{}{(Gdy)}
                        \wrd{846}{on}
                        \wrd{846}{im}
                        \wrd{5023}{to}
                        \wrd{2980}{mówił,}
                        \wrd{2400}{oto}
                        \wrd{1520}{jeden}
                        \wrd{}{(z)}
                        \underline{\wrd{758}{posiadających władzę}}
                        \wrd{2064}{przyszedł}
                        \wrd{}{(i)}
                        \doubleline{\wrd{4352}{oddał}}
                        \wrd{846}{mu}
                        \doubleline{\wrd{4352}{pokłon,}}
                        \wrd{3004}{mówiąc,}
                        \wrd{3754}{że:}
                        \wrd{3450}{Moja}
                        \wrd{2364}{córka}
                        \underline{\wrd{737}{przed chwilą}}
                        \wrd{5053}{umarła,}
                        \wrd{235}{ale}
                        \wrd{2064}{przyjdź,}
                        \wrd{2007}{połóż}
                        \wrd{1909}{na}
                        \wrd{846}{nią}
                        \wrd{4675}{swą}
                        \wrd{5495}{rękę,}
                        \wrd{2532}{a~}
                        \doubleline{\wrd{2198}{będzie żyła.}}
                \bverse{19}
                        \wrd{2532}{A~}
                        \wrd{2424}{Jezus}
                        \wrd{1453}{powstawszy,}
                        \underline{\wrd{190}{poszedł za}}
                        \wrd{846}{nim,}
                        \wrd{2532}{a~}
                        \wrd{}{(także)}
                        \wrd{3101}{uczniowie}
                        \wrd{846}{jego.}
                \bverse{20}
                        \wrd{2532}{A~}
                        \wrd{2400}{oto}
                        \wrd{1135}{kobieta,}
                        \wrd{131}{krwawiąca}
                        \underline{\wrd{1427}{od dwunastu}}
                        \wrd{2094}{lat,}
                        \doubleline{\wrd{4334}{gdy podeszła}}
                        \underline{\wrd{3693}{z tyłu,}}
                        \wrd{680}{dotknęła}
                        \wrd{2899}{krawędzi}
                        \wrd{846}{jego}
                        \doubleline{\wrd{2440}{wierzchniej szaty.}}
                \bverse{21}
                        \wrd{3004}{Myślała}
                        \wrd{1063}{bowiem}
                        \wrd{1722}{sobie:}
                        \wrd{1438}{sobie:}
                        \wrd{1437}{Gdybym}
                        \wrd{3440}{tylko}
                        \wrd{680}{dotknęła}
                        \wrd{846}{jego}
                        \underline{\wrd{2440}{wierzchniej szaty,}}
                        \doubleline{\wrd{4982}{będę uratowana.}}
                \bverse{22}
                        \wrd{1161}{A~}
                        \wrd{2424}{Jezus,}
                        \underline{\wrd{1994}{gdy się obrócił}}
                        \wrd{2532}{i~}
                        \wrd{846}{ją}
                        \wrd{1492}{dostrzegł,}
                        \wrd{2036}{powiedział:}
                        \wrd{2293}{Odwagi,}
                        \wrd{2364}{córko,}
                        \wrd{4675}{twoja}
                        \wrd{4102}{wiara}
                        \wrd{4571}{cię}
                        \wrd{4982}{uratowała.}
                        \wrd{2532}{I~}
                        \wrd{575}{od}
                        \wrd{1565}{tej}
                        \wrd{5610}{chwili}
                        \wrd{1135}{kobieta}
                        \doubleline{\wrd{4982}{była uratowana.}}
                \bverse{23}
                        \wrd{2532}{A~}
                        \wrd{2424}{Jezus,}
                        \underline{\wrd{2064}{gdy przyszedł}}
                        \wrd{1519}{do}
                        \wrd{3614}{domu}
                        \doubleline{\wrd{758}{posiadającego władzę}}
                        \wrd{2532}{i~}
                        \wrd{1492}{dostrzegł}
                        \underline{\wrd{834}{grających na aulosie}}
                        \wrd{2532}{oraz}
                        \wrd{3793}{tłum}
                        \doubleline{\wrd{2350}{czyniący wrzawę,}}
                \bverse{24}
                        \wrd{3004}{Powiedział}
                        \wrd{846}{im:}
                        \wrd{402}{Odejdźcie,}
                        \wrd{1063}{bowiem}
                        \wrd{2877}{dziewczynka}
                        \wrd{3756}{nie}
                        \wrd{599}{umarła,}
                        \wrd{235}{ale}
                        \wrd{2518}{śpi.}
                        \wrd{2532}{I~}
                        \wrd{2606}{drwili}
                        \wrd{}{(z)}
                        \wrd{846}{niego.}
                \bverse{25}
                        \wrd{3753}{A~}
                        \wrd{1161}{gdy}
                        \wrd{3793}{tłum}
                        \underline{\wrd{1544}{został wyrzucony,}}
                        \wrd{}{(Jezus)}
                        \wrd{1525}{wszedł,}
                        \wrd{2902}{chwycił}
                        \wrd{846}{jej}
                        \wrd{5495}{rękę,}
                        \wrd{2532}{a~}
                        \wrd{2877}{dziewczynka}
                        \wrd{1453}{wstała.}
                \bverse{26}
                        \wrd{2532}{I~}
                        \underline{\wrd{1831}{rozeszła się}}
                        \wrd{3778}{ta}
                        \wrd{5345}{wieść}
                        \wrd{1519}{po}
                        \wrd{3650}{całej}
                        \wrd{1565}{tamtejszej}
                        \wrd{1093}{ziemi.}
                \bverse{27}
                        \wrd{2532}{A~}
                        \wrd{3855}{odchodzącemu}
                        \wrd{1564}{stamtąd}
                        \wrd{2424}{Jezusowi}
                        \underline{\wrd{190}{zaczęli towarzyszyć}}
                        \underline{\wrd{846}{zaczęli towarzyszyć}}
                        \wrd{1417}{dwaj}
                        \wrd{5185}{ślepi,}
                        \doubleline{\wrd{2896}{którzy wołali}}
                        \wrd{2532}{i~}
                        \wrd{3004}{mówili:}
                        \underline{\wrd{1653}{Zlituj się}}
                        \doubleline{\wrd{2248}{nad nami,}}
                        \wrd{5207}{Synu}
                        \wrd{1138}{Dawida.}
                \bverse{28}
                        \wrd{1161}{A~}
                        \underline{\wrd{2064}{gdy przyszedł}}
                        \wrd{1519}{do}
                        \wrd{3614}{domu,}
                        \doubleline{\wrd{4334}{podeszli do}}
                        \wrd{846}{niego}
                        \wrd{5185}{ślepi,}
                        \wrd{2532}{a~}
                        \wrd{2424}{Jezus}
                        \wrd{846}{im}
                        \wrd{3004}{powiedział:}
                        \wrd{4100}{Wierzycie,}
                        \wrd{3754}{że}
                        \underline{\wrd{1410}{jestem w stanie}}
                        \wrd{5124}{to}
                        \wrd{4160}{uczynić?}
                        \wrd{3004}{Odpowiedzieli}
                        \wrd{846}{mu:}
                        \wrd{3483}{Tak,}
                        \wrd{2962}{Panie!}
                \bverse{29}
                        \wrd{5119}{Wtedy}
                        \wrd{680}{dotknął}
                        \wrd{846}{ich}
                        \wrd{3788}{oczy,}
                        \wrd{3004}{mówiąc:}
                        \underline{\wrd{2596}{Zgodnie z}}
                        \wrd{5216}{waszą}
                        \wrd{4102}{wiarą}
                        \doubleline{\wrd{1096}{niech się}}
                        \wrd{5213}{wam}
                        \doubleline{\wrd{1096}{stanie}}
                \bverse{30}
                        \wrd{2532}{I~}
                        \underline{\wrd{455}{otworzyły się}}
                        \wrd{846}{ich}
                        \wrd{3788}{oczy.}
                        \wrd{2532}{A~}
                        \wrd{2424}{Jezus}
                        \wrd{846}{ich}
                        \doubleline{\wrd{1690}{ostro upomniał,}}
                        \wrd{3004}{mówiąc:}
                        \wrd{3708}{Uważajcie,}
                        \wrd{}{(aby)}
                        \wrd{3367}{nikt}
                        \wrd{}{(nie)}
                        \underline{\wrd{1097}{dowiedział się}}
                        \wrd{}{(o tym).}
                \bverse{31}
                        \wrd{1161}{A~}
                        \wrd{3588}{oni,}
                        \underline{\wrd{1831}{gdy wyszli,}}
                        \wrd{1310}{rozpowszechnili}
                        \wrd{846}{to}
                        \wrd{1722}{po}
                        \wrd{3650}{całej}
                        \wrd{1565}{tamtejszej}
                        \wrd{1093}{ziemi.}
                \bverse{32}
                        \wrd{1161}{A~}
                        \underline{\wrd{846}{gdy oni}}
                        \wrd{1831}{wychodzili,}
                        \wrd{2400}{oto}
                        \wrd{4374}{przyniesiono}
                        \wrd{846}{mu}
                        \wrd{2974}{głuchego}
                        \wrd{444}{człowieka,}
                        \wrd{1139}{opętanego.}
                \bverse{33}
                        \wrd{2532}{A~}
                        \underline{\wrd{1544}{gdy został wyrzucony}}
                        \wrd{1140}{demon,}
                        \wrd{2974}{głuchy}
                        \wrd{2980}{przemówił.}
                        \wrd{2532}{I~}
                        \doubleline{\wrd{2296}{zdziwiły się}}
                        \wrd{3793}{tłumy,}
                        \wrd{3004}{mówiąc}
                        \underline{\wrd{3763}{Nigdy nie}}
                        \doubleline{\wrd{5316}{ukazała się}}
                        \wrd{3779}{taka}
                        \wrd{}{(rzecz)}
                        \wrd{1722}{w~}
                        \wrd{2474}{Izraelu.}
                \bverse{34}
                        \wrd{1161}{Lecz}
                        \wrd{5330}{faryzeusze}
                        \wrd{3004}{mówili:}
                        \wrd{1544}{Wyrzuca}
                        \wrd{1140}{demony}
                        \wrd{1722}{poprzez}
                        \underline{\wrd{758}{posiadającego władzę}}
                        \wrd{}{(nad)}
                        \wrd{1140}{demonami.}
                \bverse{35}
                        \wrd{2532}{I~}
                        \wrd{4013}{obchodził}
                        \wrd{2424}{Jezus}
                        \wrd{3956}{wszystkie}
                        \wrd{4172}{miasta}
                        \wrd{2532}{i~}
                        \wrd{2968}{wsie,}
                        \wrd{1321}{nauczając}
                        \wrd{1722}{w~}
                        \wrd{846}{ich}
                        \wrd{4864}{synagogach}
                        \wrd{2532}{i~}
                        \wrd{2784}{ogłaszając}
                        \underline{\wrd{2098}{dobrą nowinę}}
                        \wrd{932}{królestwa}
                        \wrd{2532}{oraz}
                        \wrd{2323}{uzdrawiając}
                        \wrd{3956}{każdą}
                        \wrd{3554}{chorobę}
                        \wrd{2532}{i~}
                        \wrd{3956}{każdą}
                        \wrd{3119}{niemoc}
                        \wrd{1722}{wśród}
                        \wrd{2992}{ludu.}
                \bverse{36}
                        \wrd{1161}{A~}
                        \underline{\wrd{1492}{gdy dostrzegł}}
                        \wrd{3793}{tłumy,}
                        \doubleline{\wrd{4697}{ulitował się}}
                        \wrd{4012}{nad}
                        \wrd{846}{nimi,}
                        \wrd{3754}{bo}
                        \wrd{2258}{byli}
                        \wrd{4660}{utrudzeni}
                        \wrd{2532}{i~}
                        \wrd{4496}{porzuceni,}
                        \wrd{5616}{jak}
                        \wrd{4263}{owce,}
                        \underline{\wrd{2192}{które}}
                        \wrd{3361}{nie}
                        \underline{\wrd{2192}{mają}}
                        \wrd{4166}{pasterza.}
                \bverse{37}
                        \wrd{5119}{Wtedy}
                        \wrd{3004}{powiedział}
                        \wrd{846}{swoim}
                        \wrd{3101}{uczniom:}
                        \wrd{3303}{Chociaż}
                        \wrd{2326}{żniwo}
                        \wrd{4183}{wielkie,}
                        \wrd{1161}{ale}
                        \wrd{2040}{robotników}
                        \wrd{3641}{niewielu.}
                \bverse{38}
                        \wrd{1189}{Proście}
                        \wrd{3767}{więc}
                        \wrd{2962}{Pana}
                        \wrd{2326}{żniwa,}
                        \wrd{3704}{aby}
                        \wrd{1544}{wyrzucił}
                        \wrd{2040}{robotników}
                        \wrd{1519}{na}
                        \wrd{846}{swoje}
                        \wrd{2326}{żniwo.} \section{} \IndTwo
                \bverse{1}
                        \wrd{2532}{A~}
                        \underline{\wrd{4341}{gdy przywołał}}
                        \wrd{846}{swoich}
                        \wrd{1427}{dwunastu}
                        \wrd{3101}{uczniów,}
                        \wrd{1325}{dał}
                        \wrd{846}{im}
                        \wrd{1849}{moc}
                        \doubleline{\wrd{4151}{nad duchami}}
                        \wrd{169}{nieczystymi,}
                        \wrd{5620}{by}
                        \wrd{846}{je}
                        \wrd{1544}{wyrzucali}
                        \wrd{2532}{oraz}
                        \wrd{2323}{uzdrawiali}
                        \wrd{3956}{każdą}
                        \wrd{3554}{chorobę}
                        \wrd{2532}{i~}
                        \wrd{3956}{każdą}
                        \wrd{3119}{niemoc.}
                \bverse{2}
                        \wrd{1161}{A~}
                        \wrd{3686}{imiona}
                        \wrd{1427}{dwunastu}
                        \wrd{652}{apostołów}
                        \wrd{2076}{są}
                        \wrd{5023}{takie:}
                        \wrd{4413}{pierwszym–}
                        \wrd{4613}{Szymon,}
                        \wrd{3004}{zwany}
                        \wrd{4074}{Piotrem}
                        \wrd{2532}{oraz}
                        \wrd{846}{jego}
                        \wrd{80}{brat,}
                        \wrd{406}{Andrzej;}
                        \wrd{2385}{Jakub–}
                        \wrd{3588}{ten,}
                        \wrd{}{(który jest synem)}
                        \wrd{2199}{Zebedeusza}
                        \wrd{2532}{oraz}
                        \wrd{846}{jego}
                        \wrd{80}{brat,}
                        \wrd{2491}{Jan;}
                \bverse{3}
                        \wrd{5376}{Filip}
                        \wrd{2532}{i~}
                        \wrd{918}{Bartłomiej;}
                        \wrd{2381}{Tomasz}
                        \wrd{2532}{i~}
                        \wrd{3156}{Mateusz–}
                        \underline{\wrd{5057}{poborca podatków;}}
                        \wrd{2385}{Jakub–}
                        \wrd{3588}{ten,}
                        \wrd{}{(który jest synem)}
                        \wrd{256}{Alfeusza}
                        \wrd{2532}{i~}
                        \wrd{3002}{Lebeusz,}
                        \wrd{1941}{zwany}
                        \wrd{2280}{Tadeuszem;}
                \bverse{4}
                        \wrd{4613}{Szymon}
                        \wrd{2581}{Kananejczyk}
                        \wrd{2532}{i~}
                        \wrd{2455}{Juda}
                        \wrd{2469}{Iskariota–}
                        \underline{\wrd{3588}{ten, co}}
                        \wrd{846}{go}
                        \wrd{2532}{również}
                        \wrd{3860}{wydał.}
                \bverse{5}
                        \wrd{5128}{Tę}
                        \wrd{1427}{dwunastkę}
                        \wrd{649}{posłał}
                        \wrd{2424}{Jezus}
                        \wrd{}{(i)}
                        \wrd{3853}{nakazał}
                        \wrd{846}{im,}
                        \wrd{3004}{mówiąc:}
                        \wrd{1519}{Na}
                        \wrd{3598}{drogę}
                        \wrd{1484}{narodów}
                        \wrd{3361}{nie}
                        \wrd{565}{wchodźcie}
                        \wrd{2532}{oraz}
                        \wrd{3361}{nie}
                        \wrd{1525}{wchodźcie}
                        \wrd{1519}{do}
                        \wrd{4172}{miast}
                        \wrd{4541}{Samarytan.}
                \bverse{6}
                        \wrd{1161}{Ale}
                        \wrd{4198}{idźcie}
                        \wrd{3123}{raczej}
                        \wrd{4314}{do}
                        \wrd{622}{zagubionych}
                        \wrd{4263}{owiec}
                        \wrd{3624}{domu}
                        \wrd{2474}{Izraela.}
                \bverse{7}
                        \wrd{1161}{A~}
                        \wrd{4198}{idąc,}
                        \wrd{2784}{głoście,}
                        \wrd{3004}{mówiąc,}
                        \wrd{3754}{że}
                        \underline{\wrd{1448}{przybliżyło się}}
                        \wrd{932}{królestwo}
                        \wrd{3772}{niebios.}
                \bverse{8}
                        \wrd{770}{Słabych}
                        \wrd{2323}{uzdrawiajcie,}
                        \wrd{3015}{trędowatych}
                        \wrd{2511}{oczyszczajcie,}
                        \wrd{1544}{wyrzucajcie}
                        \wrd{1140}{demony.}
                        \wrd{1432}{Darmo}
                        \wrd{2983}{wzięliście,}
                        \wrd{1432}{darmo}
                        \wrd{1325}{dawajcie.}
                \bverse{9}
                        \wrd{3361}{Nie}
                        \wrd{2932}{posiadajcie}
                        \wrd{5557}{złota}
                        \wrd{3366}{ani}
                        \wrd{696}{srebra,}
                        \wrd{3366}{ani}
                        \wrd{5475}{miedzi}
                        \wrd{1519}{w~}
                        \wrd{5216}{waszych}
                        \wrd{2223}{pasach,}
                \bverse{10}
                        \wrd{3361}{ani}
                        \wrd{4082}{torby}
                        \wrd{1519}{na}
                        \wrd{3598}{drogę,}
                        \wrd{3366}{ani}
                        \wrd{1417}{dwóch}
                        \underline{\wrd{5509}{szat spodnich,}}
                        \wrd{3366}{ani}
                        \wrd{5266}{sandałów,}
                        \wrd{3366}{ani}
                        \wrd{4464}{laski,}
                        \wrd{1063}{ponieważ}
                        \wrd{2040}{robotnik}
                        \wrd{514}{godny}
                        \wrd{2076}{jest}
                        \wrd{846}{swego}
                        \wrd{5160}{pokarmu.}
                \bverse{11}
                        \wrd{1161}{A~}
                        \wrd{302}{[-]}
                        \wrd{}{(w)}
                        \wrd{4172}{mieście}
                        \wrd{2228}{lub}
                        \wrd{2968}{wsi,}
                        \wrd{1519}{do}
                        \wrd{3739}{której}
                        \wrd{1525}{wejdziecie,}
                        \wrd{1833}{wypytajcie,}
                        \wrd{5101}{kto}
                        \wrd{1722}{w~}
                        \wrd{846}{niej}
                        \wrd{2076}{jest}
                        \wrd{514}{godny.}
                        \underline{\wrd{2546}{I tam}}
                        \wrd{3306}{pozostańcie,}
                        \doubleline{\wrd{2193}{aż do}}
                        \wrd{302}{[-]}
                        \wrd{1831}{wyjścia.}
                \bverse{12}
                        \wrd{1161}{A~}
                        \underline{\wrd{1525}{gdy wejdziecie}}
                        \wrd{1519}{do}
                        \wrd{3614}{domu,}
                        \wrd{782}{pozdrówcie}
                        \wrd{846}{go.}
                \bverse{13}
                        \wrd{2532}{I~}
                        \wrd{1437}{jeśli}
                        \wrd{3303}{[-]}
                        \wrd{3614}{dom}
                        \underline{\wrd{5600}{okaże się}}
                        \wrd{514}{godny,}
                        \doubleline{\wrd{2064}{niech przyjdzie}}
                        \wrd{1515}{pokój}
                        \wrd{5216}{wasz}
                        \wrd{1909}{na}
                        \wrd{846}{niego.}
                        \wrd{1161}{A~}
                        \wrd{1437}{gdyby}
                        \wrd{3361}{nie}
                        \wrd{5600}{był}
                        \wrd{514}{godny,}
                        \wrd{5216}{wasz}
                        \wrd{1515}{pokój}
                        \underline{\wrd{4314}{niech powróci}}
                        \wrd{1994}{do}
                        \wrd{5209}{was.}
                \bverse{14}
                        \wrd{2532}{A~}
                        \wrd{3739}{kto}
                        \wrd{3361}{nie}
                        \wrd{1437}{przyjąłby}
                        \wrd{1209}{przyjąłby}
                        \wrd{5209}{was}
                        \wrd{3366}{ani}
                        \wrd{}{(nie)}
                        \wrd{191}{wysłuchał}
                        \wrd{3056}{słów}
                        \wrd{5216}{waszych,}
                        \underline{\wrd{1831}{wychodząc z}}
                        \wrd{3614}{domu}
                        \wrd{2228}{lub}
                        \wrd{}{(z)}
                        \wrd{1565}{tego}
                        \wrd{4172}{miasta,}
                        \wrd{1621}{strząśnijcie}
                        \wrd{2868}{pył}
                        \wrd{}{(ze)}
                        \wrd{4228}{stóp}
                        \wrd{5216}{waszych.}
                \bverse{15}
                        \wrd{281}{Zaprawdę,}
                        \wrd{3004}{mówię}
                        \wrd{5213}{wam:}
                        \wrd{414}{Lżej}
                        \wrd{2071}{będzie}
                        \wrd{1093}{ziemi}
                        \wrd{4670}{Sodomy}
                        \wrd{2532}{i~}
                        \wrd{1116}{Gomory}
                        \wrd{1722}{w~}
                        \wrd{2250}{dniu}
                        \wrd{2920}{sądu,}
                        \wrd{2228}{niż}
                        \wrd{1565}{temu}
                        \wrd{4172}{miastu.}
                \bverse{16}
                        \wrd{2400}{Oto}
                        \wrd{1473}{ja}
                        \wrd{5209}{was}
                        \wrd{649}{posyłam}
                        \wrd{5613}{jak}
                        \wrd{4263}{owce}
                        \wrd{1722}{między}
                        \wrd{3319}{między}
                        \wrd{3074}{wilki.}
                        \wrd{1096}{Bądźcie}
                        \wrd{3767}{więc}
                        \wrd{5429}{mądrzy,}
                        \wrd{5613}{jak}
                        \wrd{3789}{węże}
                        \wrd{2532}{i~}
                        \wrd{185}{czyści,}
                        \wrd{5613}{jak}
                        \wrd{4058}{gołębie.}
                \bverse{17}
                        \wrd{1161}{I~}
                        \underline{\wrd{4337}{wystrzegajcie się}}
                        \wrd{575}{[-]}
                        \wrd{444}{ludzi,}
                        \wrd{1063}{ponieważ}
                        \doubleline{\wrd{3860}{będą wydawać}}
                        \wrd{5209}{was}
                        \wrd{1519}{przed}
                        \wrd{4892}{sanhedryny}
                        \wrd{2532}{i~}
                        \wrd{1722}{w~}
                        \wrd{846}{swoich}
                        \wrd{4864}{synagogach}
                        \underline{\wrd{3146}{będą}}
                        \wrd{5209}{was}
                        \underline{\wrd{3146}{biczować.}}
                \bverse{18}
                        \wrd{2532}{Oraz}
                        \wrd{1909}{przed}
                        \wrd{2232}{rządców}
                        \wrd{1161}{[-]}
                        \wrd{2532}{i~}
                        \wrd{935}{królów}
                        \underline{\wrd{71}{będziecie prowadzeni}}
                        \doubleline{\wrd{1752}{z~}}
                        \wrd{1700}{mojego}
                        \doubleline{\wrd{1752}{powodu,}}
                        \wrd{1519}{na}
                        \wrd{3142}{świadectwo}
                        \wrd{846}{im}
                        \wrd{2532}{oraz}
                        \wrd{1484}{narodom.}
                \bverse{19}
                        \wrd{1161}{A~}
                        \wrd{3752}{gdy}
                        \wrd{5209}{was}
                        \wrd{3860}{wydadzą,}
                        \wrd{3361}{nie}
                        \underline{\wrd{3309}{troszczcie się,}}
                        \wrd{4459}{jak}
                        \wrd{2228}{lub}
                        \wrd{5101}{co}
                        \wrd{2980}{powiecie,}
                        \doubleline{\wrd{1063}{bowiem będzie}}
                        \wrd{1325}{dane}
                        \wrd{5213}{wam}
                        \wrd{1722}{w~}
                        \wrd{1565}{tej}
                        \wrd{5610}{chwili,}
                        \wrd{5101}{co}
                        \underline{\wrd{2980}{macie mówić,}}
                \bverse{20}
                        \wrd{1063}{ponieważ}
                        \wrd{3756}{nie}
                        \wrd{5210}{wy}
                        \wrd{2075}{jesteście}
                        \underline{\wrd{2980}{tymi, którzy mówią,}}
                        \wrd{235}{ale}
                        \wrd{4151}{duch}
                        \wrd{5216}{waszego}
                        \wrd{3962}{Ojca,}
                        \doubleline{\wrd{2980}{który mówi}}
                        \wrd{1722}{w~}
                        \wrd{5213}{was.}
                \bverse{21}
                        \wrd{1161}{A~}
                        \wrd{80}{brat}
                        \wrd{3860}{wyda}
                        \wrd{80}{brata}
                        \wrd{1519}{na}
                        \wrd{2288}{śmierć}
                        \wrd{2532}{i~}
                        \wrd{3962}{ojciec}
                        \wrd{5043}{dziecko,}
                        \wrd{2532}{i~}
                        \wrd{1881}{powstaną}
                        \wrd{5043}{dzieci}
                        \wrd{1909}{przeciwko}
                        \wrd{1118}{rodzicom,}
                        \wrd{2532}{i~}
                        \wrd{2289}{zabiją}
                        \wrd{846}{ich.}
                \bverse{22}
                        \wrd{2532}{I~}
                        \wrd{2071}{będziecie}
                        \wrd{3404}{znienawidzeni}
                        \wrd{5259}{przez}
                        \wrd{3956}{wszystkich}
                        \underline{\wrd{1223}{z powodu}}
                        \wrd{3450}{mojego}
                        \wrd{3686}{imienia.}
                        \wrd{1161}{A~}
                        \doubleline{\wrd{5278}{kto wytrwa}}
                        \wrd{1519}{do}
                        \wrd{5056}{końca,}
                        \wrd{3778}{ten}
                        \underline{\wrd{4982}{zostanie wybawiony.}}
                \bverse{23}
                        \wrd{1161}{A~}
                        \wrd{3752}{gdyby}
                        \wrd{5209}{was}
                        \wrd{1377}{prześladowali}
                        \wrd{1722}{w~}
                        \wrd{3778}{tym}
                        \wrd{4172}{mieście,}
                        \wrd{5343}{uciekajcie}
                        \wrd{1519}{do}
                        \wrd{243}{drugiego.}
                        \wrd{281}{Zaprawdę}
                        \wrd{1063}{bowiem}
                        \wrd{3004}{mówię}
                        \wrd{5213}{wam,}
                        \wrd{3756}{nie}
                        \wrd{3361}{[-]}
                        \wrd{5055}{skończycie}
                        \wrd{}{(obchodzić)}
                        \wrd{4172}{miast}
                        \wrd{2474}{Izraela,}
                        \wrd{2193}{aż}
                        \wrd{302}{[-]}
                        \wrd{2064}{przyjdzie}
                        \wrd{5207}{Syn}
                        \wrd{444}{Człowieczy.}
                \bverse{24}
                        \wrd{3101}{Uczeń}
                        \wrd{3756}{nie}
                        \wrd{2076}{jest}
                        \wrd{5228}{ponad}
                        \wrd{1320}{nauczycielem,}
                        \wrd{3761}{ani}
                        \wrd{1401}{niewolnik}
                        \wrd{5228}{ponad}
                        \wrd{846}{swego}
                        \wrd{2962}{pana.}
                \bverse{25}
                        \wrd{713}{Wystarczające}
                        \underline{\wrd{3101}{dla ucznia,}}
                        \wrd{2443}{aby}
                        \doubleline{\wrd{1096}{się stał}}
                        \wrd{5613}{jak}
                        \wrd{846}{jego}
                        \wrd{1320}{nauczyciel,}
                        \wrd{2532}{i~}
                        \wrd{1401}{niewolnik–}
                        \wrd{5613}{jak}
                        \wrd{846}{jego}
                        \wrd{2962}{pan.}
                        \wrd{1487}{Jeśli}
                        \underline{\wrd{3617}{pana domu}}
                        \wrd{2564}{nazywali}
                        \wrd{954}{Belzebubem,}
                        \wrd{}{(to o)}
                        \wrd{4214}{ile}
                        \wrd{3123}{bardziej}
                        \wrd{846}{jego}
                        \wrd{3615}{domowników.}
                \bverse{26}
                        \wrd{3767}{Więc}
                        \wrd{3361}{nie}
                        \underline{\wrd{5399}{bójcie się}}
                        \wrd{846}{ich.}
                        \doubleline{\wrd{2076}{Nie ma}}
                        \wrd{1063}{bowiem}
                        \wrd{3762}{nic}
                        \wrd{2572}{ukrytego,}
                        \wrd{3739}{co}
                        \wrd{3756}{nie}
                        \underline{\wrd{601}{będzie objawione,}}
                        \wrd{2532}{ani}
                        \wrd{2927}{ukrytego,}
                        \wrd{3739}{co}
                        \wrd{3756}{nie}
                        \doubleline{\wrd{1097}{będzie poznane.}}
                \bverse{27}
                        \underline{\wrd{3739}{To, co}}
                        \wrd{3004}{mówię}
                        \wrd{5213}{wam}
                        \wrd{1722}{w~}
                        \wrd{4653}{ciemności,}
                        \wrd{2036}{powiedzcie}
                        \wrd{1722}{w~}
                        \wrd{5457}{świetle,}
                        \wrd{2532}{i~}
                        \wrd{3739}{co}
                        \wrd{1519}{na}
                        \wrd{3775}{ucho}
                        \wrd{191}{słyszycie,}
                        \wrd{2784}{głoście}
                        \wrd{1909}{na}
                        \wrd{1430}{dachach.}
                \bverse{28}
                        \wrd{2532}{I~}
                        \wrd{3361}{nie}
                        \underline{\wrd{5399}{bójcie się}}
                        \wrd{575}{[-]}
                        \doubleline{\wrd{615}{tych, którzy zabijają}}
                        \wrd{4983}{ciało,}
                        \wrd{1161}{ale}
                        \wrd{5590}{duszy}
                        \wrd{3361}{nie}
                        \underline{\wrd{1410}{są w stanie}}
                        \wrd{615}{zabić.}
                        \wrd{1161}{Ale}
                        \doubleline{\wrd{5399}{bójcie się}}
                        \wrd{3123}{raczej}
                        \underline{\wrd{1410}{tego, co może}}
                        \wrd{2532}{i~}
                        \wrd{5590}{duszę,}
                        \wrd{2532}{i~}
                        \wrd{4983}{ciało}
                        \wrd{622}{zniszczyć}
                        \wrd{1722}{w~}
                        \wrd{1067}{Gehennie.}
                \bverse{29}
                        \underline{\wrd{3780}{Czy nie}}
                        \doubleline{\wrd{787}{za asa}}
                        \underline{\wrd{4453}{są sprzedawane}}
                        \wrd{1417}{dwa}
                        \wrd{4765}{wróble?}
                        \wrd{2532}{Nawet}
                        \wrd{1520}{jeden}
                        \wrd{1537}{z~}
                        \wrd{846}{nich}
                        \wrd{3756}{nie}
                        \wrd{4098}{spadnie}
                        \wrd{1909}{na}
                        \wrd{1093}{ziemię}
                        \wrd{427}{bez}
                        \wrd{}{(woli)}
                        \wrd{5216}{waszego}
                        \wrd{3962}{Ojca.}
                \bverse{30}
                        \wrd{1161}{A~}
                        \wrd{2532}{nawet}
                        \wrd{3956}{wszystkie}
                        \wrd{2359}{włosy}
                        \underline{\wrd{5216}{na waszej}}
                        \wrd{2776}{głowie}
                        \wrd{1526}{są}
                        \wrd{705}{policzone.}
                \bverse{31}
                        \underline{\wrd{3767}{A zatem:}}
                        \wrd{3361}{nie}
                        \doubleline{\wrd{5399}{bójcie się.}}
                        \wrd{5210}{Wy}
                        \underline{\wrd{1308}{jesteście ważniejsi,}}
                        \doubleline{\wrd{4183}{niż liczne}}
                        \wrd{4765}{wróble.}
                \bverse{32}
                        \wrd{3956}{Każdy}
                        \wrd{3767}{więc,}
                        \wrd{3748}{kto}
                        \underline{\wrd{3670}{przyzna się}}
                        \wrd{1722}{do}
                        \wrd{1698}{mnie}
                        \wrd{1715}{przed}
                        \wrd{444}{ludźmi,}
                        \doubleline{\wrd{2504}{i ja}}
                        \underline{\wrd{3670}{się przyznam}}
                        \wrd{1722}{do}
                        \wrd{846}{niego}
                        \wrd{1715}{przed}
                        \wrd{3450}{moim}
                        \wrd{3962}{Ojcem}
                        \wrd{1722}{w~}
                        \wrd{3772}{niebiosach.}
                \bverse{33}
                        \wrd{1161}{A~}
                        \wrd{3748}{kto}
                        \wrd{302}{[-]}
                        \underline{\wrd{720}{by się wyparł}}
                        \wrd{3165}{mnie}
                        \wrd{1715}{przed}
                        \wrd{444}{ludźmi,}
                        \doubleline{\wrd{2504}{i ja}}
                        \underline{\wrd{720}{się wyprę}}
                        \wrd{846}{go}
                        \wrd{1715}{przed}
                        \wrd{3450}{moim}
                        \wrd{3962}{Ojcem}
                        \wrd{1722}{w~}
                        \wrd{3772}{niebiosach.}
                \bverse{34}
                        \wrd{3361}{Nie}
                        \wrd{3543}{myślcie,}
                        \wrd{3754}{że}
                        \wrd{2064}{przyszedłem}
                        \wrd{906}{rzucić}
                        \wrd{1515}{pokój}
                        \wrd{1909}{na}
                        \wrd{1093}{ziemię.}
                        \wrd{3756}{Nie}
                        \wrd{2064}{przyszedłem}
                        \wrd{906}{rzucić}
                        \wrd{1515}{pokoju,}
                        \wrd{235}{ale}
                        \wrd{3162}{miecz.}
                \bverse{35}
                        \wrd{1063}{Ponieważ}
                        \wrd{2064}{przyszedłem}
                        \underline{\wrd{1369}{uczynić rozdwojenie,}}
                        \wrd{}{(aby)}
                        \wrd{444}{człowiek}
                        \wrd{}{(był)}
                        \wrd{2596}{przeciwko}
                        \wrd{846}{swemu}
                        \wrd{3962}{ojcu,}
                        \wrd{2532}{i~}
                        \wrd{2364}{córka–}
                        \wrd{2596}{przeciwko}
                        \wrd{846}{swojej}
                        \wrd{3384}{matce,}
                        \wrd{2532}{i~}
                        \doubleline{\wrd{3565}{panna młoda}}
                        \wrd{2596}{przeciwko}
                        \wrd{846}{swojej}
                        \wrd{3994}{teściowej,}
                \bverse{36}
                        \wrd{2532}{a~}
                        \wrd{2190}{nieprzyjaciółmi}
                        \wrd{444}{człowieka}
                        \wrd{3615}{domownicy}
                        \wrd{846}{jego.}
                \bverse{37}
                        \underline{\wrd{5368}{Kto kocha}}
                        \wrd{3962}{ojca}
                        \wrd{2228}{lub}
                        \wrd{3384}{matkę}
                        \doubleline{\wrd{5228}{bardziej, niż}}
                        \wrd{1691}{mnie,}
                        \wrd{3756}{nie}
                        \wrd{2076}{jest}
                        \wrd{3450}{mnie}
                        \wrd{514}{godny.}
                        \wrd{2532}{I~}
                        \underline{\wrd{5368}{kto kocha}}
                        \wrd{5207}{syna}
                        \wrd{2228}{lub}
                        \wrd{2364}{córkę}
                        \doubleline{\wrd{5228}{bardziej, niż}}
                        \wrd{1691}{mnie,}
                        \wrd{3756}{nie}
                        \wrd{2076}{jest}
                        \wrd{3450}{mnie}
                        \wrd{514}{godny.}
                \bverse{38}
                        \wrd{2532}{I~}
                        \wrd{3739}{kto}
                        \wrd{3756}{nie}
                        \wrd{2983}{bierze}
                        \wrd{846}{swego}
                        \wrd{4716}{krzyża,}
                        \wrd{2532}{a~}
                        \wrd{190}{idzie}
                        \wrd{3694}{za}
                        \wrd{3450}{mną,}
                        \wrd{3756}{nie}
                        \wrd{2076}{jest}
                        \wrd{3450}{mnie}
                        \wrd{514}{godny.}
                \bverse{39}
                        \wrd{3588}{Ten,}
                        \underline{\wrd{2147}{który znalazł}}
                        \wrd{846}{swoją}
                        \wrd{5590}{duszę,}
                        \wrd{622}{straci}
                        \wrd{846}{ją,}
                        \wrd{2532}{a~}
                        \wrd{3588}{ten,}
                        \doubleline{\wrd{622}{kto stracił}}
                        \wrd{846}{swoją}
                        \wrd{5590}{duszę}
                        \underline{\wrd{1752}{z powodu}}
                        \wrd{1700}{mnie,}
                        \wrd{2147}{znajdzie}
                        \wrd{846}{ją.}
                \bverse{40}
                        \wrd{3588}{Ten,}
                        \underline{\wrd{1209}{kto przyjmuje}}
                        \wrd{5209}{was,}
                        \wrd{1691}{mnie}
                        \wrd{1209}{przyjmuje;}
                        \wrd{2532}{a~}
                        \wrd{3588}{ten,}
                        \doubleline{\wrd{1209}{który przyjmuje}}
                        \wrd{1691}{mnie,}
                        \wrd{1209}{przyjmuje}
                        \wrd{3588}{tego,}
                        \underline{\wrd{649}{który}}
                        \wrd{3165}{mnie}
                        \underline{\wrd{649}{posłał.}}
                \bverse{41}
                        \underline{\wrd{1209}{Kto przyjmuje}}
                        \wrd{4396}{proroka}
                        \doubleline{\wrd{1519}{ze względu na}}
                        \wrd{3686}{imię}
                        \wrd{4396}{proroka,}
                        \wrd{2983}{weźmie}
                        \wrd{3408}{zapłatę}
                        \wrd{4396}{proroka,}
                        \wrd{2532}{i~}
                        \underline{\wrd{1209}{ten, który przyjmuje}}
                        \wrd{1342}{sprawiedliwego}
                        \doubleline{\wrd{1519}{ze względu na}}
                        \wrd{3686}{imię}
                        \wrd{1342}{sprawiedliwego,}
                        \wrd{2983}{weźmie}
                        \wrd{3408}{zapłatę}
                        \wrd{1342}{sprawiedliwego,}
                \bverse{42}
                        \wrd{2532}{i~}
                        \wrd{3739}{kto}
                        \wrd{1437}{by}
                        \wrd{4222}{napoił}
                        \wrd{1520}{jednego}
                        \wrd{}{(z)}
                        \wrd{5130}{tych}
                        \wrd{3398}{małych}
                        \wrd{4221}{kielichem}
                        \wrd{5593}{zimnej}
                        \wrd{}{(wody)}
                        \wrd{3440}{jedynie}
                        \underline{\wrd{1519}{ze względu na}}
                        \wrd{3686}{imię}
                        \wrd{3101}{ucznia,}
                        \wrd{281}{zaprawdę,}
                        \wrd{3004}{mówię}
                        \wrd{5213}{wam,}
                        \wrd{3756}{nie}
                        \wrd{3361}{[-]}
                        \wrd{622}{straci}
                        \wrd{846}{swojej}
                        \wrd{3408}{zapłaty.}
\end{document} 