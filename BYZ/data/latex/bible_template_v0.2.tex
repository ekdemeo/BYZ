\documentclass[10pt,paper=a5,pagesize=pdftex]{article}
\usepackage[utf8]{inputenc}
\usepackage[T1]{fontenc}
\usepackage{polski}
\usepackage[pagestyles]{titlesec}
\usepackage{lettrine}
\usepackage{setspace}
\usepackage[polish]{babel}
\usepackage{soul}
\usepackage{ebgaramond}
%usepackage{lmodern}  
\usepackage{lipsum}
\usepackage{ragged2e}
%\usepackage[usenames,dvipsnames,svgnames,table]{xcolor}
%\usepackage{fancyhdr}
%  \pagestyle{fancy}
%  \fancyhf{}
%  \fancyhead[RO,LE]{\rightmark}
%  \renewcommand{\headrulewidth}{.5pt}
\selectlanguage{polish}



\newcounter{versecounter}
\renewcommand{\thesection}{\arabic{section}}
\newcommand{\theverse}{\arabic{versecounter}}

% fonts
\font\strong=phvr at 5pt

% definition of the page style with required headers
%\newpagestyle{Biblestyle}{
%  \setheadrule{.009pt}
%  \sethead[\thepage][\chaptertitle]
%    [\toptitlemarks\thesection:\toptitlemarks\theverse---%
%      \bottitlemarks\thesection:\bottitlemarks\theverse]%
%   {\toptitlemarks\thesection:\toptitlemarks\theverse---%
%     \bottitlemarks\thesection:\bottitlemarks\theverse}{\chaptertitle}{\thepage}
%}

% definition of the page style with required headers
\newpagestyle{Biblestyle}{
  \setheadrule{.009pt}
  \sethead[\thepage][\chaptertitle][]{}{\chaptertitle}{\thepage}
  %  [\toptitlemarks\thesection:\toptitlemar{ks\theverse---%
  %    \bottitlemarks\thesection:\bottitlemarks\theverse]%
  % {\toptitlemarks\thesection:\toptitlemarks\theverse---%
  %   \bottitlemarks\thesection:\bottitlemarks\theverse}{\chaptertitle}{\thepage}
}


% sets the marks to be used (section and subsection)
\setmarks{section}{subsection}

% sections and subsections formatting
\titleformat{\section}
{}{\lettrine{\thesection}}{0em}{}[\vskip-0.65\baselineskip]
\titleformat{\subsection}[runin]
{\small\bfseries}{\thesubsection}{1em}{}
\titlespacing{\section}{1em}{-1pt}{0pt}


\pagestyle{Biblestyle}

\renewcommand{\LettrineFontHook}{\bfseries}

\setlength{\parindent}{1pt}

\newlength\NumLen
\newlength\LinLen
% indents one line of text. Indentation= width of section number + 1em
\newcommand\IndOne{%
  \settowidth\NumLen{\thesection}
  \addtolength\NumLen{0.7em}
  \setlength\LinLen{\dimexpr\textwidth-\NumLen}%\the\NumLen\the\LinLen
  \parshape 2 \NumLen \LinLen 0pt \textwidth}
  
% indents two lines of text. Indentation= width of section number + 1em
\newcommand\IndTwo{%
  \settowidth\NumLen{\thesection}
  \addtolength\NumLen{0.7em}
  \setlength\LinLen{\dimexpr\textwidth-\NumLen}%\the\NumLen\the\LinLen
  \parshape 3 \NumLen \LinLen \NumLen \LinLen 0pt \textwidth}
 
% macros

% word
\newcommand\wrd[2]{%
  \leavevmode
  \vbox{\offinterlineskip
    \halign{%
      \hfil##\hfil\cr
      {\strong\vphantom{p}#1}\cr
      \noalign{\vskip\lineskip}%
      \vphantom{A}#2\cr
    }%
  }%
}

\newcommand\doubleline[1]{\underline{{\underline{#1}}}}
\newcommand*\bverse[1]{%
 \stepcounter{versecounter}%
 \textsuperscript{\textbf{\normalsize  #1}}}
 
\begin{document}
\doublespacing
\renewcommand{\chaptertitle}{Mateusza}
\justify
 \section{} \IndTwo
                \bverse{1}
                        \underline{\wrd{976}{Zwój księgi}}
                        \wrd{}{(o)}
                        \wrd{1078}{narodzinach}
                        \wrd{2424}{Jezusa}
                        \wrd{5547}{Chrystusa,}
                        \wrd{5207}{syna}
                        \wrd{1138}{Dawida,}
                        \wrd{5207}{syna}
                        \wrd{11}{Abrahama.}
                \bverse{2}
                        \wrd{11}{Abraham}
                        \wrd{1080}{zrodził}
                        \wrd{2464}{Izaaka,}
                        \wrd{1161}{a~}
                        \wrd{2464}{Izaak}
                        \wrd{1080}{zrodził}
                        \wrd{2384}{Jakuba,}
                        \wrd{1161}{a~}
                        \wrd{2384}{Jakub}
                        \wrd{1080}{zrodził}
                        \wrd{2455}{Judę}
                        \wrd{2532}{i~}
                        \wrd{80}{braci}
                        \wrd{846}{jego,}
                \bverse{3}
                        \wrd{1161}{a~}
                        \wrd{2455}{Juda}
                        \wrd{1080}{zrodził}
                        \wrd{5329}{Faresa}
                        \wrd{2532}{i~}
                        \wrd{2196}{Zarę}
                        \wrd{1537}{z~}
                        \wrd{2283}{Tamary,}
                        \wrd{1161}{a~}
                        \wrd{5329}{Fares}
                        \wrd{1080}{zrodził}
                        \wrd{2074}{Esroma,}
                        \wrd{1161}{a~}
                        \wrd{2074}{Esrom}
                        \wrd{1080}{zrodził}
                        \wrd{689}{Arama,}
                
                \bverse{18}
                        \wrd{1161}{A~}
                        \wrd{1083}{narodzenie}
                        \wrd{2424}{Jezusa}
                        \wrd{5547}{Chrystusa}
                        \wrd{2258}{było}
                        \wrd{3779}{takie:}
                        \wrd{1063}{ponieważ}
                        \wrd{3384}{matka}
                        \wrd{846}{jego,}
                        \wrd{3137}{Maria,}
                        \underline{\wrd{3423}{będąc zaślubiona}}
                        \wrd{2501}{Józefowi,}
                        \doubleline{\wrd{4250}{wcześniej, zanim}}
                        \wrd{2228}{[-]}
                        \underline{\wrd{4905}{się}}
                        \wrd{846}{oni}
                        \underline{\wrd{4905}{zeszli,}}
                        \doubleline{\wrd{2147}{znalazła się}}
                        \underline{\wrd{2192}{tą, która ma}}
                        \wrd{}{(dziecko)}
                        \wrd{1722}{w~}
                        \wrd{1064}{łonie}
                        \doubleline{\wrd{1537}{za sprawą}}
                        \wrd{4151}{Ducha}
                        \wrd{40}{Świętego,}
                \bverse{19}
                        \wrd{1161}{a~}
                        \wrd{2501}{Józef,}
                        \wrd{846}{jej}
                        \wrd{435}{mąż,}
                        \wrd{5607}{będąc}
                        \wrd{1342}{sprawiedliwym}
                        \wrd{2532}{i~}
                        \wrd{3361}{nie}
                        \wrd{2309}{chcąc}
                        \wrd{846}{jej}
                        \underline{\wrd{3856}{wystawić na pośmiewisko,}}
                        \wrd{1014}{chciał}
                        \wrd{2977}{potajemnie}
                        \wrd{846}{ją}
                        \wrd{630}{uwolnić,}
                \bverse{20}
                        \wrd{1161}{a~}
                        \wrd{}{(gdy)}
                        \wrd{846}{on}
                        \wrd{5023}{to}
                        \wrd{1760}{obmyślił,}
                        \wrd{2400}{oto}
                        \wrd{32}{anioł}
                        \wrd{2962}{Pana}
                        \underline{\wrd{5316}{ukazał się}}
                        \wrd{846}{mu}
                        \wrd{2596}{podczas}
                        \wrd{3677}{snu,}
                        \wrd{3004}{mówiąc:}
                        \wrd{2501}{Józefie,}
                        \wrd{5207}{synu}
                        \wrd{1138}{Dawida,}
                        \wrd{3361}{nie}
                        \doubleline{\wrd{5399}{bój się}}
                        \wrd{3880}{wziąć}
                        \wrd{}{(do siebie)}
                        \wrd{3137}{Marii,}
                        \wrd{4675}{twojej}
                        \wrd{1135}{żony,}
                        \wrd{1063}{ponieważ}
                        \underline{\wrd{3588}{to, co}}
                        \wrd{1722}{w~}
                        \wrd{846}{niej,}
                        \wrd{1080}{zrodzone}
                        \wrd{2076}{jest}
                        \doubleline{\wrd{1537}{za sprawą}}
                        \wrd{4151}{Ducha}
                        \wrd{40}{Świętego;}
                \bverse{22}
                        \wrd{1161}{A~}
                        \wrd{5124}{to}
                        \wrd{3650}{wszystko}
                        \underline{\wrd{1096}{stało się,}}
                        \wrd{2443}{aby}
                        \doubleline{\wrd{4137}{wypełniło się}}
                        \underline{\wrd{3588}{to, co}}
                        \doubleline{\wrd{4483}{zostało powiedziane}}
                        \wrd{5259}{poprzez}
                        \wrd{2962}{Pana}
                        \underline{\wrd{1223}{za pośrednictwem}}
                        \wrd{4396}{proroka,}
                        \wrd{3004}{mówiącego:}
                \section{} \IndTwo
                \bverse{1}
                        \wrd{2532}{A~}
                        \underline{\wrd{1684}{gdy wszedł}}
                        \wrd{1519}{do}
                        \wrd{4143}{łodzi,}
                        \doubleline{\wrd{1276}{przeprawił się}}
                        \wrd{2532}{i~}
                        \wrd{2064}{przybył}
                        \wrd{1519}{do}
                        \wrd{2398}{swojego}
                        \wrd{4172}{miasta.}
                \bverse{2}
                        \wrd{2532}{I~}
                        \wrd{2400}{oto}
                        \wrd{4374}{przynieśli}
                        \wrd{846}{mu}
                        \wrd{3885}{sparaliżowanego,}
                        \wrd{906}{leżącego}
                        \wrd{1909}{na}
                        \wrd{2825}{łożu.}
                        \wrd{2532}{A~}
                        \wrd{2424}{Jezus,}
                        \underline{\wrd{1492}{gdy zobaczył}}
                        \wrd{846}{ich}
                        \wrd{4102}{wiarę,}
                        \wrd{2036}{powiedział}
                        \wrd{3885}{sparaliżowanemu:}
                        \wrd{2293}{Odwagi,}
                        \wrd{5043}{dziecko,}
                        \doubleline{\wrd{863}{odpuszczone są}}
                        \wrd{4671}{tobie}
                        \wrd{266}{grzechy}
                        \wrd{4675}{twoje.}
                \bverse{3}
                        \wrd{2532}{A~}
                        \wrd{2400}{oto}
                        \wrd{5100}{niektórzy}
                        \underline{\wrd{1122}{ze znawców Pisma}}
                        \wrd{2036}{pomyśleli}
                        \wrd{1722}{w~}
                        \wrd{1438}{sobie:}
                        \wrd{3778}{On}
                        \wrd{987}{znieważa.}
                \bverse{4}
                        \wrd{2532}{A~}
                        \wrd{2424}{Jezus,}
                        \underline{\wrd{1492}{gdy zobaczył}}
                        \wrd{846}{ich}
                        \wrd{1761}{myśli,}
                        \wrd{2036}{powiedział:}
                        \wrd{2443}{Dlaczego}
                        \wrd{2444}{Dlaczego}
                        \wrd{5210}{wymyślacie}
                        \wrd{1760}{wymyślacie}
                        \wrd{4190}{zło}
                        \wrd{1722}{w~}
                        \wrd{5216}{waszych}
                        \wrd{2588}{sercach?}
                \bverse{5}
                        \wrd{5101}{Co}
                        \wrd{1063}{bowiem}
                        \wrd{2076}{jest}
                        \wrd{2123}{łatwiejsze,}
                        \wrd{2036}{powiedzieć:}
                        \underline{\wrd{863}{Odpuszczone są}}
                        \wrd{4675}{twoje}
                        \wrd{266}{grzechy,}
                        \wrd{2228}{czy}
                        \wrd{2036}{powiedzieć:}
                        \wrd{1453}{Powstań}
                        \wrd{2532}{i~}
                        \wrd{4043}{chodź?}
                \bverse{6}
                        \wrd{1161}{Ale}
                        \wrd{2443}{żebyście}
                        \wrd{1492}{wiedzieli,}
                        \wrd{3754}{że}
                        \wrd{5207}{Syn}
                        \wrd{444}{Człowieczy}
                        \wrd{2192}{ma}
                        \wrd{1909}{na}
                        \wrd{1093}{ziemi}
                        \wrd{1849}{moc}
                        \wrd{863}{odpuszczania}
                        \wrd{266}{grzechów–}
                        \wrd{5119}{wtedy}
                        \wrd{3004}{powiedział}
                        \wrd{3885}{sparaliżowanemu–}
                        \wrd{1453}{Powstań,}
                        \wrd{142}{zabierz}
                        \wrd{4675}{swe}
                        \wrd{2825}{łoże}
                        \wrd{2532}{i~}
                        \wrd{5217}{odejdź}
                        \wrd{1519}{do}
                        \wrd{4675}{swojego}
                        \wrd{3624}{domu.}
                \bverse{7}
                        \wrd{2532}{A~}
                        \underline{\wrd{1453}{gdy wstał,}}
                        \wrd{565}{poszedł}
                        \wrd{1519}{do}
                        \wrd{846}{swego}
                        \wrd{3624}{domu.}
                \bverse{8}
                        \wrd{1161}{A~}
                        \wrd{3793}{tłumy,}
                        \underline{\wrd{1492}{gdy}}
                        \wrd{}{(to)}
                        \underline{\wrd{1492}{zobaczyły,}}
                        \doubleline{\wrd{2296}{dziwiły się}}
                        \wrd{2532}{i~}
                        \underline{\wrd{1392}{oddały chwałę}}
                        \wrd{2316}{Bogu,}
                        \doubleline{\wrd{1325}{że dał}}
                        \wrd{444}{ludziom}
                        \wrd{5108}{taką}
                        \wrd{1849}{moc.}
                \bverse{9}
                        \wrd{2532}{A~}
                        \wrd{3855}{odchodząc}
                        \wrd{1564}{stamtąd,}
                        \wrd{2424}{Jezus}
                        \wrd{1492}{dostrzegł}
                        \wrd{444}{człowieka,}
                        \wrd{2521}{siedzącego}
                        \wrd{1909}{w~}
                        \underline{\wrd{5058}{miejscu pobierania podatków,}}
                        \doubleline{\wrd{3004}{którego nazywano}}
                        \wrd{3156}{Mateuszem,}
                        \wrd{2532}{i~}
                        \wrd{3004}{powiedział}
                        \wrd{846}{mu:}
                        \underline{\wrd{190}{Idź za}}
                        \wrd{3427}{mną.}
                        \wrd{2532}{A~}
                        \doubleline{\wrd{450}{gdy wstał,}}
                        \underline{\wrd{190}{zaczął iść za}}
                        \wrd{846}{nim.}
                \bverse{10}
                        \wrd{2532}{A~}
                        \underline{\wrd{846}{gdy on}}
                        \doubleline{\wrd{345}{leżał przy stole}}
                        \wrd{1722}{w~}
                        \wrd{3614}{domu,}
                        \underline{\wrd{1096}{stało się,}}
                        \wrd{2532}{a~}
                        \wrd{2400}{oto}
                        \wrd{4183}{liczni}
                        \doubleline{\wrd{5057}{poborcy podatków}}
                        \wrd{2532}{i~}
                        \wrd{268}{grzesznicy,}
                        \wrd{2064}{przyszedłszy,}
                        \underline{\wrd{4873}{leżeli przy stole}}
                        \doubleline{\wrd{2424}{z Jezusem}}
                        \wrd{2532}{oraz}
                        \wrd{846}{jego}
                        \wrd{3101}{uczniami.}
                \bverse{11}
                        \wrd{2532}{A~}
                        \underline{\wrd{1492}{gdy zobaczyli}}
                        \wrd{}{(to)}
                        \wrd{5330}{faryzeusze,}
                        \wrd{2036}{mówili}
                        \wrd{846}{jego}
                        \wrd{3101}{uczniom:}
                        \wrd{1223}{Dlaczego}
                        \wrd{5101}{Dlaczego}
                        \wrd{5216}{wasz}
                        \wrd{1320}{nauczyciel}
                        \wrd{2068}{je}
                        \wrd{3326}{z~}
                        \doubleline{\wrd{5057}{poborcami podatków}}
                        \wrd{2532}{i~}
                        \wrd{268}{grzesznikami?}
                \bverse{12}
                        \wrd{1161}{A~}
                        \wrd{2424}{Jezus,}
                        \underline{\wrd{191}{gdy}}
                        \wrd{}{(to)}
                        \underline{\wrd{191}{usłyszał,}}
                        \wrd{2036}{powiedział}
                        \wrd{846}{im:}
                        \wrd{3756}{Nie}
                        \wrd{2480}{zdrowi}
                        \wrd{5532}{potrzebują}
                        \wrd{2192}{potrzebują}
                        \wrd{2395}{lekarza,}
                        \wrd{235}{ale}
                        \doubleline{\wrd{2192}{ci, co się}}
                        \wrd{2560}{źle}
                        \doubleline{\wrd{2192}{mają.}}
\end{document} 