% !TEX TS-program = XeLaTeX

\documentclass{article}

\usepackage{fontspec}
\usepackage[polish]{babel}
\usepackage{lettrine}
\usepackage[pagestyles]{titlesec}
\setmainfont{Palatino Linotype}
\setsansfont{Arial}
\newfontfamily\versefont{Tahoma}
\selectlanguage{polish}
\font\strong=phvr at 6pt

\titleformat{\section}
{}{\lettrine{\thesection}}{5em}{}[\vskip-0.75\baselineskip]
\titlespacing{\section}{1em}{5pt}{1pt}

\usepackage{expex}
\lingset{
	everygla=\strong, % formatting the numbers in the gla line
    glwordalign=center, % center alignment within glwords
    aboveglbskip=-0.5ex, % narrows the vertical gap between the gla and glb lines
    glhangstyle=none, % no hanging indent
    glspace=!0pt plus .3em, % widens the allowable space between glwords to avoid overfull lines
    glrightskip=0pt plus .3\hsize} % widens the allowable space between the right margin and the end of the last glword on a line to avoid overfull lines

\newcommand\doubleline[1]{\underline{{\underline{#1}}}}
\newcommand{\vs}[1]{{\bfseries\versefont #1}}

\begin{document}
\exdisplay 
%\section{}
%\lettrine[lines=4,lraise=0,findent=0.5em]{1}{} 
\begingl
\gla {} 4250 2228 4905 846 2147 1111 1111 1111 1111 1111 1111 1111 1111 {} 11 11 12//
\glb \vs{18} \underline{wcześniej, zanim} [-] się oni znalazła się \underline{tą, która ma} (dziecko) w łonie \underline{za sprawą} \doubleline{Ducha Świętego} \vs{19} a Jozef  \doubleline{jej mąż ktory był}, \doubleline{nie chcąc wystawić jej na pośmiewisko} //
\endgl
\section{}
\begingl
\gla {} 4250 2228 4905 846 2147 1111 1111 1111 1111 1111 1111 1111 1111 {} 11 11 12//
\glb \vs{18} \underline{wcześniej, zanim} [-] się oni znalazła się \underline{tą, która ma} (dziecko) w łonie \underline{za sprawą} \doubleline{Ducha Świętego} \vs{19} a Jozef  \doubleline{jej mąż ktory był}, \doubleline{nie chcąc wystawić jej na pośmiewisko} //
\endgl
\xe
\end{document}