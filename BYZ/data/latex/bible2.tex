\documentclass[paper=a5,pagesize=pdftex]{scrbook}

\usepackage[utf8]{inputenc}
\usepackage[T1]{fontenc}
\usepackage{polski}
\usepackage[pagestyles]{titlesec}
\usepackage{lettrine}
\usepackage{setspace}
\usepackage[polish]{babel}
\usepackage{lipsum}
\usepackage[usenames,dvipsnames,svgnames,table]{xcolor}
\selectlanguage{polish}

% fonts
\font\strong=phvr at 6pt

\renewcommand{\thesection}{\arabic{section}}
%\renewcommand{\thesubsection}{\arabic{subsection}}

\newcommand{\subsection}[1]{%
  \pagebreak[2]
  \refstepcounter{subsection}
  \addcontentsline{toc}{subsection}{
    {\protect\makebox[0.3in][r]{\thesubsection.} \hspace*{3pt}#1}}
  \noindent
  \textbf{\thesubsection\space\space{#1}. }
}

% definition of the page style with required headers
\newpagestyle{Biblestyle}{
  \setheadrule{.009pt}
  \sethead[\thepage][\chaptertitle]
    [\toptitlemarks\thesection:\toptitlemarks\thesubsection---%
      \bottitlemarks\thesection:\bottitlemarks\thesubsection]%
    {\toptitlemarks\thesection:\toptitlemarks\thesubsection---%
      \bottitlemarks\thesection:\bottitlemarks\thesubsection}{\chaptertitle}{\thepage}
}

% sets the marks to be used (section and subsection)
\setmarks{section}{subsection}

% sections and subsections formatting
\titleformat{\section}
{}{\lettrine{\thesection}}{0em}{}[\vskip-0.65\baselineskip]
\titleformat{\subsection}[runin]{\small\bfseries}{\thesubsection}{1em}{}
\titlespacing{\section}{1em}{-1pt}{0pt}
%\titlespacing{\subsection}{0pt}{0pt}{1em}
\titlespacing{\subsection}{0pt}{0pt}{\wordsep}

\pagestyle{Biblestyle}
\renewcommand{\LettrineFontHook}{\bfseries}
\setlength{\parindent}{0pt}

\newlength\NumLen
\newlength\LinLen
% indents one line of text. Indentation= width of section number + 1em
\newcommand\IndOne{%
  \settowidth\NumLen{\thesection}
  \addtolength\NumLen{0.8em}
  \setlength\LinLen{\dimexpr\textwidth-\NumLen}%\the\NumLen\the\LinLen
  \parshape 2 \NumLen \LinLen 0pt \textwidth}
  
% indents two lines of text. Indentation= width of section number + 1em
\newcommand\IndTwo{%
  \settowidth\NumLen{\thesection}
  \addtolength\NumLen{1em}
  \setlength\LinLen{\dimexpr\textwidth-\NumLen}%\the\NumLen\the\LinLen
  \parshape 3 \NumLen \LinLen \NumLen \LinLen 0pt \textwidth}
 
% macros
\newcommand\wrd[2]{%
  \leavevmode
  \vbox{\offinterlineskip
    \halign{%
      \hfil##\hfil\cr
      {\strong\vphantom{p}#1}\cr
      \noalign{\vskip\lineskip}%
      \vphantom{A}#2\cr
    }%
  }%
}
 
\begin{document}
\doublespacing
\section{}
\subsection{}
\IndTwo
\wrd{976}{Zwój księgi}
\wrd{}{(o)}
\wrd{1078}{narodzinach}
\wrd{2424}{Jezusa}
\wrd{5547}{syna}
\wrd{5547}{Dawida,}
\wrd{5547}{syna}
\wrd{11}{Abrahama.}
\subsection{}
\wrd{11}{Abraham}
\wrd{1080}{zrodził}
\wrd{2464}{Izaaka,}
\wrd{1161}{a}
\wrd{2464}{Izaak}
\wrd{1080}{zrodził}
\wrd{2384}{Jakuba,}
\wrd{1161}{a}
\wrd{2384}{Jakub}
\wrd{1080}{zrodził}
\wrd{2455}{Judę}
\wrd{2532}{i}
\wrd{80}{braci}
\wrd{846}{jego}


\section{}

\IndTwo
\underline{\underline{\wrd{976}{Zwój księgi}}}
\wrd{}{(o)}
\wrd{1078}{narodzinach}
\wrd{2424}{Jezusa}
\wrd{5547}{syna}
\wrd{5547}{Dawida,}
\wrd{5547}{syna}
\wrd{11}{Abrahama.}
\subsection{}
\wrd{11}{Abraham}
\wrd{1080}{zrodził}
\wrd{2464}{Izaaka,}
\wrd{1161}{a}
\wrd{2464}{Izaak}
\wrd{1080}{zrodził}
\wrd{2384}{Jakuba,}
\wrd{1161}{a}
\wrd{2384}{Jakub}
\wrd{1080}{zrodził}
\wrd{2455}{Judę}
\wrd{2532}{i}
\wrd{80}{braci}
\wrd{846}{jego}
\subsection{}
\wrd{11}{Abraham}
\wrd{1080}{zrodził}
\wrd{2464}{Izaaka,}
\wrd{1161}{a}
\wrd{2464}{Izaak}
\wrd{1080}{zrodził}
\wrd{2384}{Jakuba,}
\wrd{1161}{a}
\wrd{2384}{Jakub}
\wrd{1080}{zrodził}
\wrd{2455}{Judę}
\wrd{2532}{i}
\wrd{80}{braci}
\wrd{846}{jego}

\end{document} 