\renewcommand{\chaptertitle}{}
\begingl
\lettrine[loversize=1,lraise=-1.3]{1 }{}%
\gla
 976 {} 1078 2424 5547 5207 1138 5207 11
{} 11 1080 2464 1161 2464 1080 2384 1161 2384 1080 2455 2532 80 846
{} 1161 2455 1080 5329 2532 2196 1537 2283 1161 5329 1080 2074 1161 2074 1080 689
{} 1161 689 1080 284 1161 284 1080 3476 1161 3476 1080 4533
{} 1161 4533 1080 1003 1537 4477 1161 1003 1080 5601 1537 4503 1161 5601 1080 2421
{} 1161 2421 1080 1138 935 1161 1138 935 1080 4672 1537 3588 {} 3774
{} 1161 4672 1080 4497 1161 4497 1080 7 1161 7 1080 760
{} 1161 760 1080 2498 1161 2498 1080 2496 1161 2496 1080 3604
{} 1161 3604 1080 2488 1161 2488 1080 881 1161 881 1080 1478
{} 1161 1478 1080 3128 1161 3128 1080 300 1161 300 1080 2502
{} 1161 2502 1080 2423 2532 80 846 1909 3350 {} 897
{} 1161 3326 3350 {} 897 2423 1080 4528 1161 4528 1080 2216
{} 1161 2216 1080 10 1161 10 1080 1662 1161 1662 1080 107
{} 1161 107 1080 4524 1161 4524 1080 885 1161 885 1080 1664
{} 1161 1664 1080 1648 1161 1648 1080 3157 1161 3157 1080 2384
{} 1161 2384 1080 2501 435 3137 1537 3739 1080 2424 3004 5547
{} 3956 3767 1074 575 11 2193 1138 1074 1180 2532 575 1138 2193 3350 {} 897 1074 1180 2532 575 3350 {} 897 2193 5547 1074 1180
{} 1161 1083 2424 5547 2258 3779 1063 3384 846 3137 3423 2501 4250 2228 4905 846 4905 2147 2192 {} 1722 1064 1537 4151 40
{} 1161 2501 846 435 5607 1342 2532 3361 2309 846 3856 1014 2977 846 630
{} 1161 {} 846 5023 1760 2400 32 2962 5316 846 2596 3677 3004 2501 5207 1138 3361 5399 3880 {} 3137 4675 1135 1063 3588 1722 846 1080 2076 1537 4151 40
{} 1161 5088 5207 2532 2564 846 3686 2424 1063 846 4982 2992 846 575 846 266
{} 1161 5124 3650 1096 2443 4137 3588 4483 5259 2962 1223 4396 3004
{} 2400 3933 2192 1722 1064 2532 5088 5207 2532 2564 846 3686 1694 3739 2076 3177 3326 2257 2316
{} 1161 2501 1326 575 5258 4160 5613 4367 846 32 2962 2532 3880 846 1135
{} 2532 3756 1097 846 2193 3739 5088 5207 846 4416 2532 2564 846 3686 2424
//
\glb
 \underline{Zwój księgi} (o) narodzinach Jezusa Chrystusa, syna Dawida, syna Abrahama.
\vs{2} Abraham zrodził Izaaka, a~ Izaak zrodził Jakuba, a~ Jakub zrodził Judę i~ braci jego,
\vs{3} a~ Juda zrodził Faresa i~ Zarę z~ Tamary, a~ Fares zrodził Esroma, a~ Esrom zrodził Arama,
\vs{4} a~ Aram zrodził Aminadaba, a~ Aminadab zrodził Naassona, a~ Naasson zrodził Salmona,
\vs{5} a~ Salmon zrodził Booza z~ Rahab, a~ Booz zrodził Jobeda z~ Rut, a~ Jobed zrodził Jesaja,
\vs{6} a~ Jesaj zrodził Dawida– króla, a~ Dawid– król zrodził Salomona z~ tej, (która należała do) Uriasza,
\vs{7} a~ Salomon zrodził Roboama, a~ Roboam zrodził Abiasa, a~ Abias zrodził Asafa,
\vs{8} a~ Asaf zrodził Jozafata, a~ Jozafat zrodził Jorama, a~ Joram zrodził Ozjasza,
\vs{9} a~ Ozjasz zrodził Joatama, a~ Joatam zrodził Achaza, a~ Achaz zrodził Ezechiasza,
\vs{10} a~ Ezechiasz zrodził Manassesa, a~ Manasses zrodził Amona, a~ Amon zrodził Jozjasza,
\vs{11} a~ Jozjasz zrodził Jechoniasza oraz braci jego \underline{w czasie} przesiedlenia (do) Babilonu,
\vs{12} a~ po przesiedleniu (do) Babilonu Jechoniasz zrodził Salatiela, a~ Salatiel zrodził Zorobabela,
\vs{13} a~ Zorobabel zrodził Abiuda, a~ Abiud zrodził Eliakima, a~ Eliakim zrodził Azora,
\vs{14} a~ Azor zrodził Sadoka, a~ Sadok zrodził Achima, a~ Achim zrodził Eliuda,
\vs{15} a~ Eliud zrodził Eleazara, a~ Eleazar zrodził Mattana, a~ Mattan zrodził Jakuba,
\vs{16} a~ Jakub zrodził Józefa, męża Marii, z~ której \underline{został zrodzony} Jezus, nazywany Chrystusem.
\vs{17} Wszystkich więc pokoleń od Abrahama do Dawida– pokoleń czternaście, i~ od Dawida do przesiedlenia (do) Babilonu– pokoleń czternaście, i~ od przesiedlenia (do) Babilonu do Chrystusa– pokoleń czternaście.
\vs{18} A~ narodzenie Jezusa Chrystusa było takie: ponieważ matka jego, Maria, \underline{będąc zaślubiona} Józefowi, \doubleline{wcześniej, zanim} [-] \underline{się} oni \underline{zeszli,} \doubleline{znalazła się} \underline{tą, która ma} (dziecko) w~ łonie \doubleline{za sprawą} Ducha Świętego,
\vs{19} a~ Józef, jej mąż, będąc sprawiedliwym i~ nie chcąc jej \underline{wystawić na pośmiewisko,} chciał potajemnie ją uwolnić,
\vs{20} a~ (gdy) on to obmyślił, oto anioł Pana \underline{ukazał się} mu podczas snu, mówiąc: Józefie, synu Dawida, nie \doubleline{bój się} wziąć (do siebie) Marii, twojej żony, ponieważ \underline{to, co} w~ niej, zrodzone jest \doubleline{za sprawą} Ducha Świętego;
\vs{21} i~ urodzi syna, i~ nazwiesz go imieniem Jezus, ponieważ On wybawi lud swój od ich grzechów.
\vs{22} A~ to wszystko \underline{stało się,} aby \doubleline{wypełniło się} \underline{to, co} \doubleline{zostało powiedziane} poprzez Pana \underline{za pośrednictwem} proroka, mówiącego:
\vs{23} Oto panna \underline{„będzie mieć} w~ łonie” i~ urodzi syna, i~ nazwą go imieniem Emmanuel, co jest tłumaczone: \doubleline{razem z} nami Bóg.
\vs{24} A~ Józef, \underline{obudziwszy się} ze snu, uczynił, jak nakazał mu anioł Pana, i~ przyjął swoją żonę,
\vs{25} i~ nie poznawał jej, aż [-] urodziła syna swego pierworodnego; i~ nazwał go imieniem Jezus.
//
\endgl
\begingl
\lettrine[loversize=1,lraise=-1.3]{2 }{}%
\gla
 1161 1080 2424 1080 1722 965 2449 1722 2250 2264 935 2400 3097 575 395 3854 1519 2414
{} 3004 4226 2076 {} 5088 935 2453 1492 1063 846 792 1722 395 2532 2064 4352 846 4352
{} 1161 191 {} 191 935 2264 5015 2532 3326 846 3956 2414
{} 2532 4863 3956 749 2532 1122 {} 2992 4441 3844 846 4226 1080 5547
{} 1161 3588 2036 846 1722 965 2449 3779 1063 1125 1223 4396
{} 2532 4771 965 1093 2448 3760 1488 1646 1722 2232 2448 1063 1537 4675 1831 2233 3748 4165 2992 3450 2474
{} 5119 2264 2977 2564 3097 198 3844 846 {} 5550 5316 792
{} 2532 3992 846 1519 965 2036 4198 199 1833 4012 3813 1161 1875 {} 2147 518 3427 3704 2504 2064 {} 4352 846 4352
{} 1161 3588 191 935 4198 2532 2400 792 3739 1492 1722 395 4254 846 2193 2064 {} 2476 1883 {} 3757 2258 3813
{} 1161 1492 792 5463 5479 4970 3173
{} 2532 2064 1519 3614 3708 3813 3326 3137 846 3384 2532 4098 4352 846 4352 2532 455 846 2344 4374 846 1435 5557 2532 3030 2532 4666
{} 2532 5537 2596 3677 5537 {} 3361 344 4314 2264 1223 243 3598 402 1519 846 5561
{} 1161 402 846 402 2400 32 2962 5316 2596 3677 2501 3004 1453 3880 3813 2532 846 3384 2532 5343 1519 125 2532 2468 1563 2193 302 4671 2036 3195 1063 2264 2212 3813 {} 846 622
{} 1161 3588 1453 3880 3571 3588 3813 2532 846 3384 2532 402 1519 125
{} 2532 2258 1563 2193 5054 {} 2264 2443 4137 3588 4483 5259 2962 1223 4396 3004 1537 125 2564 5207 3450 {}
{} 5119 2264 1492 3754 1702 5259 3097 3029 2373 2532 649 {} 337 3956 3816 1722 965 2532 1722 3956 846 3725 575 1332 {} 2532 2736 {} 2596 5550 {} 3739 198 3844 3097
{} 5119 4137 {} 4483 5259 4396 2408 3004
{} 1722 4471 191 5456 2355 2532 2805 2532 4183 3602 4478 2799 846 5043 2532 3756 2309 3870 3754 {} 3756 1526 {}
{} 1161 5053 2264 5053 {} 2400 32 2962 2596 3677 5316 2501 1722 125
{} 3004 1453 3880 3813 2532 846 3384 2532 4198 1519 1093 2474 2348 1063 3588 2212 5590 3813
{} 1161 3588 1453 3880 3813 2532 846 3384 2532 2064 1519 1093 2474
{} 1161 191 3754 745 936 1909 2449 473 2264 846 3962 5399 1563 565 1161 5537 2596 3677 402 1519 3313 1056
{} 2532 2064 {} 2064 2730 1519 4172 3004 3478 3704 4137 3588 4483 1223 4396 3754 2564 3480
//
\glb
 A~ \underline{kiedy} Jezus \underline{urodził się} w~ Betlejem, \doubleline{w Judei,} za dni Heroda– króla, oto magowie ze wschodu przybyli do Jerozolimy,
\vs{2} mówiąc: Gdzie \underline{się znajduje} (ten) narodzony król Judejczyków? Zobaczyliśmy bowiem jego gwiazdę na wschodzie i~ przyszliśmy \doubleline{oddać} mu \doubleline{pokłon.}
\vs{3} A~ \underline{gdy} (to) \underline{usłyszał} król Herod, \doubleline{przestraszył się,} a~ \underline{razem z} nim cała Jerozolima.
\vs{4} I~ zebrawszy wszystkich arcykapłanów i~ \underline{znawców Pisma} (spośród) ludu, \doubleline{dowiadywał się} od nich, gdzie \underline{się rodzi} Mesjasz.
\vs{5} A~ oni powiedzieli mu: w~ Betlejem, \underline{w Judei,} tak bowiem \doubleline{jest napisane} przez proroka:
\vs{6} I~ ty, Betlejem, ziemio Judy, \underline{wcale nie} jesteś najmniejsze spośród rządców Judy, ponieważ z~ ciebie wyjdzie rządzący, który \doubleline{paść będzie} lud mój– Izraela.
\vs{7} Wtedy Herod, potajemnie wezwawszy magów, \underline{dokładnie dowiedział się} od nich (o) czasie \doubleline{ukazującej się} gwiazdy.
\vs{8} I~ wysławszy ich do Betlejem, powiedział: dotrzyjcie, dokładnie wypytajcie o~ dziecko. A~ gdy (je) znajdziecie, oznajmijcie mi, żebym \underline{i ja} przyszedł (i) \doubleline{oddał} mu \doubleline{pokłon.}
\vs{9} A~ oni, \underline{gdy wysłuchali} króla, wyruszyli; a~ oto gwiazda, którą dostrzegli na wschodzie, wyprzedzała ich, aż przybyła (i) stanęła ponad (miejscem), gdzie było dziecko.
\vs{10} A~ \underline{gdy dostrzegli} gwiazdę, \doubleline{uradowali się} radością bardzo wielką.
\vs{11} I~ \underline{gdy przyszli} do domu, zobaczyli dziecko, \doubleline{razem z} Marią, jego matką, i~ upadłszy, \underline{oddali} mu \underline{pokłon,} a~ otworzywszy swoje skarby, ofiarowali mu dary: złoto i~ kadzidło, i~ mirrę.
\vs{12} A~ \underline{gdy} podczas snu \underline{otrzymali pouczenie,} (aby) nie powracać do Heroda, poprzez inną drogę odeszli na swoje terytorium.
\vs{13} A~ \underline{gdy} oni \underline{odeszli,} oto anioł Pana \doubleline{ukazał się} podczas snu Józefowi, mówiąc: powstań, weź dziecko i~ jego matkę, i~ uciekaj do Egiptu, i~ bądź tam, aż [-] ci powiem. Zamierza bowiem Herod odszukać dziecko, (żeby) je stracić.
\vs{14} A~ on, \underline{gdy powstał,} wziął nocą to dziecko oraz jego matkę i~ odszedł do Egiptu.
\vs{15} I~ był tam \underline{aż do} końca (życia) Heroda, aby \doubleline{wypełniło się} \underline{to, co} \doubleline{zostało powiedziane} przez Pana \underline{za pośrednictwem} proroka, mówiącego: z~ Egiptu wywołałem syna mego. (Oz 11:1)
\vs{16} Wtedy Herod, \underline{gdy dostrzegł,} że \doubleline{został wyszydzony} przez magów, bardzo \underline{się rozgniewał} i~ posławszy (swoich ludzi), zgładził wszystkich chłopców w~ Betlejem oraz we wszelkich jego granicach, od dwulatków (począwszy) oraz poniżej (tego wieku), \doubleline{zgodnie z} czasem, (o) którym \underline{dokładnie się dowiedział} od magów.
\vs{17} Wtedy \underline{wypełniło się} (to, co) \doubleline{zostało powiedziane} przez proroka Jeremiasza, mówiącego:
\vs{18} W~ Rama \underline{był słyszany} głos: lament i~ płacz, i~ wielkie narzekanie; Rachel opłakuje swoje dzieci i~ nie chce \doubleline{zostać pocieszona,} ponieważ (ich już) nie ma. (Jr 31:15)
\vs{19} A~ \underline{gdy} Herod \underline{doszedł do końca} (życia), oto anioł Pana poprzez sen \doubleline{ukazał się} Józefowi w~ Egipcie,
\vs{20} mówiąc: Powstań, weź dziecko i~ jego matkę i~ wyrusz do ziemi Izraela. Umarli bowiem \underline{ci, którzy} szukają duszy dziecka.
\vs{21} A~ On, \underline{gdy powstał,} wziął dziecko i~ jego matkę, i~ przybył do ziemi Izraela.
\vs{22} Ale \underline{gdy usłyszał,} że Archelaos \doubleline{jest królem} w~ Judei– zamiast Heroda, swojego ojca– \underline{przestraszył się} tam wejść; a~ \doubleline{otrzymawszy pouczenie} podczas snu, odszedł ku obszarom Galilei.
\vs{23} A~ \underline{gdy} (tam) \underline{przybył,} zamieszkał w~ mieście, zwanym Nazaret, aby \doubleline{wypełniło się} \underline{to, co} \doubleline{zostało powiedziane} \underline{za pośrednictwem} proroków, że \doubleline{będzie nazwany} Nazarejczykiem.
//
\endgl
\begingl
\lettrine[loversize=1,lraise=-1.3]{3 }{}%
\gla
 1161 1722 1565 2250 3854 2491 910 2784 1722 2048 2449
{} 2532 3004 3340 1063 1448 932 3772
{} 2076 3778 1063 4483 5259 4396 2268 3004 5456 994 1722 2048 2090 3598 2962 2117 4160 846 5147
{} 846 1161 2491 2192 846 1742 575 2359 2574 2532 1193 2223 4012 846 3751 1161 5160 846 2258 200 2532 66 3192
{} 5119 1607 4314 846 2414 2532 3956 2449 2532 3956 4066 2446
{} 2532 907 5259 846 1722 2446 1843 846 266
{} 1161 1492 4183 5330 2532 4523 2064 1909 {} 846 908 2036 846 1081 2191 5101 5213 5263 5343 575 3195 3709
{} 4160 3767 2590 514 3341
{} 2532 3361 1380 {} 3004 1722 1438 2192 3962 11 1063 3004 5213 3754 2316 1410 1537 5130 3037 1453 5043 11
{} 1161 2235 2532 513 4314 4491 1186 2749 3956 3767 1186 3361 4160 2570 2590 1581 2532 906 1519 4442
{} 1473 3303 5209 907 1722 5204 1519 3341 1161 3588 2064 3694 3450 2076 2478 3450 3756 1510 2425 {} 3739 5266 941 846 5209 907 1722 4151 40
{} 3739 4425 1722 846 5495 2532 1245 846 257 2532 4863 846 4621 1519 596 1161 892 2618 762 4442 762
{} 5119 3854 2424 575 1056 1909 2446 4314 2491 907 5259 846 907
{} 1161 2491 1254 846 3004 1473 2192 5532 {} 907 5259 4675 2532 4771 2064 4314 3165
{} 1161 2424 611 2036 4314 846 863 737 3779 1063 4241 2076 4137 2254 3956 1343 5119 846 863
{} 2424 2532 907 2117 305 575 5204 2532 2400 455 846 3772 2532 1492 4151 2316 2597 5616 4058 2532 2064 1909 846
{} 2532 2400 5456 1537 3772 3004 3778 2076 3450 27 5207 1722 3739 2106
//
\glb
 A~ w~ owe dni przybył Jan Chrzciciel, \underline{który głosił} na pustkowiu Judei
\vs{2} i~ mówił: \underline{zmieniajcie myślenie,} ponieważ \doubleline{zbliżyło się} królestwo niebios.
\vs{3} Jest on bowiem przepowiedziany przez proroka Izajasza, \underline{który mówi:} Głos wołającego na pustkowiu: przygotujcie drogę Pana; prostymi czyńcie Jego ścieżki.
\vs{4} Sam zaś Jan miał swe odzienie z~ sierści wielbłąda oraz skórzany pas wokół swoich bioder, a~ pokarmem jego była szarańcza i~ dziki miód.
\vs{5} Wtedy wychodziła do niego Jerozolima i~ cała Judea, i~ cała \underline{sąsiednia okolica} Jordanu;
\vs{6} i~ \underline{byli zanurzani} przez niego w~ Jordanie– \doubleline{ci, którzy wyznawali} swoje grzechy.
\vs{7} A~ \underline{gdy dostrzegł} licznych faryzeuszów i~ saduceuszów, przychodzących do (dokonywanego przez) niego zanurzenia, powiedział im: potomstwo żmij, kto wam pokazał, \doubleline{jak uciec} od \underline{mającego nadejść} gniewu?
\vs{8} Zrodźcie zatem owoc, godny \underline{zmiany myślenia.}
\vs{9} I~ nie \underline{sądźcie, że słuszne} (jest) mówić wobec siebie: mamy ojca– Abrahama, gdyż mówię wam, że Bóg \doubleline{jest w stanie} z~ tych kamieni wzbudzić dzieci Abrahamowi.
\vs{10} Ale \underline{już właśnie} i~ siekiera do korzenia drzew \doubleline{jest przystawiona.} Każde więc drzewo, \underline{które nie} rodzi dobrego owocu, \doubleline{zostaje wycięte} i~ wrzucane w~ ogień.
\vs{11} Ja [-] was zanurzam w~ wodzie ku \underline{zmianie myślenia,} ale \doubleline{ten, który} przychodzi za mną, jest potężniejszy \underline{ode mnie;} nie jestem \doubleline{dostatecznie ważny,} (by) mu sandały podnieść. On was zanurzy w~ Duchu Świętym
\vs{12} którego \underline{szufla do odwiewania} \doubleline{jest w} jego ręku; i~ oczyści swoje klepisko, i~ zbierze swą pszenicę do magazynu, zaś plewy spali \underline{w~} ogniu, \underline{niemożliwym do ugaszenia.}
\vs{13} Wtedy przybył Jezus z~ Galilei nad Jordan do Jana, \underline{by dać się} przez niego \underline{zanurzyć.}
\vs{14} Zaś Jan powstrzymywał go, mówiąc: ja mam potrzebę, (by) \underline{być zanurzonym} przez ciebie, a~ ty przychodzisz do mnie?
\vs{15} A~ Jezus, odpowiadając, powiedział do niego: dopuść teraz, tak bowiem stosowne jest wypełnić nam wszelką sprawiedliwość. Wtedy go dopuścił.
\vs{16} Jezus zaś, \underline{gdy został zanurzony,} natychmiast wyszedł z~ wody. I~ oto \doubleline{otwarte zostały} mu niebiosa i~ dostrzegł ducha Boga, zstępującego \underline{jak gdyby} gołąb i~ przychodzącego na niego.
\vs{17} I~ oto głos z~ nieba mówił: Ten jest moim umiłowanym synem, w~ którym \underline{mam upodobanie.}
//
\endgl
\begingl
\lettrine[loversize=1,lraise=-1.3]{4 }{}%
\gla
 5119 2424 321 1519 2048 5259 4151 {} 3985 5259 1228
{} 2532 3522 5062 2250 2532 5062 3571 5305 3983
{} 2532 4334 3985 4334 846 2036 1487 1488 5207 2316 2036 2443 3778 3037 1096 740
{} 1161 3588 611 2036 1125 3756 1909 3441 740 444 2198 235 1909 3956 4487 1607 1223 4750 2316
{} 5119 3880 846 1228 1519 40 4172 2532 2476 846 1909 4419 2411
{} 2532 3004 846 1487 1488 5207 2316 906 4572 2736 1125 1063 1125 3754 846 32 1781 4012 4675 2532 1909 5495 4571 142 3379 4350 4675 4228 4314 3037
{} 2424 846 5346 1125 3825 1125 3756 1598 2962 4675 2316
{} 3880 846 3825 1228 1519 3029 5308 3735 2532 1166 846 3956 932 2889 2532 846 1391
{} 2532 3004 846 1325 5023 3956 4671 1437 4098 {} 4352 3427 4352
{} 5119 2424 3004 846 5217 3694 3450 4567 1063 1125 2962 4675 2316 4352 2532 846 3441 3000
{} 5119 863 846 1228 2532 2400 32 4334 2532 1247 846
{} 1161 191 2424 191 3754 2491 3860 402 1519 1056
{} 2532 2641 3478 2064 {} 2730 1519 2584 3864 1722 3725 2194 2532 3508
{} 2443 4137 3588 4483 1223 4396 2268 3004
{} 1093 2194 2532 1093 3508 3598 2281 4008 2446 1056 1484
{} 2992 2521 1722 4655 3708 3173 5457 2532 846 2521 1722 5561 2532 4639 2288 393 5457
{} 575 5119 2424 756 2784 2532 3004 3340 1063 1448 932 3772
{} 4043 1161 3844 2281 1056 1492 1417 80 4613 3004 4074 2532 406 846 80 906 293 1519 2281 2258 1063 231
{} 2532 3004 846 1205 3694 3450 2532 4160 5209 231 444
{} 1161 3588 2112 863 1350 190 846
{} 2532 4260 1564 4260 1492 1417 243 80 2385 {} 3588 2199 2532 2491 846 80 {} 1722 4143 3326 3962 846 2199 2675 846 1350 2532 2564 846
{} 3588 1161 2112 863 4143 2532 3962 846 190 846
{} 2532 4013 2424 3650 1056 1321 1722 846 4864 2532 2784 2098 932 2532 2323 3956 3554 2532 3956 3119 1722 2992
{} 2532 565 189 846 1519 3650 4947 2532 4374 846 3956 2560 2192 4912 4164 3554 2532 931 2532 1139 2532 4583 2532 3885 2532 2323 846
{} 2532 190 846 4183 3793 575 1056 2532 1179 2532 2414 2532 2449 2532 4008 2446
//
\glb
 Wtedy Jezus \underline{został wyprowadzony} na pustkowie przez ducha, (by) \doubleline{poddawanym próbie} przez diabła.
\vs{2} I~ \underline{gdy przepościł} czterdzieści dni i~ czterdzieści nocy, wreszcie \doubleline{odczuł głód.}
\vs{3} I~ \underline{gdy} \doubleline{poddający próbie} \underline{podszedł} \underline{do niego,} powiedział: jeżeli jesteś synem Boga powiedz, aby te kamienie \doubleline{stały się} chlebami.
\vs{4} A~ on, odpowiadając, powiedział: \underline{Jest napisane:} nie \doubleline{z powodu} samego chleba człowiek \underline{będzie żył,} lecz \doubleline{z powodu} każdego słowa, wychodzącego przez usta Boga.
\vs{5} Wtedy wziął go diabeł do świętego miasta i~ postawił go na skraju świątyni
\vs{6} i~ mówił mu: jeśli jesteś synem Boga, zrzuć \underline{się sam} \doubleline{na dół,} \underline{jest} bowiem \underline{napisane,} że swym aniołom rozkaże odnośnie ciebie; i~ na rękach cię uniosą, \doubleline{abyś nie} uderzył swą nogą o~ kamień.
\vs{7} Jezus mu powiedział: \underline{jest} także \underline{napisane:} nie \doubleline{będziesz wystawiał na próbę} Pana, twego Boga.
\vs{8} Wziął go znów diabeł na bardzo wysoką górę i~ pokazał mu wszystkie królestwa świata oraz ich chwałę.
\vs{9} I~ powiedział mu: dam to wszystko tobie, jeśli upadniesz (i) \underline{oddasz} mi \underline{pokłon.}
\vs{10} Wtedy Jezus powiedział mu: odejdź ode mnie, szatanie, ponieważ \underline{jest napisane:} Panu, twemu Bogu, \doubleline{będziesz oddawał pokłon} i~ jemu samemu \underline{będziesz służył.}
\vs{11} Wówczas opuścił go diabeł, a~ oto aniołowie podeszli i~ usługiwali mu.
\vs{12} A~ \underline{gdy} Jezus \underline{usłyszał,} że Jan \doubleline{został wydany,} odszedł ku Galilei.
\vs{13} I~ opuściwszy Nazaret, przybył (by) zamieszkać w~ Kafarnaum nadmorskim, w~ granicach Zabulona i~ Neftalego,
\vs{14} aby \underline{wypełniło się} \doubleline{to, co} \underline{zostało powiedziane} \doubleline{za pośrednictwem} proroka Izajasza, mówiącego:
\vs{15} Ziemia Zabulona i~ ziemia Neftalego, droga morska \underline{po drugiej stronie} Jordanu, Galilea narodów.
\vs{16} Lud, siedzący w~ ciemności, ujrzał wielkie światło, a~ im, osiadłym w~ krainie i~ cieniu śmierci, wzeszło światło.
\vs{17} Od wtedy Jezus zaczął głosić i~ mówić: \underline{zmieniajcie myślenie,} gdyż \doubleline{zbliżyło się} królestwo niebios.
\vs{18} Idąc zaś wzdłuż morza Galilejskiego, dostrzegł dwóch braci: Szymona, nazywanego Piotrem, oraz Andrzeja, jego brata, zarzucających niewód w~ morze. Byli bowiem rybakami.
\vs{19} I~ powiedział im: chodźcie za mną, a~ uczynię was rybakami ludzi.
\vs{20} A~ oni, natychmiast zostawiwszy sieci, \underline{poszli za} nim.
\vs{21} I~ \underline{gdy przeszedł} stamtąd \underline{dalej,} dostrzegł dwóch innych braci: Jakuba, (syna) owego Zebedeusza i~ Jana, jego brata, (jak) w~ łodzi z~ ojcem swym, Zebedeuszem, przygotowywali swoje sieci. I~ wezwał ich.
\vs{22} Oni zaś natychmiast, zostawiwszy łódź i~ ojca swego, \underline{zaczęli iść za} nim.
\vs{23} I~ obchodził Jezus całą Galileę, ucząc w~ ich synagogach, i~ głosił \underline{dobrą nowinę} królestwa, i~ uzdrawiał wszelką chorobę i~ każdą niemoc wśród ludzi.
\vs{24} I~ \underline{rozeszła się} wieść \doubleline{o nim} po całej Syrii i~ znosili mu wszystkich, źle \underline{się mających,} przyciśniętych rozmaitymi chorobami i~ męczarniami, i~ opętanych, i~ epileptyków, i~ sparaliżowanych, i~ uzdrowił ich.
\vs{25} I~ \underline{szły za} nim liczne tłumy z~ Galilei i~ Dekapolis, i~ Jerozolimy, i~ Judei, i~ \doubleline{z drugiej strony} Jordanu.
//
\endgl
\begingl
\lettrine[loversize=1,lraise=-1.3]{5 }{}%
\gla
 1161 1492 3793 305 1519 3735 2532 2523 846 2523 4334 846 846 3101
{} 2532 455 4750 846 1321 846 3004
{} 3107 4434 4151 3754 846 2076 932 3772
{} 3107 3996 3754 846 3870
{} 3107 4239 3754 846 2816 1093
{} 3107 3983 1343 2532 1372 {} 3754 846 5526
{} 3107 1655 3754 846 1653
{} 3107 2513 2588 3754 846 3700 2316
{} 3107 1518 3754 846 2564 5207 2316
{} 3107 1377 1752 1343 3754 846 2076 932 3772
{} 3107 2075 3752 3679 5209 3679 2532 1377 2532 2036 3956 4190 4487 2596 5216 5574 1752 1700
{} 5463 2532 21 3754 3408 5216 4183 1722 3772 3779 1063 1377 4396 4253 5216
{} 5210 2075 217 1093 1437 1161 217 3471 1722 5101 233 1519 3762 2480 2089 {} 2480 1487 3361 {} 906 1854 2532 2662 5259 444
{} 5210 2075 5457 2889 3756 1410 2928 4172 2749 1883 3735
{} 3761 2545 {} 3088 2532 5087 846 5259 3426 235 1909 3087 2532 2989 3956 1722 3614
{} 3779 2989 5216 5457 1715 444 3704 1492 5216 2570 2041 2532 1392 3962 5216 1722 3772
{} 3361 3543 3754 2064 2647 3551 2228 4396 3756 2064 2647 235 4137
{} 281 1063 3004 5213 2193 302 3928 3772 2532 1093 1520 2503 2228 1520 2762 3756 3361 3928 575 3551 2193 302 3956 1096
{} 1437 3767 3739 3089 1520 5130 1646 1785 2532 1321 3779 444 1646 2564 1722 932 3772 3739 1161 302 4160 2532 1321 3778 3173 2564 1722 932 3772
{} 1063 3004 5213 3754 1437 5216 1343 3361 4052 4119 {} 1122 2532 5330 3756 3361 1525 1519 932 3772
{} 191 3754 4483 744 3756 5407 1161 3739 {} 302 5407 1777 2071 2920
{} 1473 1161 3004 5213 3754 3956 3588 3710 1500 {} 80 846 1777 2071 2920 1161 3739 302 2036 846 80 4469 1777 2071 4892 1161 3739 302 2036 3474 1777 2071 1519 4442 1067
{} 1437 3767 4374 1435 4675 1909 2379 2532 1563 3415 3754 4675 80 2192 5100 2596 4675
{} 863 1563 1435 4675 1715 2379 2532 5217 4412 1259 4675 80 2532 5119 2064 4374 1435 4675
{} 2468 2132 476 4675 5035 2193 3755 1488 3326 846 1722 3598 3379 476 3379 3860 4571 2923 2532 2923 3860 4571 5257 2532 906 1519 5438
{} 281 3004 4671 3756 3361 1831 1564 2193 302 591 2078 2835
{} 191 3754 4483 3756 3431
{} 1161 1473 5213 3004 3754 3956 991 1135 4314 846 1937 2235 3431 846 3431 1722 846 2588
{} 1161 1487 4675 1188 3788 4624 4571 1807 846 2532 906 575 4675 4851 1063 4851 4671 2443 622 1520 4675 3196 2532 3361 {} 3650 4675 4983 906 1519 1067
{} 2532 1487 4675 1188 5495 4624 4571 1581 846 2532 906 575 4675 4851 1063 4851 4671 2443 622 1520 4675 3196 2532 3361 {} 3650 4675 4983 906 1519 1067
{} 4483 1161 3754 3739 302 630 846 1135 1325 846 1325 647
{} 1161 1473 5213 3004 3754 3739 302 630 846 1135 3924 3056 4202 4160 846 3429 2532 1437 3739 630 1060 3429
{} 191 3825 3754 4483 744 3756 1964 1161 591 2962 4675 3727
{} 1161 1473 5213 3004 3654 3361 3660 3383 1722 3772 3754 2076 2362 2316
{} 3383 1722 1093 3754 2076 5286 4228 846 3383 1519 2414 3754 2076 4172 3173 935
{} 3383 1722 4675 2776 3383 3660 3754 3756 1410 1520 2359 4160 3022 2228 3189
{} 1161 5216 3056 2077 3483 3483 3756 3756 1161 3588 4053 5130 1537 4190 2076
{} 191 3754 4483 3788 473 3788 2532 3599 473 3599
{} 1161 1473 5213 3004 3361 436 4190 235 3748 4474 4571 1909 1188 4600 4762 846 2532 243
{} 2532 3588 2309 4671 2919 2532 2983 4675 5509 863 846 2532 2440
{} 2532 3748 4571 29 {} 1520 3400 5217 3326 846 1417
{} 3588 4571 154 1325 2532 3588 2309 575 4675 1155 3361 654
{} 191 3754 4483 25 4675 4139 2532 4675 2190 3404
{} 1161 1473 5213 3004 25 5216 2190 2127 3588 5209 2672 2573 4160 3588 5209 3404 2532 4336 5228 3588 5209 1908 2532 1377 5209
{} 3704 1096 5207 5216 3962 1722 3772 3754 846 2246 393 1909 4190 2532 18 2532 1026 1909 1342 2532 94
{} 1437 1063 25 3588 5209 25 5101 2192 3408 3780 4160 846 2532 5057
{} 2532 1437 782 3440 5216 5384 5101 4053 4160 3780 4160 3779 2532 5057
{} 5210 3767 2071 5046 5618 5046 2076 5216 3962 3588 1722 3772
//
\glb
 Gdy zobaczył tłumy, wszedł na górę. I~ \underline{gdy} sam \underline{usiadł,} \doubleline{podeszli do} niego jego uczniowie.
\vs{2} I~ otworzywszy usta swoje, uczył ich, mówiąc:
\vs{3} Szczęśliwi ubodzy duchem, ponieważ ich jest królestwo niebios.
\vs{4} Szczęśliwi \underline{smucący się,} bo oni \doubleline{będą pocieszeni.}
\vs{5} Szczęśliwi łagodni, ponieważ oni odziedziczą ziemię.
\vs{6} Szczęśliwi \underline{odczuwający głód} sprawiedliwości i~ pragnący (jej), bo oni \doubleline{będą nakarmieni.}
\vs{7} Szczęśliwi litościwi, ponieważ oni \underline{litości dostąpią.}
\vs{8} Szczęśliwi czyści sercem, bo oni zobaczą Boga.
\vs{9} Szczęśliwi \underline{pokój czyniący,} bo oni \doubleline{będą nazwani} synami Boga.
\vs{10} Szczęśliwi prześladowani \underline{z powodu} sprawiedliwości, ponieważ ich jest królestwo niebios.
\vs{11} Szczęśliwi jesteście, gdy \underline{robią} wam \underline{wyrzuty,} i~ prześladują, i~ mówią wszelkie złe słowa przeciwko wam, kłamiąc \doubleline{z powodu} mnie.
\vs{12} \underline{Radujcie się} i~ \doubleline{niezmiernie się cieszcie,} ponieważ zapłata wasza wielka w~ niebiosach. Tak bowiem prześladowali proroków przed wami.
\vs{13} Wy jesteście solą ziemi. Gdyby zaś sól \underline{stała się bezużyteczna,} [-] czym \doubleline{będzie posolona?} Na nic \underline{się} już (nie) \underline{przydaje.} Czy nie (na) wyrzucenie \doubleline{na zewnątrz} i~ zdeptanie przez ludzi?
\vs{14} Wy jesteście światłem świata. Nie \underline{jest w stanie} \doubleline{ukryć się} miasto, położone na górze.
\vs{15} \underline{I nie} zapala (nikt) lampy, [-] umieszczając ją pod korcem, lecz na podstawce; i~ świeci wszystkim w~ domu.
\vs{16} Tak \underline{niech rozświeci się} wasze światło przed ludźmi, aby dostrzegli wasze piękne czyny i~ uwielbili Ojca waszego w~ niebiosach.
\vs{17} Nie sądźcie, że przyszedłem zniszczyć Torę lub Proroków. Nie przyszedłem zniszczyć, lecz wypełnić.
\vs{18} Zaprawdę bowiem mówię wam: \underline{aż do} [-] przeminięcia nieba i~ ziemi, jedna jota lub jedna kreska nie [-] przeminie z~ Tory, aż [-] wszystko \doubleline{się stanie.}
\vs{19} Gdyby więc ktoś rozwiązał jedno \underline{z tych} najmniejszych przykazań i~ uczył tak ludzi, najmniejszym \doubleline{będzie nazwany} w~ królestwie niebios. Kto zaś [-] \underline{by czynił} i~ uczył, ten wielkim \doubleline{będzie nazwany} w~ królestwie niebios.
\vs{20} Albowiem mówię wam, że jeśliby wasza sprawiedliwość nie \underline{była pełniejsza,} \underline{była pełniejsza,} (niż) \doubleline{znawców Pisma} i~ faryzeuszów, nie [-] wejdziecie do królestwa niebios.
\vs{21} Słyszeliście, że powiedziano przodkom: nie \underline{będziesz zabijał,} a~ kto (by) [-] zabił, podlegać będzie sądowi.
\vs{22} Ja zaś mówię wam, że każdy, kto \underline{się gniewa} \doubleline{bez powodu} (na) brata swego, podlegać będzie sądowi. A~ kto [-] \underline{by powiedział} swemu bratu: \doubleline{„pusta głowo”,} podlegać będzie sanhedrynowi. A~ kto [-] \underline{by powiedział:} „bezbożniku”, podlegać będzie [-] ogniowi Gehenny.
\vs{23} Jeśli więc przyniesiesz dar swój na ołtarz i~ tam \underline{przypomnisz sobie,} że twój brat ma coś przeciwko tobie,
\vs{24} pozostaw tam dar swój przed ołtarzem, i~ idź najpierw \underline{pojednać się ze} swoim bratem, a~ wtedy przyjdź, przynosząc dar swój.
\vs{25} \underline{Odmień relacje} \underline{Odmień relacje} \doubleline{z przeciwnikiem} swoim szybko, dopóki [-] jesteś z~ nim w~ drodze, \underline{aby} przeciwnik \underline{nie} wydał cię sędziemu, a~ sędzia wydałby cię podwładnemu, i~ \doubleline{wrzucono by cię} do więzienia.
\vs{26} Zaprawdę powiadam ci: Nie [-] wyjdziesz stamtąd, dopóki [-] \underline{nie oddasz} ostatniej \doubleline{czwartej części asa.}
\vs{27} Słyszeliście, że powiedziano Nie cudzołóż.
\vs{28} Ale ja wam mówię, że każdy, \underline{kto patrzy na} kobietę, aby jej pożądać, już \doubleline{popełnił} \underline{z nią} \doubleline{cudzołóstwo} w~ swoim sercu.
\vs{29} A~ jeżeli twoje prawe oko \underline{sprawia, że upadasz,} \underline{sprawia, że upadasz,} wyrwij je i~ odrzuć \doubleline{z dala od} siebie. \underline{Pożyteczniej} bowiem \underline{jest} \doubleline{dla ciebie,} aby stracić jeden \underline{z twoich} członków, a~ nie (żeby) całe twoje ciało \doubleline{było wrzucone} do Gehenny.
\vs{30} A~ jeżeli twoja prawa ręka \underline{sprawia, że upadasz,} \underline{sprawia, że upadasz,} wytnij ją i~ odrzuć \doubleline{z dala od} siebie. \underline{Pożyteczniej} bowiem \underline{jest} \doubleline{dla ciebie,} aby stracić jeden \underline{z twoich} członków, a~ nie (żeby) całe twoje ciało \doubleline{było wrzucone} do Gehenny.
\vs{31} Powiedziano też, że kto by uwolnił swoją żonę, \underline{niech} jej \underline{da} \doubleline{akt uwolnienia.}
\vs{32} Lecz ja wam mówię, że kto by uwolnił swoją żonę– \underline{z wyjątkiem,} \doubleline{gdy mowa o} nierządzie– czyni ją cudzołożną. A~ gdyby ktoś uwolnioną \underline{wziął za żonę–} cudzołoży.
\vs{33} Słyszeliście też, że powiedziano przodkom: Nie \underline{będziesz fałszywie przysięgał,} ale oddasz Panu swe przysięgi.
\vs{34} Ale ja wam mówię: \underline{W ogóle} nie przysięgajcie– ani na niebiosa, ponieważ są tronem Boga.
\vs{35} Ani na ziemię, ponieważ jest podnóżkiem nóg jego. Ani na Jerozolimę, ponieważ jest miastem wielkiego króla.
\vs{36} \underline{Ani} na swoją głowę \underline{nie} \doubleline{będziesz przysięgał,} ponieważ nie \underline{jesteś w stanie} jednego włosa uczynić białym albo czarnym.
\vs{37} Ale wasza mowa \underline{niech będzie:} Tak– tak, nie– nie. A~ co ponadto, to od złego pochodzi.
\vs{38} Słyszeliście, że powiedziano: „Oko za oko” oraz „ząb za ząb”.
\vs{39} Ale ja wam mówię: Nie \underline{przeciwstawiajcie się} złemu, \doubleline{ale jeśli} ktoś uderzy cię w~ prawy policzek, \underline{zwróć ku} niemu i~ drugi.
\vs{40} A~ \underline{temu, kto} \doubleline{chce się} \underline{z tobą} sądzić i~ wziąć twoją \doubleline{szatę spodnią,} zostaw mu i~ \underline{szatę wierzchnią.}
\vs{41} I~ ktokolwiek cię zmusza (do) jednej mili, przejdź z~ nim dwie.
\vs{42} \underline{Temu, kto} cię prosi, daj, a~ \doubleline{od tego, kto} chce od ciebie pożyczyć, nie \underline{odwracaj się.}
\vs{43} Słyszeliście, że powiedziano: \underline{Będziesz miłował} swego bliźniego, a~ swego nieprzyjaciela \doubleline{będziesz nienawidził?}
\vs{44} A~ ja wam mówię: Miłujcie waszych nieprzyjaciół, błogosławcie \underline{tym, którzy} wam złorzeczą, dobrze czyńcie \doubleline{tym, którzy} was nienawidzą oraz \underline{módlcie się} za \doubleline{tych, którzy} was oczerniają i~ prześladują was,
\vs{45} abyście \underline{stali się} synami waszego Ojca w~ niebiosach. Ponieważ jego słońce wschodzi nad złymi i~ dobrymi, i~ \doubleline{deszcz zsyła} na sprawiedliwych i~ niesprawiedliwych.
\vs{46} Gdybyście bowiem miłowali \underline{tych, którzy} was miłują, jaką macie zapłatę? \doubleline{Czy nie} czynią tego także \underline{poborcy podatków?}
\vs{47} I~ jeśli pozdrawiacie tylko waszych przyjaciół, cóż ponadto czynicie? \underline{Czy nie} czynią tego także \doubleline{poborcy podatków?}
\vs{48} Wy zatem bądźcie doskonali, jak doskonały jest wasz Ojciec, \underline{który jest} w~ niebiosach.
//
\endgl
\begingl
\lettrine[loversize=1,lraise=-1.3]{6 }{}%
\gla
 4337 {} 5216 1654 3361 4160 1715 444 {} 4314 846 2300 1161 1487 3361 3756 2192 3408 3844 5216 3962 3588 1722 3772
{} 3767 3752 4160 1654 3361 4537 1715 4675 5618 4160 5273 1722 4864 2532 1722 4505 3704 1392 5259 444 281 3004 5213 568 846 3408
{} 1161 4675 4160 1654 1097 3361 1097 4675 710 {} 5101 4160 4675 1188
{} 3704 4675 1654 5600 1722 2927 2532 4675 3962 991 1722 2927 846 591 4671 1722 5318
{} 2532 3752 4336 3756 2071 5618 5273 3754 5368 4336 2476 1722 4864 2532 1722 1137 4113 3704 302 5316 444 281 3004 5213 3754 568 846 3408
{} 1161 3752 4771 4336 1525 1519 4675 5009 2532 2808 2374 4675 4336 4675 3962 3588 1722 2927 2532 4675 3962 991 1722 2927 591 4671 1722 5318
{} 1161 4336 3361 945 5618 1482 1063 1380 3754 1722 846 4180 1522
{} 3767 3361 3666 846 1063 5216 3962 1492 3739 2192 5532 4253 5209 846 154
{} 5210 3767 3779 4336 2257 3962 1722 3772 37 4675 3686
{} 2064 4675 932 1096 4675 2307 5613 1722 3772 {} 2532 1909 1093
{} 1325 2254 4594 2257 1967 740
{} 2532 863 2254 3783 2257 5613 2532 2249 863 2257 3781
{} 2532 3361 1533 2248 1519 3986 235 4506 2248 575 4190 3754 4675 2076 932 2532 1411 2532 1391 1519 165 281
{} 1437 1063 863 444 846 3900 863 2532 5213 3962 5216 3770
{} 1437 1161 3361 863 444 846 3900 3761 3962 5216 3761 863 3900 5216
{} 1161 3752 3522 3361 1096 4659 5618 5273 853 1063 846 4383 3704 5316 444 3522 281 3004 5213 3754 568 3408 846
{} 1161 4771 3522 218 4675 2776 2532 3538 4675 4383
{} 3704 3361 444 5316 {} 3522 235 4675 3962 3588 1722 2927 2532 4675 3962 3588 991 1722 2927 591 4671
{} 3361 2343 5213 2344 1909 1093 3699 4597 2532 1035 853 2532 3699 2812 1358 2532 2813
{} 1161 2343 5213 2344 1722 3772 3699 3777 4597 3777 1035 {} 853 2532 3699 2812 3756 1358 3761 2813
{} 3699 1063 2076 2344 5216 1563 2071 2532 2588 5216
{} 3088 {} 4983 2076 3788 1437 3767 4675 3788 5600 573 3650 4983 4675 2071 5460
{} 1437 1161 4675 3788 5600 4190 3650 4675 4983 2071 4652 1487 3767 5457 1722 4671 2076 4655 4214 {} 4655
{} 3762 1410 1417 2962 1398 1063 2228 1520 3404 2532 2087 25 2228 1520 472 2532 2087 2706 3756 1410 1398 2316 2532 3126
{} 1223 5124 3004 5213 3361 3309 5216 5590 5101 5315 2532 5101 4095 3366 5216 4983 5101 {} 1746 3780 5590 3780 2076 4119 {} 5160 2532 4983 {} 1742
{} 1689 1519 4071 3772 3754 3756 4687 3761 {} 2325 3761 {} 4863 1519 596 2532 3962 5216 3770 5142 846 3756 5210 3756 {} 3123 1308 {} 846
{} 1161 5101 1537 5216 3309 1410 4369 1909 846 2244 1520 4083
{} 2532 4012 1742 5101 3309 2648 2918 68 4459 837 3756 2872 3761 {} 3514
{} 1161 3004 5213 3754 3761 4672 1722 3956 1391 846 {} 5613 {} 4016 5613 1520 5130
{} 1487 1161 5528 68 5607 4594 5607 2532 839 1519 2823 906 2316 3779 294 {} 3756 4183 3123 5209 {} 3640
{} 3361 3309 3767 3004 5101 5315 2228 5101 4095 2228 5101 4016
{} 1063 5023 3956 1484 1934 1063 3962 5216 3770 1492 3754 5535 5130 537
{} 1161 2212 4412 932 2316 2532 846 1343 2532 5023 3956 4369 5213 4369
{} 3361 3309 3767 1519 839 1063 839 3309 1438 713 {} 2250 2549 846
//
\glb
 \underline{Pilnujcie się,} (by) waszego \doubleline{gestu miłosierdzia} nie czynić przed ludźmi (po to), aby was widzieli, bo inaczej inaczej nie \underline{będziecie mieli} zapłaty u~ waszego Ojca– tego w~ niebiosach.
\vs{2} Dlatego, gdy czynisz \underline{gest miłosierdzia,} nie trąb przed sobą, \doubleline{jak to} robią hipokryci w~ synagogach i~ na ulicach, aby \underline{być chwalonymi} przez ludzi. Zaprawdę powiadam wam: otrzymują swoją zapłatę.
\vs{3} Gdy ty czynisz \underline{gest miłosierdzia,} \doubleline{niech} nie \doubleline{wie} twoja lewa (ręka), co czyni twoja prawa,
\vs{4} aby twój \underline{gest miłosierdzia} pozostał w~ ukryciu, a~ twój Ojciec, \doubleline{który widzi} w~ ukryciu, on odda ci w~ \underline{sposób jawny.}
\vs{5} A~ gdy \underline{się modlicie,} nie bądźcie jak hipokryci. Ponieważ kochają \doubleline{się modlić,} stojąc w~ synagogach i~ na narożnikach ulic, aby [-] \underline{pokazać się} ludziom. Zaprawdę powiadam wam, że odbierają swoją zapłatę.
\vs{6} Ale gdy ty \underline{się modlisz,} wejdź do swego pokoju i~ zamknij drzwi swoje, \doubleline{modląc się} \underline{do twego} Ojca, \doubleline{który jest} w~ ukryciu. A~ twój Ojciec, \underline{który widzi} w~ ukryciu, odda ci w~ \doubleline{sposób jawny.}
\vs{7} A~ \underline{modląc się,} nie paplajcie jak poganie, ponieważ sądzą, że \doubleline{z powodu} swojej wielomówności \underline{będą wysłuchani.}
\vs{8} Więc nie \underline{upodabniajcie się} \doubleline{do nich,} bowiem wasz Ojciec dostrzega, jakie macie potrzeby, zanim wy go poprosicie.
\vs{9} Wy więc tak \underline{się módlcie:} Nasz Ojcze w~ niebiosach, \doubleline{niech będzie uświęcone} twoje imię.
\vs{10} \underline{Niech przyjdzie} twoje królestwo. \doubleline{Niech się dzieje} twoja wola: jak w~ niebiosach, (tak) i~ na ziemi.
\vs{11} Daj nam dzisiaj naszego codziennego chleba.
\vs{12} I~ odpuść nam długi nasze, jak i~ my odpuszczamy naszym dłużnikom.
\vs{13} I~ nie wprowadzaj nas w~ doświadczenie, ale wybaw nas od zła, ponieważ twoje jest królestwo i~ moc, i~ chwała na wieczność. Amen.
\vs{14} Jeśli bowiem odpuścicie ludziom ich upadki, odpuści i~ wam Ojciec wasz niebiański.
\vs{15} Gdybyście jednak nie odpuścili ludziom ich upadków, \underline{to i} Ojciec wasz \underline{nie} odpuści upadków waszych.
\vs{16} A~ gdy pościcie, nie \underline{stawajcie się} smutni jak hipokryci; niszczą bowiem swoje oblicza, aby \doubleline{pokazać się} ludziom, \underline{że poszczą.} Zaprawdę powiadam wam, że otrzymują zapłatę swoją.
\vs{17} Ale ty, \underline{gdy pościsz,} namaść swoją głowę i~ umyj swoje oblicze;
\vs{18} Aby nie ludziom ukazać, (że) pościsz, lecz twojemu Ojcu, \underline{który jest} w~ ukryciu; a~ twój Ojciec, który widzi w~ ukryciu, odda ci
\vs{19} Nie gromadźcie sobie skarbów na ziemi, gdzie mól i~ rdza niszczą, i~ gdzie złodzieje \underline{podkopują się} i~ kradną;
\vs{20} Ale gromadźcie sobie skarby w~ niebiosach, gdzie ani mól, ani rdza (nie) niszczą i~ gdzie złodzieje nie \underline{podkopują się} \doubleline{i nie} kradną.
\vs{21} Gdzie bowiem jest skarb wasz, tam będzie i~ serce wasze.
\vs{22} Lampą (dla) ciała jest oko. Jeśli więc twoje oko jest szczere, całe ciało twoje jest jasne.
\vs{23} Jeśli zaś twoje oko jest złe, całe twoje ciało jest ciemne. Jeśli więc światło w~ tobie jest ciemnością, \underline{jakże wielka} (to) ciemność.
\vs{24} \underline{Nikt nie} \doubleline{jest w stanie} dwom panom służyć, gdyż albo jednego znienawidzi, a~ drugiego umiłuje, albo jednego \underline{będzie się trzymał,} a~ drugim pogardzi. Nie \doubleline{jesteście w stanie} służyć Bogu i~ mamonie.
\vs{25} Dlatego Dlatego mówię wam: Nie \underline{troszczcie się o} waszą duszę, co \doubleline{będziecie jeść} albo co \underline{będziecie pić,} \doubleline{ani o} wasze ciało, co (nań) przyoblec. \underline{Czy} dusza \underline{nie} jest \doubleline{czymś więcej,} (niż) pokarm, a~ ciało, (niż) odzienie?
\vs{26} Popatrzcie na ptaki niebios, że nie sieją ani (nie) żną, ani (nie) zbierają do magazynów, a~ Ojciec wasz niebiański żywi je. \underline{Czy} wy \underline{nie} (jesteście) dużo ważniejsi, (niż) one?
\vs{27} A~ kto z~ was, \underline{troszcząc się,} \doubleline{jest w stanie} dodać do swego wzrostu jeden łokieć?
\vs{28} A~ o~ odzienie dlaczego \underline{się troszczycie?} \doubleline{Przypatrzcie się} liliom polnym, jak rosną. Nie \underline{trudzą się,} ani (nie) przędą;
\vs{29} A~ mówię wam, że nawet Salomon w~ całej chwale swojej (nie) \underline{był} (tak) odziany, \underline{jak} jedna \doubleline{z nich.}
\vs{30} Jeśli więc trawę polną, \underline{która} dziś \underline{jest,} a~ jutro do pieca \doubleline{będzie wrzucona,} Bóg tak odziewa, (czyż) nie dużo bardziej was, (ludzie) \underline{małej wiary?}
\vs{31} Nie \underline{troszczcie się} więc, mówiąc: Cóż \doubleline{będziemy jeść?} Albo: Co \underline{będziemy pić?} Albo: \doubleline{W co} \underline{się odziejemy?}
\vs{32} Bowiem tego wszystkiego narody pragną. Bo Ojciec wasz niebiański dostrzega, że potrzebujecie tego wszystkiego.
\vs{33} Ale szukajcie \underline{przede wszystkim} królestwa bożego i~ jego sprawiedliwości, a~ to wszystko \doubleline{będzie} wam \doubleline{dodane.}
\vs{34} Nie \underline{troszczcie się} więc o~ jutro, bowiem jutro \doubleline{będzie się troszczyło} \underline{o siebie.} Dosyć (ma) dzień zła swego.
//
\endgl
\begingl
\lettrine[loversize=1,lraise=-1.3]{7 }{}%
\gla
 3361 2919 2443 3361 2919
{} 1722 3739 1063 2917 2919 {} 2919 2532 1722 3739 3358 3354 {} 3354 5213 3354
{} 1161 5101 991 2595 1722 3788 80 4675 1161 1722 4674 3788 1385 3756 2657
{} 2228 4459 2046 80 4675 863 1544 2595 575 3788 4675 2532 2400 1385 {} 1722 3788 4675
{} 5273 1544 4412 1385 1537 3788 4675 2532 5119 1227 {} 1544 2595 1537 3788 80 4675
{} 3361 1325 2965 3588 40 3366 906 5216 3135 1715 5519 3379 2662 846 846 1722 4228 2532 4762 {} 4486 5209
{} 154 2532 1325 5213 1325 2212 2532 2147 2925 2532 455 5213 455
{} 3956 1063 154 2983 2532 2212 2147 2532 2925 455
{} 2228 5101 1537 5216 2076 444 3739 1437 846 5207 154 {} 740 3361 3037 846 1929
{} 2532 1437 {} 2486 154 3361 3789 846 1929
{} 1487 3767 5210 5607 4190 1492 1325 18 1390 5216 5043 4214 3123 3962 5216 1722 3772 1325 154 846 {} 18
{} 3956 3767 3745 302 2309 2443 5213 444 4160 3779 2532 5210 846 4160 3778 2076 1063 3551 2532 4396
{} 1525 1223 4728 4439 3754 4116 {} 4439 2532 2149 3598 3588 520 1519 684 2532 4183 1526 {} 1223 846 1525
{} {} 5101 4728 {} 4439 2532 2346 3598 3588 520 1519 2222 2532 3641 1526 {} 846 2147
{} 1161 4337 575 5578 3748 2064 4314 5209 1722 1742 4263 1161 2081 1526 727 3074
{} 575 846 2590 846 1921 3385 4816 575 173 4718 2228 575 5146 4810
{} 3779 3956 18 1186 4160 2570 2590 1161 4550 1186 4160 4190 2590
{} 3756 1410 18 1186 4160 4190 2590 3761 4550 1186 4160 2570 2590
{} 3956 1186 4160 3361 4160 2570 2590 1581 2532 906 1519 4442
{} 686 1065 575 846 2590 1921 846
{} 3756 3956 3004 3427 3004 2962 2962 1525 1519 932 3772 235 4160 2307 3450 3962 1722 3772
{} 4183 2046 3427 1722 1565 2250 2962 2962 3756 {} 4674 3686 4395 2532 4674 3686 1140 1544 2532 4674 3686 4160 4183 1411
{} 2532 5119 3670 846 3754 3763 5209 3763 1097 672 575 1700 2038 458
{} 3956 3767 3748 5128 3056 3450 191 2532 4160 846 846 3666 {} 5429 435 3748 3618 846 3614 1909 4073
{} 2532 2597 1028 2532 2064 4215 2532 4154 417 2532 4363 {} 1565 3614 2532 3756 4098 2311 1063 1909 4073
{} 2532 3956 191 5128 3056 3450 2532 3361 4160 846 3666 {} 3474 435 3748 3618 846 3614 1909 285
{} 2532 2597 1028 2532 2064 4215 2532 4154 417 2532 4350 {} 1565 3614 2532 4098 2532 4431 846 2258 3173
{} 2532 1096 {} 3753 2424 4931 5128 3056 1605 3793 1909 846 1322
{} 2258 1063 1321 846 5613 2192 1849 2532 3756 5613 1122
//
\glb
 Nie sądźcie, abyście nie \underline{byli osądzeni.}
\vs{2} Jakim Jakim bowiem sądem sądzicie, (takim) \underline{będziecie osądzeni,} i~ jaką jaką miarą mierzycie, (taką) \doubleline{będzie} wam \doubleline{odmierzone.}
\vs{3} A~ czemu widzisz źdźbło w~ oku brata swego, a~ w~ swoim oku belki nie dostrzegasz?
\vs{4} Albo jak powiesz bratu swojemu: Pozwól, \underline{niech wyrzucę} źdźbło z~ oka twego, a~ oto belka (jest) w~ oku twoim?
\vs{5} Hipokryto, wyrzuć \underline{przede wszystkim} belkę z~ oka swojego, a~ wtedy przejrzysz, (aby) wyrzucić źdźbło z~ oka brata swojego.
\vs{6} Nie dawajcie psom \underline{tego, co} święte, \doubleline{ani nie} rzucajcie swoich pereł przed świnie, \underline{aby nie} zdeptały ich swoimi [-] nogami, a~ \doubleline{obróciwszy się–} (nie) roztrzaskały was.
\vs{7} Proście, a~ \underline{będzie} wam \underline{dane;} szukajcie, a~ znajdziecie; pukajcie, a~ \doubleline{będzie} wam \doubleline{otworzone.}
\vs{8} Każdy bowiem proszący– dostaje, a~ szukający– znajduje, a~ pukającemu \underline{będzie otworzone.}
\vs{9} Albo kto z~ was jest człowiekiem, który, gdy jego syn poprosi (o) chleb, [-] kamień mu poda?
\vs{10} Lub gdy (o) rybę poprosi, [-] węża mu poda?
\vs{11} Jeśli więc wy, będąc złymi, umiecie dawać dobre dary swoim dzieciom, \underline{o ile} bardziej Ojciec wasz w~ niebiosach da proszącym go (to, co) dobre.
\vs{12} Wszystko więc, co [-] \underline{byście chcieli,} aby wam ludzie czynili, tak i~ wy im czyńcie. Taka jest bowiem Tora i~ Prorocy.
\vs{13} Wejdźcie przez wąską bramę. Bo szeroka (jest) brama i~ szeroka droga, która prowadzi na zniszczenie, i~ liczni są (ci, którzy) przez nią wchodzą.
\vs{14} (A) \underline{cóż za} wąska (jest) brama i~ ciasna droga, która prowadzi do życia, a~ nieliczni są, (którzy) ją znajdują.
\vs{15} Ale \underline{wystrzegajcie się} [-] \doubleline{fałszywych proroków,} którzy przychodzą do was w~ odzieniu owiec, a~ wewnątrz są rabującymi wilkami.
\vs{16} Po ich owocach ich poznacie. Czy \underline{zbiera się} z~ ciernia \doubleline{kiście winogron,} albo z~ ostu– figi?
\vs{17} Tak każde dobre drzewo rodzi piękne owoce, a~ zgniłe drzewo– rodzi złe owoce.
\vs{18} Nie \underline{jest w stanie} dobre drzewo rodzić złych owoców, ani zgniłe drzewo– rodzić pięknych owoców.
\vs{19} Każde drzewo, \underline{które} nie \underline{rodzi} pięknego owocu, \doubleline{jest wycinane} i~ wrzucane w~ ogień.
\vs{20} \underline{A zatem} [-] z~ ich owoców poznacie ich.
\vs{21} Nie każdy, \underline{który} mi \underline{mówi:} „Panie, Panie”, wejdzie do królestwa niebios, ale \doubleline{ten, który czyni} wolę mojego Ojca w~ niebiosach.
\vs{22} Wielu powie mi w~ tamtym dniu: „Panie, Panie, nie (w) twoim imieniu prorokowaliśmy? I~ twoim imieniem demony wyrzuciliśmy! I~ twoim imieniem uczyniliśmy liczne cuda!”
\vs{23} A~ wtedy wyznam im, że: \underline{„Nigdy} was \underline{nie} poznałem. Odejdźcie ode mnie, czyniący nieprawość”.
\vs{24} Każdy więc, kto tych słów moich słucha i~ czyni je, ten \underline{będzie podobny} (do) mądrego męża, który zbudował swój dom na skale.
\vs{25} I~ zstąpił deszcz, i~ przybyły rzeki, i~ zawiały wiatry, i~ wpadły (na) ten dom, ale nie upadł, \underline{utwierdzony był} bowiem na skale.
\vs{26} A~ każdy, \underline{kto słucha} tych słów moich, ale nie czyni ich, \doubleline{będzie podobny} (do) głupiego męża, który zbudował swój dom na piasku.
\vs{27} I~ zstąpił deszcz, i~ przybyły rzeki, i~ zawiały wiatry, i~ uderzyły (na) ten dom, i~ upadł, a~ upadek jego był wielki.
\vs{28} I~ \underline{stało się,} (że) gdy Jezus dokończył tych słów, \doubleline{zdumiewały się} tłumy nad jego nauką.
\vs{29} Albowiem Albowiem uczył ich, jak posiadający moc, a~ nie jak \underline{znawcy Pisma.}
//
\endgl
\begingl
\lettrine[loversize=1,lraise=-1.3]{8 }{}%
\gla
 1161 2597 846 2597 575 3735 190 846 4183 3793
{} 2532 2400 3015 2064 4352 846 4352 {} 3004 2962 1437 2309 1410 3165 2511
{} 2532 1614 5495 2424 680 846 3004 2309 2511 2532 2112 2511 846 3014
{} 2532 3004 846 2424 3708 3367 {} 2036 235 5217 1166 4572 2409 2532 4374 1435 3739 4367 3475 1519 3142 846
{} 1161 1525 846 1525 1519 2584 4334 846 1543 3870 846
{} 2532 3004 2962 3450 3816 906 1722 3614 3885 928 1171 928
{} 2532 2424 3004 846 1473 2064 {} 846 2323
{} 2532 1543 611 5346 2962 3756 1510 2425 {} 2443 1525 5259 3450 4721 235 2036 3440 3056 2532 3450 3816 2390
{} 1063 2532 1473 {} 1510 444 5259 1849 2192 5259 1683 4757 2532 3004 5129 4198 2532 4198 2532 243 2064 2532 2064 2532 3450 1401 4160 5124 2532 4160
{} 1161 2424 191 {} 191 2296 2532 2036 190 {} 281 3004 5213 3761 2147 1722 2474 5118 4102
{} 1161 3004 5213 3754 2240 4183 575 395 2532 1424 2532 347 {} 1722 932 3772 3326 11 2532 2464 2532 2384
{} 1161 5207 932 1544 1519 1857 4655 1563 2071 2805 2532 1030 3599
{} 2532 2036 2424 1543 5217 2532 5613 4100 {} 1096 4671 1096 2532 1722 1565 5610 2390 846 3816
{} 2532 2424 2064 1519 3614 4074 1492 846 3994 906 2532 4445
{} 2532 680 846 5495 2532 863 846 4446 2532 1453 2532 1247 846
{} 1161 {} 1096 3798 4374 846 4183 1139 2532 3056 1544 4151 2532 2323 3956 2192 2560 2192
{} 3704 4137 3588 4483 1223 4396 2268 3004 846 2983 2257 769 2532 3554 941
{} 1161 2424 1492 4183 3793 4012 846 2753 565 1519 4008
{} 2532 1520 {} 1122 4334 {} 2036 846 1320 190 4671 3699 1437 {} 565
{} 2532 3004 846 2424 258 2192 5454 2532 4071 3772 2682 1161 5207 444 3756 2192 4226 2776 2827
{} 1161 846 2087 3101 2036 846 2962 2010 3427 4412 565 2532 2290 3450 3962
{} 1161 2424 2036 846 190 3427 2532 863 3498 2290 1438 3498
{} 2532 1684 846 1684 1519 4143 190 846 846 3101
{} 2532 2400 3173 4578 1096 1722 2281 5620 4143 2572 5259 2949 846 1161 2518
{} 2532 3101 4334 1453 846 3004 2962 4982 2248 622
{} 2532 3004 846 5101 2075 1169 3640 5119 1453 2008 417 2532 2281 2532 1096 3173 1055
{} 1161 444 2296 3004 4217 3778 2076 3754 2532 417 2532 2281 5219 846 5219
{} 2532 846 2064 1519 4008 1519 5561 1086 5221 846 5221 1417 1139 {} 1831 1537 3419 3029 5467 5620 5100 3361 2480 3928 1223 1565 3598
{} 2532 2400 2896 3004 5101 2254 2532 4671 2424 5207 2316 2064 5602 4253 2540 928 2248 928
{} 1161 3112 575 846 2258 34 4183 1006 5519
{} 1161 1142 3870 846 3004 1487 2248 1544 2010 2254 565 1519 34 5519
{} 2532 2036 846 5217 1161 1831 565 1519 34 5519 2532 2400 3956 34 5519 3729 2596 2911 1519 2281 2532 599 1722 5204
{} 1161 1006 5343 2532 565 1519 4172 518 3956 2532 3588 1139
{} 2532 2400 3956 4172 1831 2424 1519 4877 2532 1492 846 1492 3870 3704 3327 {} 575 846 3725
//
\glb
 A~ \underline{gdy} on \underline{zszedł} z~ góry, \doubleline{szły za} nim liczne tłumy.
\vs{2} A~ oto trędowaty podszedł, \underline{oddał} mu \underline{pokłon,} (i) powiedział: Panie, gdybyś zechciał, \doubleline{jesteś w stanie} mnie oczyścić.
\vs{3} I~ wyciągnąwszy rękę, Jezus dotknął go, mówiąc: Chcę, \underline{zostań oczyszczony.} I~ natychmiast \doubleline{został oczyszczony} jego trąd.
\vs{4} I~ powiedział mu Jezus: Uważaj, nikomu (nie) mów, ale odejdź, pokaż się kapłanowi i~ zanieś dar, który nakazał Mojżesz, na świadectwo \underline{dla nich.}
\vs{5} A~ \underline{gdy} on \underline{wszedł} do Kafarnaum, podszedł \doubleline{do niego} setnik, prosząc go
\vs{6} i~ mówiąc: Panie, mój chłopiec leży w~ domu sparaliżowany, \underline{poddawany} strasznej \underline{próbie.}
\vs{7} A~ Jezus rzekł mu: Ja przyjdę (i) go uzdrowię.
\vs{8} A~ setnik, odpowiadając, powiedział: Panie, nie jestem wystarczająco (godzien), abyś wszedł pod mój dach, ale powiedz tylko słowo, a~ mój chłopiec \underline{zostanie uzdrowiony.}
\vs{9} Bowiem i~ ja (choć) jestem człowiekiem pod władzą, mam pod sobą żołnierzy i~ mówię temu: Ruszaj. I~ wyrusza. A~ innemu: Przyjdź. I~ przychodzi. A~ mojemu niewolnikowi: Uczyń to. I~ czyni.
\vs{10} A~ Jezus, \underline{gdy} (to) \underline{usłyszał,} \doubleline{zdziwił się} i~ powiedział \underline{idącym za} (nim): Zaprawdę mówię wam, \doubleline{że nie} znalazłem w~ Izraelu \underline{tak wielkiej} wiary.
\vs{11} I~ mówię wam, że przyjdą liczni ze wschodu i~ zachodu, i~ \underline{położą się} (przy stole) w~ królestwie niebios z~ Abrahamem, i~ Izaakiem, i~ Jakubem.
\vs{12} A~ synowie królestwa \underline{zostaną wyrzuceni} w~ zewnętrzną ciemność. Tam będzie płacz i~ zgrzytanie zębów.
\vs{13} I~ powiedział Jezus setnikowi: Odejdź, a~ jak uwierzyłeś, (tak) \underline{niech} ci \underline{się stanie.} I~ w~ tej chwili \doubleline{został uzdrowiony} jego chłopiec.
\vs{14} A~ Jezus, \underline{gdy przybył} do domu Piotra, zobaczył jego teściową, leżącą i~ gorączkującą.
\vs{15} I~ dotknął jej rękę, i~ opuściła ją gorączka. I~ wstała, i~ służyła im.
\vs{16} A~ (gdy) nastał wieczór, przyprowadzali mu licznych opętanych. I~ słowem wyrzucił duchy, oraz uzdrowił wszystkich, \underline{którzy się} źle \underline{mieli,}
\vs{17} aby \underline{wypełniło się} \doubleline{to, co} \underline{zostało powiedziane} przez proroka Izajasza, \doubleline{który mówił:} On wziął nasze słabości oraz choroby zabrał.
\vs{18} A~ Jezus, \underline{gdy zobaczył} liczny tłum wokół siebie, kazał odpłynąć na \doubleline{drugą stronę.}
\vs{19} A~ jeden (ze) \underline{znawców Pisma} podszedł (i) powiedział mu: Nauczycielu, \doubleline{pójdę za} tobą, gdziekolwiek gdziekolwiek (ty) pójdziesz.
\vs{20} I~ rzekł mu Jezus: Lisy mają nory, a~ ptaki niebiańskie– gniazda, lecz Syn Człowieczy nie ma gdzie głowy skłonić.
\vs{21} A~ Jego drugi uczeń powiedział mu: Panie, pozwól mi najpierw odejść i~ pogrzebać mego ojca.
\vs{22} A~ Jezus powiedział mu: \underline{Idź za} mną i~ zostaw zmarłym grzebanie ich zmarłych.
\vs{23} A~ \underline{kiedy} on \underline{wszedł} do łodzi, towarzyszyli mu jego uczniowie.
\vs{24} I~ oto wielki sztorm powstał na morzu, \underline{tak że} łódź \doubleline{była przykrywana} przez fale. On zaś spał.
\vs{25} A~ uczniowie podszedłszy, obudzili go, mówiąc: Panie! Ratuj nas! Giniemy!
\vs{26} I~ powiedział im: Dlaczego jesteście lękliwi, \underline{małej wiary?} Wtedy \doubleline{podniósł się,} zgromił wiatry i~ morze, i~ nastała wielka cisza.
\vs{27} A~ ludzie \underline{zdziwili się,} mówiąc: Kim on jest, że i~ wiatry, i~ morze \doubleline{są} mu \doubleline{posłuszne?}
\vs{28} A~ on, \underline{gdy przybył} na \doubleline{drugą stronę–} na terytorium Gergezeńczyków– \underline{wyszli} mu \underline{na spotkanie} dwaj opętani, (którzy) wyszli z~ grobów– bardzo uciążliwi, \doubleline{tak że} nikt nie \underline{był w stanie} przejść przez tamtą drogę.
\vs{29} I~ oto krzyczeli, mówiąc: Co nam do ciebie, Jezusie, Synu Boga? Przyszedłeś tu przed czasem \underline{poddawać} nas \underline{próbie?}
\vs{30} A~ daleko od nich była trzoda licznych, \underline{pasących się} świń.
\vs{31} A~ demony prosiły go, mówiąc: Jeśli nas wyrzucasz, pozwól nam odejść w~ trzodę świń.
\vs{32} I~ powiedział im: Odejdźcie. A~ \underline{gdy wyszły,} odeszły w~ trzodę świń. I~ oto cała trzoda świń ruszyła z~ urwiska w~ morze. I~ zginęły w~ wodach.
\vs{33} A~ pasterze uciekli i~ \underline{gdy odeszli} do miasta, opowiedzieli wszystko, także o~ opętanych.
\vs{34} I~ oto całe miasto wyszło Jezusowi na spotkanie i~ \underline{gdy} go \underline{zobaczyli,} prosili, żeby odszedł (z dala) od ich granic.
//
\endgl
\begingl
\lettrine[loversize=1,lraise=-1.3]{9 }{}%
\gla
 2532 1684 1519 4143 1276 2532 2064 1519 2398 4172
{} 2532 2400 4374 846 3885 906 1909 2825 2532 2424 1492 846 4102 2036 3885 2293 5043 863 4671 266 4675
{} 2532 2400 5100 1122 2036 1722 1438 3778 987
{} 2532 2424 1492 846 1761 2036 2443 2444 5210 1760 4190 1722 5216 2588
{} 5101 1063 2076 2123 2036 863 4675 266 2228 2036 1453 2532 4043
{} 1161 2443 1492 3754 5207 444 2192 1909 1093 1849 863 266 5119 3004 3885 1453 142 4675 2825 2532 5217 1519 4675 3624
{} 2532 1453 565 1519 846 3624
{} 1161 3793 1492 {} 1492 2296 2532 1392 2316 1325 444 5108 1849
{} 2532 3855 1564 2424 1492 444 2521 1909 5058 3004 3156 2532 3004 846 190 3427 2532 450 190 846
{} 2532 846 345 1722 3614 1096 2532 2400 4183 5057 2532 268 2064 4873 2424 2532 846 3101
{} 2532 1492 {} 5330 2036 846 3101 1223 5101 5216 1320 2068 3326 5057 2532 268
{} 1161 2424 191 {} 191 2036 846 3756 2480 5532 2192 2395 235 2192 2560 2192
{} 4198 1161 3129 5101 {} 2076 1656 2309 2532 3756 2378 3756 2064 1063 2564 1519 3341 1342 235 268
{} 5119 4334 846 3101 2491 {} 3004 1223 5101 2249 2532 5330 4183 3522 1161 4675 3101 3756 3522
{} 2532 2424 2036 846 3361 1410 5207 3567 3996 1909 3745 2076 3326 846 3566 1161 2064 2250 3752 522 575 846 3566 2532 5119 3522
{} 1161 3762 {} 1911 1915 {} 4470 46 4470 1909 3820 2440 1063 {} 846 4138 142 575 2440 2532 4978 1096 5501
{} 3761 906 3501 3631 1519 3820 779 1487 1161 3361 779 4486 2532 3631 1632 2532 779 622 235 906 3501 3631 1519 2537 779 2532 297 4933
{} {} 846 846 5023 2980 2400 1520 {} 758 2064 {} 4352 846 4352 3004 3754 3450 2364 737 5053 235 2064 2007 1909 846 4675 5495 2532 2198
{} 2532 2424 1453 190 846 2532 {} 3101 846
{} 2532 2400 1135 131 1427 2094 4334 3693 680 2899 846 2440
{} 3004 1063 1722 1438 1437 3440 680 846 2440 4982
{} 1161 2424 1994 2532 846 1492 2036 2293 2364 4675 4102 4571 4982 2532 575 1565 5610 1135 4982
{} 2532 2424 2064 1519 3614 758 2532 1492 834 2532 3793 2350
{} 3004 846 402 1063 2877 3756 599 235 2518 2532 2606 {} 846
{} 3753 1161 3793 1544 {} 1525 2902 846 5495 2532 2877 1453
{} 2532 1831 3778 5345 1519 3650 1565 1093
{} 2532 3855 1564 2424 190 846 1417 5185 2896 2532 3004 1653 2248 5207 1138
{} 1161 2064 1519 3614 4334 846 5185 2532 2424 846 3004 4100 3754 1410 5124 4160 3004 846 3483 2962
{} 5119 680 846 3788 3004 2596 5216 4102 1096 5213 1096
{} 2532 455 846 3788 2532 2424 846 1690 3004 3708 {} 3367 {} 1097 {}
{} 1161 3588 1831 1310 846 1722 3650 1565 1093
{} 1161 846 1831 2400 4374 846 2974 444 1139
{} 2532 1544 1140 2974 2980 2532 2296 3793 3004 3763 5316 3779 {} 1722 2474
{} 1161 5330 3004 1544 1140 1722 758 {} 1140
{} 2532 4013 2424 3956 4172 2532 2968 1321 1722 846 4864 2532 2784 2098 932 2532 2323 3956 3554 2532 3956 3119 1722 2992
{} 1161 1492 3793 4697 4012 846 3754 2258 4660 2532 4496 5616 4263 2192 3361 2192 4166
{} 5119 3004 846 3101 3303 2326 4183 1161 2040 3641
{} 1189 3767 2962 2326 3704 1544 2040 1519 846 2326
//
\glb
 A~ \underline{gdy wszedł} do łodzi, \doubleline{przeprawił się} i~ przybył do swojego miasta.
\vs{2} I~ oto przynieśli mu sparaliżowanego, leżącego na łożu. A~ Jezus, \underline{gdy zobaczył} ich wiarę, powiedział sparaliżowanemu: Odwagi, dziecko, \doubleline{odpuszczone są} tobie grzechy twoje.
\vs{3} A~ oto niektórzy \underline{ze znawców Pisma} pomyśleli w~ sobie: On znieważa.
\vs{4} A~ Jezus, \underline{gdy zobaczył} ich myśli, powiedział: Dlaczego Dlaczego wymyślacie wymyślacie zło w~ waszych sercach?
\vs{5} Co bowiem jest łatwiejsze, powiedzieć: \underline{Odpuszczone są} twoje grzechy, czy powiedzieć: Powstań i~ chodź?
\vs{6} Ale żebyście wiedzieli, że Syn Człowieczy ma na ziemi moc odpuszczania grzechów– wtedy powiedział sparaliżowanemu– Powstań, zabierz swe łoże i~ odejdź do swojego domu.
\vs{7} A~ \underline{gdy wstał,} poszedł do swego domu.
\vs{8} A~ tłumy, \underline{gdy} (to) \underline{zobaczyły,} \doubleline{dziwiły się} i~ \underline{oddały chwałę} Bogu, \doubleline{że dał} ludziom taką moc.
\vs{9} A~ odchodząc stamtąd, Jezus dostrzegł człowieka, siedzącego w~ \underline{miejscu pobierania podatków,} \doubleline{którego nazywano} Mateuszem, i~ powiedział mu: \underline{Idź za} mną. A~ \doubleline{gdy wstał,} \underline{zaczął iść za} nim.
\vs{10} A~ \underline{gdy on} \doubleline{leżał przy stole} w~ domu, \underline{stało się,} a~ oto liczni \doubleline{poborcy podatków} i~ grzesznicy, przyszedłszy, \underline{leżeli przy stole} \doubleline{z Jezusem} oraz jego uczniami.
\vs{11} A~ \underline{gdy zobaczyli} (to) faryzeusze, mówili jego uczniom: Dlaczego Dlaczego wasz nauczyciel je z~ \doubleline{poborcami podatków} i~ grzesznikami?
\vs{12} A~ Jezus, \underline{gdy} (to) \underline{usłyszał,} powiedział im: Nie zdrowi potrzebują potrzebują lekarza, ale \doubleline{ci, co się} źle \doubleline{mają.}
\vs{13} Idźcie i~ \underline{nauczcie się,} co (to) znaczy: Miłosierdzia chcę, a~ nie ofiary. Nie przyszedłem bowiem powołać do \doubleline{zmiany myślenia} sprawiedliwych, ale grzeszników.
\vs{14} Wtedy \underline{przyszli do} niego uczniowie Jana (i) mówili: Dlaczego Dlaczego my i~ faryzeusze dużo pościmy, a~ twoi uczniowie nie poszczą?
\vs{15} A~ Jezus powiedział im: Nie \underline{są w stanie} synowie \doubleline{domu weselnego} \underline{być w żałobie,} jak długo jest z~ nimi \doubleline{pan młody.} Ale przyjdą dni, kiedy \underline{zostanie odebrany} im im \doubleline{pan młody} i~ wtedy \underline{będą pościć.}
\vs{16} Przecież nikt (nie) nakłada naszywki (z) \underline{kawałka} surowej \underline{tkaniny} na starą \doubleline{szatę wierzchnią,} bowiem (to) jej uzupełnienie \underline{zrywa się} z~ \doubleline{szaty wierzchniej,} a~ rozdarcie \underline{staje się} gorsze.
\vs{17} \underline{Ani nie} \doubleline{wlewa się} młodego wina w~ stare \underline{skórzane bukłaki.} Bo inaczej inaczej \doubleline{skórzane bukłaki} \underline{się roztrzaskują} i~ wino \doubleline{się wylewa,} a~ \underline{skórzane bukłaki} \doubleline{się niszczą.} Ale \underline{wlewa się} młode wino w~ nowe \doubleline{skórzane bukłaki,} a~ obydwa \underline{się zachowują.}
\vs{18} (Gdy) on im to mówił, oto jeden (z) \underline{posiadających władzę} przyszedł (i) \doubleline{oddał} mu \doubleline{pokłon,} mówiąc, że: Moja córka \underline{przed chwilą} umarła, ale przyjdź, połóż na nią swą rękę, a~ \doubleline{będzie żyła.}
\vs{19} A~ Jezus powstawszy, \underline{poszedł za} nim, a~ (także) uczniowie jego.
\vs{20} A~ oto kobieta, krwawiąca \underline{od dwunastu} lat, \doubleline{gdy podeszła} \underline{z tyłu,} dotknęła krawędzi jego \doubleline{wierzchniej szaty.}
\vs{21} Myślała bowiem sobie: sobie: Gdybym tylko dotknęła jego \underline{wierzchniej szaty,} \doubleline{będę uratowana.}
\vs{22} A~ Jezus, \underline{gdy się obrócił} i~ ją dostrzegł, powiedział: Odwagi, córko, twoja wiara cię uratowała. I~ od tej chwili kobieta \doubleline{była uratowana.}
\vs{23} A~ Jezus, \underline{gdy przyszedł} do domu \doubleline{posiadającego władzę} i~ dostrzegł \underline{grających na aulosie} oraz tłum \doubleline{czyniący wrzawę,}
\vs{24} Powiedział im: Odejdźcie, bowiem dziewczynka nie umarła, ale śpi. I~ drwili (z) niego.
\vs{25} A~ gdy tłum \underline{został wyrzucony,} (Jezus) wszedł, chwycił jej rękę, a~ dziewczynka wstała.
\vs{26} I~ \underline{rozeszła się} ta wieść po całej tamtejszej ziemi.
\vs{27} A~ odchodzącemu stamtąd Jezusowi \underline{zaczęli towarzyszyć} \underline{zaczęli towarzyszyć} dwaj ślepi, \doubleline{którzy wołali} i~ mówili: \underline{Zlituj się} \doubleline{nad nami,} Synu Dawida.
\vs{28} A~ \underline{gdy przyszedł} do domu, \doubleline{podeszli do} niego ślepi, a~ Jezus im powiedział: Wierzycie, że \underline{jestem w stanie} to uczynić? Odpowiedzieli mu: Tak, Panie!
\vs{29} Wtedy dotknął ich oczy, mówiąc: \underline{Zgodnie z} waszą wiarą \doubleline{niech się} wam \doubleline{stanie}
\vs{30} I~ \underline{otworzyły się} ich oczy. A~ Jezus ich \doubleline{ostro upomniał,} mówiąc: Uważajcie, (aby) nikt (nie) \underline{dowiedział się} (o tym).
\vs{31} A~ oni, \underline{gdy wyszli,} rozpowszechnili to po całej tamtejszej ziemi.
\vs{32} A~ \underline{gdy oni} wychodzili, oto przyniesiono mu głuchego człowieka, opętanego.
\vs{33} A~ \underline{gdy został wyrzucony} demon, głuchy przemówił. I~ \doubleline{zdziwiły się} tłumy, mówiąc \underline{Nigdy nie} \doubleline{ukazała się} taka (rzecz) w~ Izraelu.
\vs{34} Lecz faryzeusze mówili: Wyrzuca demony poprzez \underline{posiadającego władzę} (nad) demonami.
\vs{35} I~ obchodził Jezus wszystkie miasta i~ wsie, nauczając w~ ich synagogach i~ ogłaszając \underline{dobrą nowinę} królestwa oraz uzdrawiając każdą chorobę i~ każdą niemoc wśród ludu.
\vs{36} A~ \underline{gdy dostrzegł} tłumy, \doubleline{ulitował się} nad nimi, bo byli utrudzeni i~ porzuceni, jak owce, \underline{które} nie \underline{mają} pasterza.
\vs{37} Wtedy powiedział swoim uczniom: Chociaż żniwo wielkie, ale robotników niewielu.
\vs{38} Proście więc Pana żniwa, aby wyrzucił robotników na swoje żniwo.
//
\endgl
\begingl
\lettrine[loversize=1,lraise=-1.3]{10 }{}%
\gla
 2532 4341 846 1427 3101 1325 846 1849 4151 169 5620 846 1544 2532 2323 3956 3554 2532 3956 3119
{} 1161 3686 1427 652 2076 5023 4413 4613 3004 4074 2532 846 80 406 2385 3588 {} 2199 2532 846 80 2491
{} 5376 2532 918 2381 2532 3156 5057 2385 3588 {} 256 2532 3002 1941 2280
{} 4613 2581 2532 2455 2469 3588 846 2532 3860
{} 5128 1427 649 2424 {} 3853 846 3004 1519 3598 1484 3361 565 2532 3361 1525 1519 4172 4541
{} 1161 4198 3123 4314 622 4263 3624 2474
{} 1161 4198 2784 3004 3754 1448 932 3772
{} 770 2323 3015 2511 1544 1140 1432 2983 1432 1325
{} 3361 2932 5557 3366 696 3366 5475 1519 5216 2223
{} 3361 4082 1519 3598 3366 1417 5509 3366 5266 3366 4464 1063 2040 514 2076 846 5160
{} 1161 302 {} 4172 2228 2968 1519 3739 1525 1833 5101 1722 846 2076 514 2546 3306 2193 302 1831
{} 1161 1525 1519 3614 782 846
{} 2532 1437 3303 3614 5600 514 2064 1515 5216 1909 846 1161 1437 3361 5600 514 5216 1515 4314 1994 5209
{} 2532 3739 3361 1437 1209 5209 3366 {} 191 3056 5216 1831 3614 2228 {} 1565 4172 1621 2868 {} 4228 5216
{} 281 3004 5213 414 2071 1093 4670 2532 1116 1722 2250 2920 2228 1565 4172
{} 2400 1473 5209 649 5613 4263 1722 3319 3074 1096 3767 5429 5613 3789 2532 185 5613 4058
{} 1161 4337 575 444 1063 3860 5209 1519 4892 2532 1722 846 4864 3146 5209 3146
{} 2532 1909 2232 1161 2532 935 71 1752 1700 1752 1519 3142 846 2532 1484
{} 1161 3752 5209 3860 3361 3309 4459 2228 5101 2980 1063 1325 5213 1722 1565 5610 5101 2980
{} 1063 3756 5210 2075 2980 235 4151 5216 3962 2980 1722 5213
{} 1161 80 3860 80 1519 2288 2532 3962 5043 2532 1881 5043 1909 1118 2532 2289 846
{} 2532 2071 3404 5259 3956 1223 3450 3686 1161 5278 1519 5056 3778 4982
{} 1161 3752 5209 1377 1722 3778 4172 5343 1519 243 281 1063 3004 5213 3756 3361 5055 {} 4172 2474 2193 302 2064 5207 444
{} 3101 3756 2076 5228 1320 3761 1401 5228 846 2962
{} 713 3101 2443 1096 5613 846 1320 2532 1401 5613 846 2962 1487 3617 2564 954 {} 4214 3123 846 3615
{} 3767 3361 5399 846 2076 1063 3762 2572 3739 3756 601 2532 2927 3739 3756 1097
{} 3739 3004 5213 1722 4653 2036 1722 5457 2532 3739 1519 3775 191 2784 1909 1430
{} 2532 3361 5399 575 615 4983 1161 5590 3361 1410 615 1161 5399 3123 1410 2532 5590 2532 4983 622 1722 1067
{} 3780 787 4453 1417 4765 2532 1520 1537 846 3756 4098 1909 1093 427 {} 5216 3962
{} 1161 2532 3956 2359 5216 2776 1526 705
{} 3767 3361 5399 5210 1308 4183 4765
{} 3956 3767 3748 3670 1722 1698 1715 444 2504 3670 1722 846 1715 3450 3962 1722 3772
{} 1161 3748 302 720 3165 1715 444 2504 720 846 1715 3450 3962 1722 3772
{} 3361 3543 3754 2064 906 1515 1909 1093 3756 2064 906 1515 235 3162
{} 1063 2064 1369 {} 444 {} 2596 846 3962 2532 2364 2596 846 3384 2532 3565 2596 846 3994
{} 2532 2190 444 3615 846
{} 5368 3962 2228 3384 5228 1691 3756 2076 3450 514 2532 5368 5207 2228 2364 5228 1691 3756 2076 3450 514
{} 2532 3739 3756 2983 846 4716 2532 190 3694 3450 3756 2076 3450 514
{} 3588 2147 846 5590 622 846 2532 3588 622 846 5590 1752 1700 2147 846
{} 3588 1209 5209 1691 1209 2532 3588 1209 1691 1209 3588 649 3165 649
{} 1209 4396 1519 3686 4396 2983 3408 4396 2532 1209 1342 1519 3686 1342 2983 3408 1342
{} 2532 3739 1437 4222 1520 {} 5130 3398 4221 5593 {} 3440 1519 3686 3101 281 3004 5213 3756 3361 622 846 3408
//
\glb
 A~ \underline{gdy przywołał} swoich dwunastu uczniów, dał im moc \doubleline{nad duchami} nieczystymi, by je wyrzucali oraz uzdrawiali każdą chorobę i~ każdą niemoc.
\vs{2} A~ imiona dwunastu apostołów są takie: pierwszym– Szymon, zwany Piotrem oraz jego brat, Andrzej; Jakub– ten, (który jest synem) Zebedeusza oraz jego brat, Jan;
\vs{3} Filip i~ Bartłomiej; Tomasz i~ Mateusz– \underline{poborca podatków;} Jakub– ten, (który jest synem) Alfeusza i~ Lebeusz, zwany Tadeuszem;
\vs{4} Szymon Kananejczyk i~ Juda Iskariota– \underline{ten, co} go również wydał.
\vs{5} Tę dwunastkę posłał Jezus (i) nakazał im, mówiąc: Na drogę narodów nie wchodźcie oraz nie wchodźcie do miast Samarytan.
\vs{6} Ale idźcie raczej do zagubionych owiec domu Izraela.
\vs{7} A~ idąc, głoście, mówiąc, że \underline{przybliżyło się} królestwo niebios.
\vs{8} Słabych uzdrawiajcie, trędowatych oczyszczajcie, wyrzucajcie demony. Darmo wzięliście, darmo dawajcie.
\vs{9} Nie posiadajcie złota ani srebra, ani miedzi w~ waszych pasach,
\vs{10} ani torby na drogę, ani dwóch \underline{szat spodnich,} ani sandałów, ani laski, ponieważ robotnik godny jest swego pokarmu.
\vs{11} A~ [-] (w) mieście lub wsi, do której wejdziecie, wypytajcie, kto w~ niej jest godny. \underline{I tam} pozostańcie, \doubleline{aż do} [-] wyjścia.
\vs{12} A~ \underline{gdy wejdziecie} do domu, pozdrówcie go.
\vs{13} I~ jeśli [-] dom \underline{okaże się} godny, \doubleline{niech przyjdzie} pokój wasz na niego. A~ gdyby nie był godny, wasz pokój \underline{niech powróci} do was.
\vs{14} A~ kto nie przyjąłby przyjąłby was ani (nie) wysłuchał słów waszych, \underline{wychodząc z} domu lub (z) tego miasta, strząśnijcie pył (ze) stóp waszych.
\vs{15} Zaprawdę, mówię wam: Lżej będzie ziemi Sodomy i~ Gomory w~ dniu sądu, niż temu miastu.
\vs{16} Oto ja was posyłam jak owce między między wilki. Bądźcie więc mądrzy, jak węże i~ czyści, jak gołębie.
\vs{17} I~ \underline{wystrzegajcie się} [-] ludzi, ponieważ \doubleline{będą wydawać} was przed sanhedryny i~ w~ swoich synagogach \underline{będą} was \underline{biczować.}
\vs{18} Oraz przed rządców [-] i~ królów \underline{będziecie prowadzeni} \doubleline{z~} mojego \doubleline{powodu,} na świadectwo im oraz narodom.
\vs{19} A~ gdy was wydadzą, nie \underline{troszczcie się,} jak lub co powiecie, \doubleline{bowiem będzie} dane wam w~ tej chwili, co \underline{macie mówić,}
\vs{20} ponieważ nie wy jesteście \underline{tymi, którzy mówią,} ale duch waszego Ojca, \doubleline{który mówi} w~ was.
\vs{21} A~ brat wyda brata na śmierć i~ ojciec dziecko, i~ powstaną dzieci przeciwko rodzicom, i~ zabiją ich.
\vs{22} I~ będziecie znienawidzeni przez wszystkich \underline{z powodu} mojego imienia. A~ \doubleline{kto wytrwa} do końca, ten \underline{zostanie wybawiony.}
\vs{23} A~ gdyby was prześladowali w~ tym mieście, uciekajcie do drugiego. Zaprawdę bowiem mówię wam, nie [-] skończycie (obchodzić) miast Izraela, aż [-] przyjdzie Syn Człowieczy.
\vs{24} Uczeń nie jest ponad nauczycielem, ani niewolnik ponad swego pana.
\vs{25} Wystarczające \underline{dla ucznia,} aby \doubleline{się stał} jak jego nauczyciel, i~ niewolnik– jak jego pan. Jeśli \underline{pana domu} nazywali Belzebubem, (to o) ile bardziej jego domowników.
\vs{26} Więc nie \underline{bójcie się} ich. \doubleline{Nie ma} bowiem nic ukrytego, co nie \underline{będzie objawione,} ani ukrytego, co nie \doubleline{będzie poznane.}
\vs{27} \underline{To, co} mówię wam w~ ciemności, powiedzcie w~ świetle, i~ co na ucho słyszycie, głoście na dachach.
\vs{28} I~ nie \underline{bójcie się} [-] \doubleline{tych, którzy zabijają} ciało, ale duszy nie \underline{są w stanie} zabić. Ale \doubleline{bójcie się} raczej \underline{tego, co może} i~ duszę, i~ ciało zniszczyć w~ Gehennie.
\vs{29} \underline{Czy nie} \doubleline{za asa} \underline{są sprzedawane} dwa wróble? Nawet jeden z~ nich nie spadnie na ziemię bez (woli) waszego Ojca.
\vs{30} A~ nawet wszystkie włosy \underline{na waszej} głowie są policzone.
\vs{31} \underline{A zatem:} nie \doubleline{bójcie się.} Wy \underline{jesteście ważniejsi,} \doubleline{niż liczne} wróble.
\vs{32} Każdy więc, kto \underline{przyzna się} do mnie przed ludźmi, \doubleline{i ja} \underline{się przyznam} do niego przed moim Ojcem w~ niebiosach.
\vs{33} A~ kto [-] \underline{by się wyparł} mnie przed ludźmi, \doubleline{i ja} \underline{się wyprę} go przed moim Ojcem w~ niebiosach.
\vs{34} Nie myślcie, że przyszedłem rzucić pokój na ziemię. Nie przyszedłem rzucić pokoju, ale miecz.
\vs{35} Ponieważ przyszedłem \underline{uczynić rozdwojenie,} (aby) człowiek (był) przeciwko swemu ojcu, i~ córka– przeciwko swojej matce, i~ \doubleline{panna młoda} przeciwko swojej teściowej,
\vs{36} a~ nieprzyjaciółmi człowieka domownicy jego.
\vs{37} \underline{Kto kocha} ojca lub matkę \doubleline{bardziej, niż} mnie, nie jest mnie godny. I~ \underline{kto kocha} syna lub córkę \doubleline{bardziej, niż} mnie, nie jest mnie godny.
\vs{38} I~ kto nie bierze swego krzyża, a~ idzie za mną, nie jest mnie godny.
\vs{39} Ten, \underline{który znalazł} swoją duszę, straci ją, a~ ten, \doubleline{kto stracił} swoją duszę \underline{z powodu} mnie, znajdzie ją.
\vs{40} Ten, \underline{kto przyjmuje} was, mnie przyjmuje; a~ ten, \doubleline{który przyjmuje} mnie, przyjmuje tego, \underline{który} mnie \underline{posłał.}
\vs{41} \underline{Kto przyjmuje} proroka \doubleline{ze względu na} imię proroka, weźmie zapłatę proroka, i~ \underline{ten, który przyjmuje} sprawiedliwego \doubleline{ze względu na} imię sprawiedliwego, weźmie zapłatę sprawiedliwego,
\vs{42} i~ kto by napoił jednego (z) tych małych kielichem zimnej (wody) jedynie \underline{ze względu na} imię ucznia, zaprawdę, mówię wam, nie [-] straci swojej zapłaty.
//
\endgl
